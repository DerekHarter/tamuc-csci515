% Created 2015-01-22 Thu 13:26
\documentclass[11pt]{article}
\usepackage[utf8]{inputenc}
\usepackage[T1]{fontenc}
\usepackage{fixltx2e}
\usepackage{graphicx}
\usepackage{longtable}
\usepackage{float}
\usepackage{wrapfig}
\usepackage{soul}
\usepackage{textcomp}
\usepackage{marvosym}
\usepackage{wasysym}
\usepackage{latexsym}
\usepackage{amssymb}
\usepackage{hyperref}
\tolerance=1000
\usepackage{array}
\usepackage{color}
\providecommand{\alert}[1]{\textbf{#1}}

\title{CSci 515 Setting Up Visual Studio Express 2010}
\author{}
\date{Spring 2015}
\hypersetup{
  pdfkeywords={},
  pdfsubject={Setting Up Visual Studio Express 2010},
  pdfcreator={Emacs Org-mode version 7.9.3f}}

\begin{document}

\maketitle


\section{Line Numbers}
\label{sec-1}


I prefer to always have line numbers enabled in my source code editor
as most compiler errors/warnings refer to specific line numbers in the
source file, and thus it is easier to quicly determine the potential
location of problems if line numbers are visible.  For Visual Studio
Express 2010 you can enable line numbers like this:

\begin{enumerate}
\item Tools $\rightarrow$ Options
\item In Options Dialog, select Text Editor $\rightarrow$ C/C++ $\rightarrow$ General
\item Enable the Line Numbers checkbox
\end{enumerate}
\section{Program Indentation and Tab Settings}
\label{sec-2}

For this class, you are required to correctly and consistently indent your code according to the
Deitel good programming guidelines, and our class coding/formatting guidelines.  You can have
Visual Studio Express 2010 automatically set your indentation for you by changing the following
settings:

\begin{enumerate}
\item Tools $\rightarrow$ Options
\item In Options Dialog, select Text Editor $\rightarrow$ C/C++ $\rightarrow$ Tabs
\item Set Indenting to Smart
\item Set Tab size to 2
\item Set indent size to 2
\item Select Insert Spaces (rather than Keep tabs)
\end{enumerate}

The last step will keep you from creating files with hardcoded,
embedded tabs, which will not display properly in other editors.
\section{Display/Undisplay Invisible (space) Characters}
\label{sec-3}

If you still have problems with getting hardcoded tabs in your source
files, most programming editors have a command that makes normally
invisible whitespace characters become visible, using symbols for the
different types of characters (space, tab, etc.).  To toggle the
display of invisible whitespace characters on and off, you can:
\subsection{Method 1}
\label{sec-3-1}

\begin{enumerate}
\item Turn on advanced settings from Tools $\rightarrow$ Settings $\rightarrow$ Expert Settings
\item Toggle display of white space on/off from Edit $\rightarrow$ Advanced $\rightarrow$ View Whitespace
\end{enumerate}
\subsection{Method 2}
\label{sec-3-2}

The keyboard shortcut for this toggle is Ctrl-R Ctrl-W.

\end{document}
