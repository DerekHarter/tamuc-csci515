% Created 2015-01-22 Thu 13:44
\documentclass[11pt]{article}
\usepackage[utf8]{inputenc}
\usepackage[T1]{fontenc}
\usepackage{fixltx2e}
\usepackage{graphicx}
\usepackage{longtable}
\usepackage{float}
\usepackage{wrapfig}
\usepackage{soul}
\usepackage{textcomp}
\usepackage{marvosym}
\usepackage{wasysym}
\usepackage{latexsym}
\usepackage{amssymb}
\usepackage{hyperref}
\tolerance=1000
\usepackage{array}
\usepackage{color}
\providecommand{\alert}[1]{\textbf{#1}}

\title{CSci 515 Setting Up Visual Studio Express 2010}
\author{}
\date{Spring 2015}
\hypersetup{
  pdfkeywords={},
  pdfsubject={Setting Up Visual Studio Express 2010},
  pdfcreator={Emacs Org-mode version 7.9.3f}}

\begin{document}

\maketitle


\section{Line Numbers}
\label{sec-1}


I prefer to always have line numbers enabled in my source code editor
as most compiler errors/warnings refer to specific line numbers in the
source file, and thus it is easier to quicly determine the potential
location of problems if line numbers are visible.  For Visual Studio
Express 2010 you can enable line numbers like this:

\begin{enumerate}
\item Tools $\rightarrow$ Options
\item In Options Dialog, select Text Editor $\rightarrow$ C/C++ $\rightarrow$ General
\item Enable the Line Numbers checkbox
\end{enumerate}
\section{Program Indentation and Tab Settings}
\label{sec-2}

For this class, you are required to correctly and consistently indent your code according to the
Deitel good programming guidelines, and our class coding/formatting guidelines.  You can have
Visual Studio Express 2010 automatically set your indentation for you by changing the following
settings:

\begin{enumerate}
\item Tools $\rightarrow$ Options
\item In Options Dialog, select Text Editor $\rightarrow$ C/C++ $\rightarrow$ Tabs
\item Set Indenting to Smart
\item Set Tab size to 2
\item Set indent size to 2
\item Select Insert Spaces (rather than Keep tabs)
\end{enumerate}

The last step will keep you from creating files with hardcoded,
embedded tabs, which will not display properly in other editors.
\section{Display/Undisplay Invisible (space) Characters}
\label{sec-3}

If you still have problems with getting hardcoded tabs in your source
files, most programming editors have a command that makes normally
invisible whitespace characters become visible, using symbols for the
different types of characters (space, tab, etc.).  To toggle the
display of invisible whitespace characters on and off, you can:
\subsection{Method 1}
\label{sec-3-1}

\begin{enumerate}
\item Turn on advanced settings from Tools $\rightarrow$ Settings $\rightarrow$ Expert Settings
\item Toggle display of white space on/off from Edit $\rightarrow$ Advanced $\rightarrow$ View Whitespace
\end{enumerate}
\subsection{Method 2}
\label{sec-3-2}

The keyboard shortcut for this toggle is Ctrl-R Ctrl-W.
\section{Cause Terminal/Console to Persist after Program Execution}
\label{sec-4}

You are forbidden to include statements in your submitted programs
whose sole purpose is to keep the terminal from closing upon program
completion, before you can see the output.  Statements like
\verb~system("pause")~ or \verb~getch()~ are hacks, and are often not portable
to other enviornments or IDE systems.  When using Visual Studio Express 2010
you can have your IDE keep the console up in the following 2 ways:
\subsection{Method 1}
\label{sec-4-1}

Simple use of debugger.  You can simply set a debug breakpoint on the
closing brace of your main function.  Then whenever you run your
program (using Debug $\rightarrow$ Start Debugging or equivalently
using the \verb~F5~ function key), your program will run till it hits this
breakpoint, and you can then look at the terminal output.
\subsection{Method 2}
\label{sec-4-2}

You can also set a property for console applications, so that if you
run without debugging, Visual Studio keeps the terminal open until
you press a key.  You need to set the following property for each
project you create (Visual Studio doesn't enable this by default
on new projects):

\begin{enumerate}
\item Project $\rightarrow$ Properties
\item In the Project Property Pages dialog select Configuration Properties $\rightarrow$ Linker $\rightarrow$ System
\item In the Linker System options, pull down the SubSystem property and select Console (/SUBSYSTEM:CONSOLE)
\item Hit OK or Apply to have your property setting saved (for this particular Project only, you have to do this for every project you create).
\end{enumerate}

Now if you run your program with debugging (Debug $\rightarrow$ Start
Without Debugging or equivalently \verb~Shift-F5~), your terminal will be
paused when the program completes execution.
\section{Increase/Decrease Editor Font Size}
\label{sec-5}

For readability, you can increase/decrease the font size of the Visual
Studio programming editor. Use \verb/Ctrl-Shift-,~ to increase font size, and ~Ctrl-Shift-./ to decrease text size.  You can also directly set
the zoom level in the lower left of the editor frame.

\end{document}
