% Created 2015-02-25 Wed 12:42
\documentclass[11pt]{article}
\usepackage[utf8]{inputenc}
\usepackage[T1]{fontenc}
\usepackage{fixltx2e}
\usepackage{graphicx}
\usepackage{longtable}
\usepackage{float}
\usepackage{wrapfig}
\usepackage{rotating}
\usepackage[normalem]{ulem}
\usepackage{amsmath}
\usepackage{textcomp}
\usepackage{marvosym}
\usepackage{wasysym}
\usepackage{amssymb}
\usepackage{hyperref}
\tolerance=1000
\usepackage{minted}
\usepackage{minted}
\usemintedstyle{default}
\author{CSci 515 Spring 2015}
\date{\textit{<2015-02-24 Tue>}}
\title{Test 01}
\hypersetup{
  pdfkeywords={},
  pdfsubject={Test 01 Spring 2015},
  pdfcreator={Emacs 24.3.1 (Org mode 8.2.4)}}
\begin{document}

\maketitle

\section*{Dates:}
\label{sec-1}
\begin{center}
\begin{tabular}{ll}
Due: & In Lab, Wednesday March 11, by 4:05 pm (lab end time)\\
\end{tabular}
\end{center}
\section*{Description}
\label{sec-2}
You have 2 hours (our regularly scheduled lab time) to complete the
following tasks.  Create a single file, named test01.cpp, in a visual
studio project called Test01.  Set up the Visual Studio projects using
the normal settings for our class and labs, as we have practiced for
the past 7 weeks.

Perform the following tasks:

\begin{enumerate}
\item In the main function of your program, write an index controlled
loop that uses an index to execute 10 times.  Inside of your loop,
generate a random number between 1 and 4 (inclusive), and display
the random number to standard output.  If the number is a 1,
display a special message "I generated a 1!".  The output from this
task and loop in your main function should look like the first part
of the example output given below.

\item After the previous task in your main function, write code to
perform the following task.  I have given you a file named
"test-01-data.txt".  This file has a single integer value on each
line.  Open this file, and inside of a sentinel controlled loop,
read in the values from the file and display the values to standard
output.  Your sentinel controlled loops should simply be a while
loop that executes until the read from the file fails.  The output
from reading the give file will be exactly as given in the example
output shown below.

\item Write a function named \verb~calculateHypotenuse~.  The function will
take two floating point values as parameters, named \verb~a~ and \verb~b~.
The function should calculate the hypotenuse of a right triangle
using the \verb~a~ and \verb~b~ parameters, according to this formula: $$ c =
   \sqrt{a^2 + b^2} $$ e.g. the hypotenuse is the square root of
adding together the square of a and b.  This value should be
calculated and returned (as a floating point result) as a result of
calling your function.  In your main function, show an example of
calling your function with values of $3.0$ and $4.0$ for the \verb~a~
and \verb~b~ parameters.

\item Create an array of integers of size $10$ named values.  Create a
function named \verb~initNegative~ that takes an array of integers and a
size as its two input parameters.  This will be a void function (it
will return no result).  This functions should initialize each
value to be equal to the negative of the index, e.g. the value at
index $0$ will be $0$, the value at index $1$ will be $-1$, the
value at index $2$ will be $-2$, etc.  Call your function in
\verb~main()~ to initialize the array.  In \verb~main~, after initializing
the array, write a loop that displays the contents of the array.
An example of the output from this task is shown below as the last
part of the example output.
\end{enumerate}

Your program output for the 4 previous tasks should look something
close to the following when I run your program:

\begin{verbatim}
----- Task 1 -----
3
1
I generated a 1!
2
2
1
I generated a 1!
3
2
1
I generated a 1!
4
4

----- Task 2 -----
Read value from file: 3
Read value from file: 1
Read value from file: 42
Read value from file: 9
Read value from file: 11
Read value from file: 12
Read value from file: 7

----- Task 3 -----
Hypotenuse of triangle with sides 3 and 4: 5

----- Task 4 -----
0
-1
-2
-3
-4
-5
-6
-7
-8
-9
\end{verbatim}
\section*{Test Submission}
\label{sec-3}

An eCollege dropbox has been created for this test.  You should upload
your version of the test by the end of test time to the eCollege
dropbox named \verb~Test 01~.  Work submitted by the end of the allotted
time will be considered, but after the test ends you may no longer
submit work, so make sure you submit your best effort by the test end
time in order to receive credit.
\section*{Requirements and Grading Rubrics}
\label{sec-4}

\subsection*{Program Execution, Output and Functional Requirements}
\label{sec-4-1}

\begin{enumerate}
\item Your program must compile, run and produce some sort of output to
be graded. You will loose at least 1/3 of the total points (33) if
your program does not compile and run when submitted.
\item 10 pts (1 letter grade).  Up to 1 letter grade will be awarded for
formatting and style issues for the test.  Your program must meet
(most) all of the standard class style/formatting guidelines that
we have been practicing and using in our labs and assignments for
this course.
\item 20 pts. Task 1.  You must use an index controlled for loop, and
have an if statement.  Your output for this task must be as
shown in the example output.
\item 20 pts.  Task 2.  You must successfully open up and read from the
given file.  You should use defensive programming to detect when
the file is not opened or found correctly.  You must use a sentinel
controlled loop to read all values from the file.  Your output
for task 2 must look like that shown in the example output.
\item 25 pts. Task 3.  You must correctly name and define the function as
required for the task.  The function must accept the correct
parameters as input, and return the correct result type.  The
function must be implemented correctly to perform the desired
calculation.
\item 25 pts.  Task 4. You must correctly name and define the function as
required for this task.  The function should take the array and the
array size as parameters and initialize the array as specified.
You must define the array in your main loop and invoke the function
with your array to be initialized.  You should use a defined
constant to specify the size of the array in main.  Your should
display your array after being initialized, as shown in the example
output.
\end{enumerate}

\subsection*{Program Style}
\label{sec-4-2}

Your programs must conform to the style and formatting guidelines given for this course.
The following is a list of the guidelines that are required for the lab to be submitted
this week.

\begin{enumerate}
\item The file header for the file with your name and program information
and the function header for your main function must be present, and
filled out correctly.
\item A function header must be present for all functions you define.
You must document the purpose, input parameters and return values
of all functions.  Your function headers must be formatted exactly
as shown in the style guidelines for the class.
\item You must indent your code correctly and have no embedded tabs in
your source code. (Don't forget about the Visual Studio Format
Selection command).
\item You must not have any statements that are hacks in order to keep
your terminal from closing when your program exits (e.g. no calls
to system() ).
\item You must have a single space before and after each binary operator.
\item You must have a single blank line after the end of your declaration
of variables at the top of a function, before the first code
statement.
\item You must have a single blank space after , and \verb~;~ operators used as a
separator in lists of variables, parameters or other control
structures.
\item You must have opening \verb~{~ and closing \verb~}~ for control statement blocks
on their own line, indented correctly for the level of the control
statement block.
\item All control statement blocks (if, for, while, etc.) must have \verb~{~
\verb~}~ enclosing them, even when they are not strictly necessary
(when there is only 1 statement in the block).
\begin{enumerate}
\item You should attempt to use meaningful variable and function names in
your program, for program clarity.  Of course, when required, you
must name functions, parameters and variables as specified in the
assignments.  Variable and function names must conform to correct
\verb~camelCaseNameingConvention~ .
\end{enumerate}
\end{enumerate}

Failure to conform to any of these formatting and programming practice
guidelines for this test will result in loosing 1 letter grade You can
get a B for this test if you do it perfectly, but have bad or missing
style/formatting.  To get an A, however, you need to follow (most) of
the style/formatting requirements for this course on your test code.
% Emacs 24.3.1 (Org mode 8.2.4)
\end{document}
