% Created 2015-02-24 Tue 16:44
\documentclass[11pt]{article}
\usepackage[utf8]{inputenc}
\usepackage[T1]{fontenc}
\usepackage{fixltx2e}
\usepackage{graphicx}
\usepackage{longtable}
\usepackage{float}
\usepackage{wrapfig}
\usepackage{rotating}
\usepackage[normalem]{ulem}
\usepackage{amsmath}
\usepackage{textcomp}
\usepackage{marvosym}
\usepackage{wasysym}
\usepackage{amssymb}
\usepackage{hyperref}
\tolerance=1000
\usepackage{minted}
\usepackage{minted}
\usemintedstyle{default}
\author{CSci 515 Spring 2015}
\date{\textit{<2015-02-24 Tue>}}
\title{Test 01}
\hypersetup{
  pdfkeywords={},
  pdfsubject={Test 01 Spring 2015},
  pdfcreator={Emacs 24.3.1 (Org mode 8.2.4)}}
\begin{document}

\maketitle

\section*{Dates:}
\label{sec-1}
\begin{center}
\begin{tabular}{ll}
Due: & In Lab, Wednesday March 4, by 4:05 pm (lab end time)\\
\end{tabular}
\end{center}
\section*{Description}
\label{sec-2}
You have 2 hours (our regularly scheduled lab time) to complete the
following tasks.  Create a single file, named test01.cpp, in a visual
studio project called Test01.  Set up the Visual Studio projects using
the normal settings for our class and labs, as we have practiced for
the past 7 weeks.

Perform the following tasks:

\begin{enumerate}
\item In the main function of your program, write a statement that
prompts the user for a seed value.  Seed the C standard library
random number generator with this seed value using the \verb~srand~
standard library function.  Write an index controlled loop that
uses an index to execute 10 times.  Inside of your loop, generate a
random number between 1 and 4, and display the random number to
standard output.  If the number is a 1, display a special message
"I generated a 1!".  The output from this task and loop in your
main function should look like the first part of the example
output given below.

\item After the previous task in your main function, write code to
perform the following task.  I have given you a file named
"test-01-data.txt".  This file has a single integer value
one each line.  Open this file, and inside of a sentinel 
controlled loop, read in the values from the file and
display the values to standard output.  Your sentinel
controlled loops should simply be a while loop that executes
until the read from the file fails.  The output from
reading the give file will be exactly as given in 
the example output shown below.

\item Write a function named \verb~calculateHypotenuse~.  The function 
will take two floating point values as parameters, named \verb~a~
and \verb~b~.  The function should calculate the hypotenuse of 
a right triangle using the \verb~a~ and \verb~b~ parameters, according
to this formula:
\end{enumerate}
$$
c = \sqrt{a^2 + b^2}
$$
  e.g. the hypotenuse is the square root of adding together the
  square of a and b.  This value should be calculated and returned
  (as a floating point result) as a result of calling your function.
  In your main function, show an example of calling your function with
  values of $3.0$ and $4.0$ for the \verb~a~ and \verb~b~ parameters.

\begin{enumerate}
\item Create an array of integers of size $10$ named values.  Initialize
all of values to $0$ in a loop in your main function.  Create a
function named \verb~initNegative~ that takes an array of integers
and a size as its two input parameters.  This functions
should initialize each value to be equal to the negative
of the index, e.g. the value at index $0$ will be $0$, the
value at index $1$ will be $-1$, the value at index $2$ will be
$-2$, etc.  Show an example of calling your function in your
\verb~main()~ function.
\end{enumerate}

Your program output should look something close to the following when I
run your program:

\begin{verbatim}
I will display a range of values from our array.
Enter index to start at: 5
Enter index to end at: 15
values[005] 0.52743745
values[006] 0.58105505
values[007] 0.16465282
values[008] 0.28310502
values[009] 0.59936452
values[010] 0.87698799
values[011] 0.34905022
values[012] 0.03936137
values[013] 0.22738582
values[014] 0.59080577
values[015] 0.87810069
\end{verbatim}


\textbf{NOTE}: Now that our programs have more functions than just the
\verb~main()~ function, the use of the function headers becomes meaningful
and required.  Make sure that all of your functions have function
headers preceding them that document the purpose of the functions, and
the input parameters and return value of the function.
\section*{Lab Submission}
\label{sec-3}

An eCollege dropbox has been created for this lab.  You should
upload your version of the lab by the end of lab time to the eCollege
dropbox named \verb~Lab 06 Processing Arrays~.  Work submitted by the end
of lab will be considered, but after the lab ends you may no longer
submit work, so make sure you submit your best effort by the lab end
time in order to receive credit.
\section*{Requirements and Grading Rubrics}
\label{sec-4}

\subsection*{Program Execution, Output and Functional Requirements}
\label{sec-4-1}

\begin{enumerate}
\item Your program must compile, run and produce some sort of output to be
graded. 0 if not satisfied.
\item 40+ pts.  Your program must have the required named function,
that accepts the required input parameters and return the required
values (if any).
\item 20+ pts. Your \verb~displayArrayValues~ function must correctly format
the displayed output on standard output.  Your program should work
if the begin and end range are equal, and should show now output
when begin is greater than the end specified.
\item 20+ pts.  You must use I/O formatting to correctly display the
output index ranges of the arrays as shown.
\item 20+ pts. Your main function must prompt the user as specified, and
display the output formatted correctly as shown.
\end{enumerate}

\subsection*{Program Style}
\label{sec-4-2}

Your programs must conform to the style and formatting guidelines given for this course.
The following is a list of the guidelines that are required for the lab to be submitted
this week.

\begin{enumerate}
\item The file header for the file with your name and program information
and the function header for your main function must be present, and
filled out correctly.
\item A function header must be present for all functions you define.
You must document the purpose, input parameters and return values
of all functions.  Your function headers must be formatted exactly
as shown in the style guidelines for the class.
\item You must indent your code correctly and have no embedded tabs in
your source code. (Don't forget about the Visual Studio Format
Selection command).
\item You must not have any statements that are hacks in order to keep
your terminal from closing when your program exits (e.g. no calls
to system() ).
\item You must have a single space before and after each binary operator.
\item You must have a single blank line after the end of your declaration
of variables at the top of a function, before the first code
statement.
\item You must have a single blank space after , and \verb~;~ operators used as a
separator in lists of variables, parameters or other control
structures.
\item You must have opening \verb~{~ and closing \verb~}~ for control statement blocks
on their own line, indented correctly for the level of the control
statement block.
\item All control statement blocks (if, for, while, etc.) must have \verb~{~
\verb~}~ enclosing them, even when they are not strictly necessary
(when there is only 1 statement in the block).
\item You should attempt to use meaningful variable and function names in
your program, for program clarity.  Of course, when required, you
must name functions, parameters and variables as specified in the
assignments.  Variable and function names must conform to correct
\verb~camelCaseNameingConvention~ .
\end{enumerate}

Failure to conform to any of these formatting and programming practice
guidelines for this lab will result in at least 1/3 of the points (33)
for the assignment being removed for each guideline that is not
followed (up to 3 before getting a 0 for the assignment). Failure to
follow other class/textbook programming guidelines may result in a
loss of points, especially for those programming practices given in
our Deitel textbook that have been in our required reading so far.
% Emacs 24.3.1 (Org mode 8.2.4)
\end{document}
