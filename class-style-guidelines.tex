% Created 2015-01-21 Wed 10:03
\documentclass[11pt]{article}
\usepackage[utf8]{inputenc}
\usepackage[T1]{fontenc}
\usepackage{fixltx2e}
\usepackage{graphicx}
\usepackage{longtable}
\usepackage{float}
\usepackage{wrapfig}
\usepackage{soul}
\usepackage{textcomp}
\usepackage{marvosym}
\usepackage{wasysym}
\usepackage{latexsym}
\usepackage{amssymb}
\usepackage{hyperref}
\tolerance=1000
\usepackage{array}
\usepackage{color}
\providecommand{\alert}[1]{\textbf{#1}}

\title{CSci 515 Coding Style Guidelines}
\author{}
\date{Spring 2015}
\hypersetup{
  pdfkeywords={},
  pdfsubject={Class Coding Style Guidelines},
  pdfcreator={Emacs Org-mode version 7.9.3f}}

\begin{document}

\maketitle


\section*{Overview}
\label{sec-1}


Learning correct style and formatting standards for written programs
are an important part of become a good computer scientist and software
engineer.  Style and formatting guidelines are to programming as
grammer, punctuation and formatting rules are to written languages.
They help you to communicate the intention and meaning of your
programs to both other readers (other programmers and software
engineers) as well as your future self trying to read and understand
code you may have written sometime in the distant past.

In general, you must follow good programming practice tips and
specifications given in our Deitel textbook for this course.  Failure
to follow follow a good programming formatting practice as specificed
in our textbook may cost you points on your lab and programming
assignments.  In general, if your follow these guidelines and format
your code to look and conform to the style shown in your textbook,
then you will not have points removed for style issues.  However, the
following are additional issues and guidelines that you must also
follow, that may be slightly different or that we will emphasize from
the Deital style formatting and guidelines.
\section*{File header documentation}
\label{sec-2}


All assignment files you create for this course must have a file
document header included at the beginning of the source code file.  We
will use a pseudo docoxygen format, and ask you to provide the
following information at the beginning of each of your source files
submitted for assignment for this course:


\begin{verbatim}
/** 
 * @author Joe Student
 * @cwid   123 45 678
 * @assg   Assignment #1
 * @ide    Visual Studio Express 2013
 * @date   January 20, 2015
 *
 * @description Provide a short description of the problem and the approach you 
 *              took to solving the problem.
 */
\end{verbatim}
\section*{Function header documentation}
\label{sec-3}


In a similar manner, all functions you write for this course should
include a function header declaration and documentation using (pseudo)
doxoxygen formatted comments.  Almost all programming shops require
that all classes and functions (including member functions) be
documented in this manner where they are declared.  The purpose is
mainly to document the input parameters to the function, and any
output results the function returns.  Use the following format to
document all functions you write for assignment for this course:


\begin{verbatim}
/** Check for process arrivals
 * Check the process table information to find processes that are
 * arriving.  Place all arriving processes at the end of the round
 * robin scheduling queue.
 *
 * @param currentTime An int value, the current time step of the
 *              simulation, check for processes in table arriving at
 *              this time.
 * @param processTable A pointer to a ProcessTable struct, a list of
 *              all the process information for processes we are
 *              simulating, including their arrival times.
 * @param rrQueue A STL list holding pointers to Process struct items.
 *              This is our simulated round robin queue. This parameter
 *              is passed as a reference parameter.  We will add
 *              any arriving processes to the end of this queue as a
 *              side effect of calling this function, since it is a
 *              reference parameter.
 * @returns bool True if a process arrived, false otherwise 
 */
bool checkProcessArrivals(int currentTime, ProcessTable* processTable, 
                            list<Process*>& rrQueue)
{
  for (int pid=0; pid < processTable->numProcesses; pid++)
  {
    Process* p = processTable->process[pid];
    if (p->arrivalTime == currentTime)
    {
      DEBUG(cout << "   Process arrives: " 
                 << p->processName << endl;)
      rrQueue.push_back(p);
      return True
    }
  }

  return False
}
\end{verbatim}
\section*{Indentation}
\label{sec-4}


Make sure you pay special attention to the Deital style guidelines
regarding proper indentation of course code.  For this course we
require you to use spaces (no embedded tabs) for indentation.  You are
required to use 2 spaces as the unit of indentation for all code/block
levels for code submitted for this class.  DO NOT use hard coded
(embdeded) tabs in your submitted programs.
\section*{Function and Variable Names}
\label{sec-5}


All functions and variables should follow camelCaseNameing convention
for your programs for this course.  When creating a user defined type,
like a class or a struct, you should use CamelCaseNameing with the
initial letter capitalized (this differentiates functions and
variables from classes and user defined types). Constants, in
enumerated types or otherwise, should use ALL$_{\mathrm{CAPS}}$$_{\mathrm{UNDERSCORE}}$
convention for nameing.  Make sure you choose meaningful variable,
function, class and constant names for your programs, as meaningful
well chosen names make your programs more readable and reduce the need
for extensive comments.  See Deitel guidelines for more hints on how
to choose meaningful variable names for your programs.
\section*{Brace Placement for Control Blocks}
\label{sec-6}


For this course you are required to place (most) all braces defining a
control block (like a for loop or if statement) on a separate line by
themselves, indented appropriately.  For example, this function has 2
levels of indentation, and all levels are consistently indented and
all opening/closing braces appear on their own line for readability:


\begin{verbatim}
/** Display a matrix
 * A helper function for debugging.  Display a state matrix to
 * standard output
 *
 * @param rows The number of rows in the matrix
 * @param cols The number of cols in the matrix
 * @param m A 2 dimensional array of rows x cols integers
 */
void displayMatrix(int rows, int cols, int v[MAX_PROCESSES][MAX_RESOURCES])
{
  int r, c;

  // display column headers
  cout << "   "; // extra space over for row labels
  for (c = 0; c < cols; c++)
  {
    cout << "R" << c << " ";
  }
  cout << endl;

  // now display data in matrix
  for (r = 0; r < rows; r++)
  {
    cout << "P" << r << " ";
    for (c = 0; c < cols; c++)
    {
      cout << setw(2) << v[r][c] << " ";
    }
    cout << endl;
  }
  cout << endl;
}
\end{verbatim}
\section*{Whitespace}
\label{sec-7}


Pay special attention to the programming guidelines regarding space
within statements and between blocks of code and functions.  For
example, always put a single space before and after all binary
operators (like +, -, <<, etc.).  Put a single space after commas (,)
and semicolons (;) separating lists of parameters or declarations in
functions/control blocks.  But a single blank line before and after a
control block inside of a function.  For this course, you should place
2 blank lines between the end of each function and the beginning of
the next function documentation.  In general, pay attention to the
formatting of whitespace in our Deitel textbook and follow the
conventions shown there.

\end{document}
