% Created 2015-01-20 Tue 18:56
\documentclass[11pt]{article}
\usepackage[utf8]{inputenc}
\usepackage[T1]{fontenc}
\usepackage{fixltx2e}
\usepackage{graphicx}
\usepackage{longtable}
\usepackage{float}
\usepackage{wrapfig}
\usepackage{rotating}
\usepackage[normalem]{ulem}
\usepackage{amsmath}
\usepackage{textcomp}
\usepackage{marvosym}
\usepackage{wasysym}
\usepackage{amssymb}
\usepackage{hyperref}
\tolerance=1000
\usepackage{minted}
\author{Derek Harter}
\date{\today}
\title{deitel-good-programming-practice}
\hypersetup{
  pdfkeywords={},
  pdfsubject={},
  pdfcreator={Emacs 24.3.1 (Org mode 8.2.4)}}
\begin{document}

\maketitle
\tableofcontents

\section{Chapter 1}
\label{sec-1}

\begin{itemize}
\item 1.1 Write your C++ programs in a simple and straightforward
manner. This is sometimes referred to as KISS (“keep it simple
stupid”). Do not “stretch” the language by trying bizarre usages.
\item 2.1 Every program should begin with a comment that describes the
purpose of the program.
\item 2.2 Use blank lines, space characters and tabs to enhance program readability.
\item 2.3 Indent the entire body of each function one level within the
braces that delimit the body of the function. This makes a program’s
functional structure stand out and makes the program easier to read.
\item 2.4 Set a convention for the size of indent you prefer, then apply
it uniformly. The tab key may be used to create indents, but tab
stops may vary. We recommend using either 1/4 inch tab stops or
(preferably) three spaces to form a level of indent. (I require 2
spaces for all indentation).
\item 2.5 Place a space after each comma ( , ) to make programs more readable.
\item 2.6 Choosing meaningful identifiers makes a program
self-documenting—a person can understand the program simply by
reading it rather than having to refer to manuals or comments.
\item 2.7 Avoid using abbreviations in identifiers. This promotes program
readability.
\item 2.8 Avoid identifiers that begin with underscores and double
underscores, because C++ compilers may use names like that for their
own purposes internally. This will prevent names you choose from
being confused with names the compilers choose.
\item 2.9 Always place a blank line between a declaration and adjacent
executable statements. This makes the declarations stand out in the
program and contributes to program clarity.
\item 2.10 Place spaces on either side of a binary operator. This makes
the operator stand out and makes the program more readable.
\item 2.11 Using redundant parentheses in complex arithmetic expressions
can make the expressions clearer.
\item 2.12 Indent the statement(s) in the body of an if statement to
enhance readability.
\item 2.13 For readability, there should be no more than one statement per
line in a program.
\item 2.14 A lengthy statement may be spread over several lines. If a
single statement must be split across lines, choose meaningful
breaking points, such as after a comma in a comma-separated list, or
after an operator in a lengthy expression. If a statement is split
across two or more lines, indent all subsequent lines and left-align
the group of indented lines.
\item 2.15 Refer to the operator precedence and associativity chart when
writing expressions containing many operators. Confirm that the
operators in the expression are performed in the order you
expect. If you are uncertain about the order of evaluation in a
complex expression, break the expression into smaller statements or
use parentheses to force the order of evaluation, exactly as you’d
do in an algebraic expression. Be sure to observe that some
operators such as assignment ( = ) associate right to left rather
than left to right.
\item 3.2 Choosing meaningful function names and meaningful parameter
names makes programs more readable and helps avoid excessive use of
comments.
\item 3.3 Place 2 blank lines between member/function definitions to
enhance program readability.
\item 4.1 Consistently applying reasonable indentation conventions
throughout your programs greatly improves program readability. We
suggest two blanks per indent. Some people prefer using tabs, but
these can vary across editors, causing a program written on one
editor to align differently when used with another.
\item 4.2 Whatever indentation convention you choose should be applied
consistently throughout your programs. It’s difficult to read
programs that do not obey uniform spacing conventions.
\item 4.3 Indent both body statements of an if \ldots{} else statement.
\item 4.4 If there are several levels of indentation, each level should be
indented the same additional amount of space to promote readability
and maintainability.
\item 4.5 Always putting the braces in an if \ldots{} else statement (or any
control statement) helps prevent their accidental omission,
especially when adding statements to an if or else clause at a later
time. To avoid omitting one or both of the braces, some programmers
prefer to type the beginning and ending braces of blocks even before
typing the individual statements within the braces.
\item 4.6 Separate declarations from other statements in functions with a
blank line for readability.
\item 4.7 Declare each variable on a separate line with its own comment
for readability.
\item 4.9 Unlike binary operators, the unary increment and decrement
operators should be placed next to their operands, with no
intervening spaces.
\item 5.1 Put a blank line before and after each control statement to make
it stand out in the program.
\item 5.2 Too many levels of nesting can make a program difficult to
understand. As a rule, try to avoid using more than three levels of
indentation.
\item 5.3 Vertical spacing above and below control statements and
indentation of the bodies of control statements give programs a
two-dimensional appearance that improves readability.
\item 5.10 Provide a default case in switch statements. Cases not
explicitly tested in a switch statement without a default case are
ignored. Including a default case focuses you on the need to process
exceptional conditions. There are situations in which no default
processing is needed. Although the case clauses and the default case
clause in a switch statement can occur in any order, it’s common
practice to place the default clause last.
\item 6.1 Capitalize the first letter of an identifier used as a
user-defined type name. (e.g. a Class or Struct name, for example)
\item 6.2 Use only uppercase letters in constant names. This makes these
constants stand out in a program and reminds you that constants are
not variables.
\item 7.2 Defining the size of an array as a constant variable instead of
a literal constant makes programs clearer. This technique eliminates
so-called magic numbers. For example, repeatedly mentioning the size
10 in array-processing code for a 10-element array gives the number
10 an artificial significance and can be confusing when the program
includes other 10s that have nothing to do with the array size.
\end{itemize}
% Emacs 24.3.1 (Org mode 8.2.4)
\end{document}
