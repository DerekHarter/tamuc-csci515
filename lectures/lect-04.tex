% Created 2015-02-06 Fri 16:40
\documentclass[11pt]{article}
\usepackage[utf8]{inputenc}
\usepackage[T1]{fontenc}
\usepackage{fixltx2e}
\usepackage{graphicx}
\usepackage{longtable}
\usepackage{float}
\usepackage{wrapfig}
\usepackage{soul}
\usepackage{textcomp}
\usepackage{marvosym}
\usepackage{wasysym}
\usepackage{latexsym}
\usepackage{amssymb}
\usepackage{hyperref}
\tolerance=1000
\usepackage{minted}
\providecommand{\alert}[1]{\textbf{#1}}

\title{Lecture 04 Notes}
\author{Derek Harter}
\date{2015-02-06}
\hypersetup{
  pdfkeywords={},
  pdfsubject={Lecture 04 Notes.},
  pdfcreator={Emacs Org-mode version 7.9.3f}}

\begin{document}

\maketitle


\section{First Session (11 - 11:40)}
\label{sec-1}
\subsection{Functional Programming}
\label{sec-1-1}

\begin{itemize}
\item 6.1 To promote software reusability, every function should be limited to performing a single
  well define task, and the name of the function should express the task effectively.
\item 6.2 If you cannot choose a concise name that expresses a function's task, you function might
  be attempting to perform too many diverse tasks.  It's usually best to break such a function
  into several smaller functions.
\end{itemize}

Functions serve several purposes from the point of view of the
developer:

\begin{enumerate}
\item Functions encapsulate an algorithm or procedure.
\item Functions can be black boxes that help reduce complexity.  You can
   build more complex algorithms on top of simple functions, without
   worrying about the details of how the functions you use are
   implemented.
\item Functions allow for reuse of code, and help to avoid code repetition.
\end{enumerate}

For example, it can be very complex to write an algorithm to compute
mathematical functions, such as the sin() trigonometric function, on a
computer.  However, there are standard libraries of math functions
that you can use that implement the sin() function for you.  When you
use a library that someone else has written, you are using the
functions as black boxes.  You know what the function should take as
input, and what its result should be.  But you do not need to concern
yourself with the messy details.  Likewise, using standard libraries
are examples of a type of reuse.  The functions are written one time,
but reused over and over again in many programs.

As a programmer you can get the same benefits by writing functions
yourself.  You can write a function once, and reuse it in many
locations in your code.  This is an example of a good programming
practice known as Don't Repeat Yourself (the DRY principle).  You
should (almost) never have code repeated in 2 or more places.  Once
you have the urge to repeat a piece of code, you have probably
identified a small procedure or task that should be encapsulated
inside of a function, that can be called when needed.

Likewise user defined functions help manage complexity when writing
complex algorithms.  There is a design strategy known as functional
decomposition.  Basically, when approaching a problem in this manner,
you try and break down the ultimate task into smaller sub-tasks.  If
those sub-tasks are complex, you further break them down into
sub-sub-tasks.  You repeat this decomposition until you have defined
problems that are simple enough to understand and code directly as
small functions.  Then you solve the larger problems by stringing
together and reusing your functions.  It takes skill and practice to
learn how best to do such decomposition, but it is fundamental to
learning how to solve real problems that are often complex and messy.
Functional decomposition allows you to encapsulate the messiness of
the details in black boxes, as functions, then forget about that
messiness to solve larger issues, on up till you address your ultimate
problem.
 
\subsection{Standard Libraries of Functions}
\label{sec-1-2}


\begin{itemize}
\item The math library
\item Need to include the header to use a standard library \#include <cmath>
\item Headers for libraries of standard functions basically define the
  function signature.  We will go into more details about functions
  signatures, also called function prototypes, in a bit.
\item Use some of these functions.
\item Show Fig 6.7 (pg 219), list of some of the standard libraries in C/C++.
\end{itemize}
\subsection{Defining a Function}
\label{sec-1-3}

\begin{itemize}
\item Function name (choose meaningful name)
\item Function Input (the parameters to the function)
\item Function Output (the return value)
\item Always include a function header documentation.  Document purpose,
  parameters and return value.
\item Implement pow(), fabs(), maybe ceil() and floor()
\end{itemize}
\section{Second Session (11:45 - 12:30)}
\label{sec-2}
\subsection{Function prototypes}
\label{sec-2-1}

\begin{itemize}
\item The signature of a function is its prototpye.
\item To use a function, alls you need is the signature.
\item The actual implementation can be somewhere else.
\item In fact, put signatures in a header file, implementations in a separate source file, and (re)use.
\item This is the basic method used for the standard libraries of functions.
\end{itemize}
\subsection{Variable and Argument Coercion}
\label{sec-2-2}


\begin{itemize}
\item C promotes certain numeric values in mixed-type expressions.  Type
  is promoted to the ``highest'' type in the expression.
\item \textbf{BEWARE} of the so called integer division error.
\item Passing a parameter to a function can also be automatically coerced in this way.
\end{itemize}
\section{Third Session (12:40 - 1:40)}
\label{sec-3}
\subsection{Scope Rules}
\label{sec-3-1}
\subsection{Function Call Stack}
\label{sec-3-2}

\end{document}
