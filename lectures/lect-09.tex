% Created 2015-03-31 Tue 21:19
\documentclass[11pt]{article}
\usepackage[utf8]{inputenc}
\usepackage[T1]{fontenc}
\usepackage{fixltx2e}
\usepackage{graphicx}
\usepackage{longtable}
\usepackage{float}
\usepackage{wrapfig}
\usepackage{rotating}
\usepackage[normalem]{ulem}
\usepackage{amsmath}
\usepackage{textcomp}
\usepackage{marvosym}
\usepackage{wasysym}
\usepackage{amssymb}
\usepackage{hyperref}
\tolerance=1000
\usepackage{minted}
\author{Derek Harter}
\date{CSci 515 Spring 2015 \textit{<2015-03-31 Tue>}}
\title{Lecture 09 Notes}
\hypersetup{
  pdfkeywords={},
  pdfsubject={Lecture 09 Notes.},
  pdfcreator={Emacs 24.3.1 (Org mode 8.2.4)}}
\begin{document}

\maketitle

\section{First Session (11 - 11:40)}
\label{sec-1}
\subsection{Introduction to Analysis of Algorithms}
\label{sec-1-1}
Analysis of algorithms is a branch of computer science that studies
the performance of algorithms, especially their run time and space
requirements.  Most classes tend to emphasize the run time complexity,
but space complexity can be just as important, e.g if your program
can't fit into the size of primary memory for a supercomputing
application, if often will go from being feasible to compute, to
impossible.

\begin{itemize}
\item The practical goal if analysis of algorithms is to predict the
performance of different algorithms, as a function of the input size,
in order to guide design decisions.
\item The goal is to make meaningful comparisons between algorithms, but
some problems:
\begin{itemize}
\item The relative performance might depend on characteristics of hardware.
Alg 1 faster machine A, Alg 2 faster machine B.  Specify a machine model.
\item Relative performance might depend on details of dataset.  Thus usually
look at best, worst and average case behavior.
\item Relative importance depends on the size of the input.  A sorting
algorithm will be slower for a small list than for a very large one.
\end{itemize}
\item Usual solution for last: specify run time (or number of operations)
as a function of problem size.  Compare functions \textbf{asymptotically} 
as the problem size increases.
\end{itemize}

Suppose you have analyzed two algorithms and expressed their run times in
terms of the size of the input: Algorithm A takes $100n + 1$ steps
to solve a problem with size $n$; Algorithm B takes $n^2 + n + 1$ steps.

Which algorithm is "better" in terms of time complexity (number of steps)?

The following table shows run time of algorithms for different problem sizes:

\begin{center}
\begin{tabular}{rrr}
Input & Run time of & Run time of\\
size & Algorithm A & Algorithm B\\
\hline
10 & 1 001 & 111\\
100 & 10 001 & 10 101\\
1 000 & 100 001 & 1 001 001\\
10 000 & 1 000 001 & $> 10^{10}$\\
\end{tabular}
\end{center}

At $n = 10$ Algorithm A looks pretty bad, it takes 10 times longer
than B.  But for $n = 100$ they are about the same, and for larger
values of $n$, Algorithm A will be much better.  

The fundamental reason is that for large values of $n$, any function
that contains an $n^2$ term will grow faster than a function whose
leading term is $n$.  The \textbf{leading term} is the term with the
highest exponent.

For Algorithm A the leading term has a large coefficient, 100, which
is why B does better than A for small $n$. But regardless of constant
coefficients like this, there will always be some point, some value of
$n$ where $a n^2 > b n$.

The same argument applies to the non-leading terms, which is why we
can ignore the $n + 1$ part of the $n^2$ algorithm B.  Even if the
run time of Algorithm A were $n + 1000000$, it would still do better
than Algorithm B for sufficiently large $n$.

In general we expect an algorithm with a smaller leading term to be
a better algorithm for large problems, but for smaller problems there
may be a \textbf{crossover point} where another algorithm is better.
\subsection{Order of Growth and the Big-Oh $O()$ Notation}
\label{sec-1-2}

An \textbf{order of growth} is a set of functions whose asymptotic growth behavior
is considered equivalent.  For example, $2n$, $100n$ and $n + 1$ belong to
the same order of growth, which is written $O(n)$ in \textbf{Big-Oh notation} and
often called \textbf{linear} because every function in the set grows linearly with
$n$.

All functions with the leading term $n^2$ belong to the $O(n^2)$;
they are \textbf{quadratic}, which is a fancy word for functions with
the leading term $n^2$.

The following table shows some of the orders of growth that appear most commonly
in algorithm analysis, in increasing order of badness.

\begin{center}
\begin{tabular}{ll}
Order of Growth & Name\\
\hline
$O(1)$ & Constant\\
$O(\mathrm{log}_b \; n)$ & logarithmic (for any b)\\
$O(n)$ & linear\\
$O(n \; \mathrm{log}_b \; n)$ & "en log en"\\
$O(n^2)$ & quadratic\\
$O(n^3)$ & cubic\\
$O(c^n)$ & exponential (for any c)\\
\end{tabular}
\end{center}
\section{Second Session (11:45 - 12:30)}
\label{sec-2}
\subsection{Searching Algorithms}
\label{sec-2-1}
\begin{itemize}
\item Why is Binary Search \$O(\mathrm{log}$_{\text{2}}$ $\backslash$; n)
\end{itemize}

\subsection{Sorting Arrays}
\label{sec-2-2}
\begin{itemize}
\item Bubble Sort
\item Insertion Sort
\item Merge Sort
\end{itemize}

\section{Third Session (12:40 - 1:40)}
\label{sec-3}

\begin{center}
\begin{tabular}{llll}
\textbf{Algorithm} & \textbf{Best case} & \textbf{Expected} & \textbf{Worst case}\\
Bubble sort & $O(N^2)$ & $O(N^2)$ & $O(N^2)$\\
Insertion sort & $O(N^2)$ & $O(N^2)$ & $O(N^2)$\\
Merge Sort & $O(N log N)$ & $O(N log N)$ & $O(N log N)$\\
Linear search & $O(1)$ & $O(N)$ & \$O(N)\\
Binary search & $O(1)$ & $O(log N)$ & $O(log N)$\\
\end{tabular}
\end{center}
% Emacs 24.3.1 (Org mode 8.2.4)
\end{document}
