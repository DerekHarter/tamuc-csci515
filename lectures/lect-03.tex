% Created 2015-01-28 Wed 19:35
\documentclass[11pt]{article}
\usepackage[utf8]{inputenc}
\usepackage[T1]{fontenc}
\usepackage{fixltx2e}
\usepackage{graphicx}
\usepackage{longtable}
\usepackage{float}
\usepackage{wrapfig}
\usepackage{rotating}
\usepackage[normalem]{ulem}
\usepackage{amsmath}
\usepackage{textcomp}
\usepackage{marvosym}
\usepackage{wasysym}
\usepackage{amssymb}
\usepackage{hyperref}
\tolerance=1000
\usepackage{minted}
\author{Derek Harter}
\date{CSci 515 Spring 2014 \textit{<2015-01-28 Wed>}}
\title{Lecture 03 Notes}
\hypersetup{
  pdfkeywords={},
  pdfsubject={Lecture 03 Notes.},
  pdfcreator={Emacs 24.3.1 (Org mode 8.2.4)}}
\begin{document}

\maketitle

\section{First Session (11 - 11:40)}
\label{sec-1}
\subsection{Streams}
\label{sec-1-1}
The IOStream library is a new (object oriented) library, added with the C++
language, to support Input and Output to source and destination devices.

The source of input can be a keyboard, a file, or some other device.  Likewise
the destination of output can be to a file, to a terminal screen, or to some
other device (for example you can send output into another C variable, like
a string in memory).

A stream is a way of visualizing how data is transferred from the source to 
destination.  A stream is inherently serial, the order in which you put things
into the stream, is the order they will be received when they come out of the
stream.
\subsection{iostream header}
\label{sec-1-2}
You've already seen many examples of specifying the iostream header using

\begin{minted}[linenos=true,frame=lines,numbersep=5pt,fontsize=\footnotesize]{c}
#include <iostream>
\end{minted}

Iostream operators and objects are defined in the std namespace, thus you
explicitly have to specify \verb~std::~ before using them, or include the

\begin{minted}[linenos=true,frame=lines,numbersep=5pt,fontsize=\footnotesize]{c}
using namespace std;
\end{minted}

directive.

In addition to iostream, if you want to do I/O to files, you need to include
the fstream header.  If you want to manipulate and format the data in/out of
the stream, you need to include the iomanip header.
\subsection{Standard Stream Objects}
\label{sec-1-3}

\begin{itemize}
\item cin, cout input from the standard input device, and output to the
standard error device respectively.  These are the keyboard and
terminal, by default, but can be connected to others (like a file)
by the OS, and program doesn't know or care.
\item cerr send output to the standard error device, can be useful for
separating error messages from normal output (and redirecting
standard error to a different location).  By default, standard error
also goes to the terminal.
\item clog also connects to the standard error output in a buffered
manner.  You don't need to be concerned with clog in this class.
\end{itemize}
\subsection{Stream Output and Input}
\label{sec-1-4}

\begin{itemize}
\item using the \verb~<<~ \verb~>>~  stream notation
\end{itemize}
\begin{minted}[linenos=true,frame=lines,numbersep=5pt,fontsize=\footnotesize]{c}
cout << x << y << z;
cin >> x << y << z;
\end{minted}

\begin{itemize}
\item Using member functions.  The streams cout, cin, are objects, they have
member functions.  For example, and put and get single characters:
\end{itemize}

\begin{minted}[linenos=true,frame=lines,numbersep=5pt,fontsize=\footnotesize]{c}
cout.put('A').put('\n');
cin.get(c);
\end{minted}

\begin{itemize}
\item Example of reading a character at a time of input and echoing until EOF
\end{itemize}

\begin{minted}[linenos=true,frame=lines,numbersep=5pt,fontsize=\footnotesize]{c}
int c; // use int, because char cannot represent EOF
while ( (character = cin.get()) != EOF)
{
  cout.put(character);
}
\end{minted}

\begin{itemize}
\item Example of reading input a line at a time
\end{itemize}

\begin{minted}[linenos=true,frame=lines,numbersep=5pt,fontsize=\footnotesize]{c}
const int SIZE = 80;
char buffer[SIZE];

cin.getline(buffer, SIZE);
\end{minted}

\begin{itemize}
\item peek, putback and ignore can be used for low level I/O.  We can
ignore a number of characters, and we can peek ahead (without
reading) or putback a character into the stream.
\end{itemize}


\section{Second Session (11:45 - 12:30)}
\label{sec-2}

\section{Third Session (12:40 - 1:40)}
\label{sec-3}
% Emacs 24.3.1 (Org mode 8.2.4)
\end{document}
