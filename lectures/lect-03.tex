% Created 2015-01-28 Wed 20:16
\documentclass[11pt]{article}
\usepackage[utf8]{inputenc}
\usepackage[T1]{fontenc}
\usepackage{fixltx2e}
\usepackage{graphicx}
\usepackage{longtable}
\usepackage{float}
\usepackage{wrapfig}
\usepackage{rotating}
\usepackage[normalem]{ulem}
\usepackage{amsmath}
\usepackage{textcomp}
\usepackage{marvosym}
\usepackage{wasysym}
\usepackage{amssymb}
\usepackage{hyperref}
\tolerance=1000
\usepackage{minted}
\author{Derek Harter}
\date{CSci 515 Spring 2014 \textit{<2015-01-28 Wed>}}
\title{Lecture 03 Notes}
\hypersetup{
  pdfkeywords={},
  pdfsubject={Lecture 03 Notes.},
  pdfcreator={Emacs 24.3.1 (Org mode 8.2.4)}}
\begin{document}

\maketitle

\section{First Session (11 - 11:40)}
\label{sec-1}
\subsection{Streams}
\label{sec-1-1}
The IOStream library is a new (object oriented) library, added with the C++
language, to support Input and Output to source and destination devices.

The source of input can be a keyboard, a file, or some other device.  Likewise
the destination of output can be to a file, to a terminal screen, or to some
other device (for example you can send output into another C variable, like
a string in memory).

A stream is a way of visualizing how data is transferred from the source to 
destination.  A stream is inherently serial, the order in which you put things
into the stream, is the order they will be received when they come out of the
stream.
\subsection{iostream header}
\label{sec-1-2}
You've already seen many examples of specifying the iostream header using

\begin{minted}[linenos=true,frame=lines,numbersep=5pt,fontsize=\footnotesize]{c}
#include <iostream>
\end{minted}

Iostream operators and objects are defined in the std namespace, thus you
explicitly have to specify \verb~std::~ before using them, or include the

\begin{minted}[linenos=true,frame=lines,numbersep=5pt,fontsize=\footnotesize]{c}
using namespace std;
\end{minted}

directive.

In addition to iostream, if you want to do I/O to files, you need to include
the fstream header.  If you want to manipulate and format the data in/out of
the stream, you need to include the iomanip header.
\subsection{Standard Stream Objects}
\label{sec-1-3}

\begin{itemize}
\item cin, cout input from the standard input device, and output to the
standard error device respectively.  These are the keyboard and
terminal, by default, but can be connected to others (like a file)
by the OS, and program doesn't know or care.
\item cerr send output to the standard error device, can be useful for
separating error messages from normal output (and redirecting
standard error to a different location).  By default, standard error
also goes to the terminal.
\item clog also connects to the standard error output in a buffered
manner.  You don't need to be concerned with clog in this class.
\end{itemize}
\subsection{Stream Output and Input}
\label{sec-1-4}

\begin{itemize}
\item using the \verb~<<~ \verb~>>~  stream notation
\end{itemize}
\begin{minted}[linenos=true,frame=lines,numbersep=5pt,fontsize=\footnotesize]{c}
cout << x << y << z;
cin >> x << y << z;
\end{minted}

\begin{itemize}
\item Using member functions.  The streams cout, cin, are objects, they have
member functions.  For example, and put and get single characters:
\end{itemize}

\begin{minted}[linenos=true,frame=lines,numbersep=5pt,fontsize=\footnotesize]{c}
cout.put('A').put('\n');
cin.get(c);
\end{minted}

\begin{itemize}
\item Example of reading a character at a time of input and echoing until EOF
\end{itemize}

\begin{minted}[linenos=true,frame=lines,numbersep=5pt,fontsize=\footnotesize]{c}
int c; // use int, because char cannot represent EOF
while ( (character = cin.get()) != EOF)
{
  cout.put(character);
}
\end{minted}

\begin{itemize}
\item Example of reading input a line at a time
\end{itemize}

\begin{minted}[linenos=true,frame=lines,numbersep=5pt,fontsize=\footnotesize]{c}
const int SIZE = 80;
char buffer[SIZE];

cin.getline(buffer, SIZE);
\end{minted}

\begin{itemize}
\item peek, putback and ignore can be used for low level I/O.  We can
ignore a number of characters, and we can peek ahead (without
reading) or putback a character into the stream.
\end{itemize}
\subsection{Formatted I/O, Precision and float output}
\label{sec-1-5}

\begin{itemize}
\item We can display integers in any base,  using hex, oct, dec and setbase()
\item For float/double types, use cout.precision(p) or setprecision(p) 
manipulator to set number of decimal places shown.
\item For float/double, you can force fixed or scientific notation output
using those output stream manipulators.
\end{itemize}
\subsection{Field width}
\label{sec-1-6}

\begin{itemize}
\item For strings (both inputs and outputs) can limit output using
cxx.width(w) or the setw(w) io manipulator.
\end{itemize}
\subsection{Other Formatting}
\label{sec-1-7}

\begin{itemize}
\item Figure 15.12 in section 15.7 shows a lot of other useful manipulators,
including to show as uppercase or lowercase, skipping white-space on input,
show or don't show the base for integers, etc.
\end{itemize}
\subsection{Left and right justify strings, padding}
\label{sec-1-8}
\begin{itemize}
\item use left and right manipulators to justify a sting output in a field.
\item needs to be used in conjunction with setw(w) manipulator normally.
\item Can pad out strings (instead of using whitespace when justifying)
by using setfill('x')
\end{itemize}
\subsection{Boolean output representation}
\label{sec-1-9}
\begin{itemize}
\item Can have booleans output as true/false (rather than 0/1) using boolalpha
manipulator.
\end{itemize}


\section{Second Session (11:45 - 12:30)}
\label{sec-2}

\subsection{File Processing}
\label{sec-2-1}
For the most part, all of the stream I/O we have seen can be done to
and from a file that you open and specify (instead of the standard
input/output device).  This is more complicated for a binary file we
want to randomly access, but we will first look at opening a plain
text stream, and reading/writing it sequentially.
\subsection{Creating a Sequential File}
\label{sec-2-2}

\begin{itemize}
\item need to include fstream libray to open files for read/write
\item At most basic, can open a file for output as
\end{itemize}

\begin{minted}[linenos=true,frame=lines,numbersep=5pt,fontsize=\footnotesize]{c}
ofstream outFile("name-of-output-file.txt");
\end{minted}

\begin{itemize}
\item And can open a file for input using:
\end{itemize}

\begin{minted}[linenos=true,frame=lines,numbersep=5pt,fontsize=\footnotesize]{c}
ifstream inFile("name-of-input-file.txt");
\end{minted}
\subsection{Writing/Reading Data from a sequential File}
\label{sec-2-3}

\begin{itemize}
\item An open output stream file using sequential access, can be written
\end{itemize}
to using the name we just created.  For example:

\begin{minted}[linenos=true,frame=lines,numbersep=5pt,fontsize=\footnotesize]{c}
outFile << x << y << z;
\end{minted}

\begin{itemize}
\item Likewise we can read input from a simple sequential file we opened:
\end{itemize}

\begin{minted}[linenos=true,frame=lines,numbersep=5pt,fontsize=\footnotesize]{c}
inFile >> x >> y >> z;
\end{minted}

\begin{itemize}
\item HINT: You need to be careful that you know what file you are opening
and where it is located on your filesystem.  I require that you
always check whether the open of the file was successful after
opening it:
\end{itemize}

\begin{minted}[linenos=true,frame=lines,numbersep=5pt,fontsize=\footnotesize]{c}
ofstream outFile("name-of-input-file.txt");

// exit program if unable to create a file
if (!outFile)
{
  cerr << "File name-of-input-file.txt could not be opened, file not found error." << endl;
  exit(1);
}
\end{minted}

\begin{itemize}
\item Also, never hardcode an absolute path to a file.  Always use a relative path name.
\item Once you successfully open a file for read/write, you can use any of
the iostream methods we have talked about to format the input/output
from/to the file.
\end{itemize}
\subsection{Reading Lines of Data from a File}
\label{sec-2-4}
\subsection{Reading comma separated values (CSV)}
\label{sec-2-5}
\subsection{Random Access Files}
\label{sec-2-6}
\begin{itemize}
\item Create using \verb~ios::binary~ specifier
\item use \verb~seekp()~ and write() read() member functions to get data in and out.
\item \verb~seekp()~ need to specify a byte location to move to in file, need to
calculate correctly or bad things will result.
\item Advantages of binary files, more compact, faster to access (because of this)
\item But overrated.  Plain text files are human readable and editable.
Storage is cheap, programmer time is expensive.
\end{itemize}
\section{Third Session (12:40 - 1:40)}
\label{sec-3}

\subsection{stdio.h, old school C I/O}
\label{sec-3-1}
\begin{itemize}
\item \verb~printf()~ function to do output and formatted output.
\end{itemize}

\begin{minted}[linenos=true,frame=lines,numbersep=5pt,fontsize=\footnotesize]{c}
printf("x = %d, f = %f, c = %c\n", someInt, someFloat, someChar)
\end{minted}

\begin{itemize}
\item Discuss table of format specifiers
\item \verb~scanf()~ to do (formatted) input
\item uses same table of format specifiers
\item \textbf{NEVER} mix iostream and stdio.h I/O in the same program.  Pick one
or the other and stick with it.  It is confusing to switch between
them, and possibly can introduce bugs if accessing for example the
standard input stream using the 2 different libraries.
\end{itemize}
\subsection{File I/O with stdio.h}
\label{sec-3-2}

\begin{itemize}
\item use \verb~fopen()~  to create a file handle than can be used for reading and
writing.
\end{itemize}

\begin{minted}[linenos=true,frame=lines,numbersep=5pt,fontsize=\footnotesize]{c}
FILE *fileHandle;

fileHandle = fopen("file-name-to-open.txt", "r");  // or "w" for an output file to write to
\end{minted}

\begin{itemize}
\item when done with file, always use \verb~fclose(fileHandle);~
\item can pass a file handle to \verb~fprintf()~ and \verb~fscanf()~ functions to read
and write formatted input/output from/to files.
\end{itemize}

\begin{minted}[linenos=true,frame=lines,numbersep=5pt,fontsize=\footnotesize]{c}
fprintf(outputFileHandle, "x = %d, f = %f, c = %d\n", x, f, c); // output values to plain text file
\end{minted}

\begin{itemize}
\item again similar syntax for input using \verb~scanf()~
\end{itemize}
% Emacs 24.3.1 (Org mode 8.2.4)
\end{document}
