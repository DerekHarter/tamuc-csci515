% Created 2015-02-24 Tue 13:47
\documentclass[11pt]{article}
\usepackage[utf8]{inputenc}
\usepackage[T1]{fontenc}
\usepackage{fixltx2e}
\usepackage{graphicx}
\usepackage{longtable}
\usepackage{float}
\usepackage{wrapfig}
\usepackage{rotating}
\usepackage[normalem]{ulem}
\usepackage{amsmath}
\usepackage{textcomp}
\usepackage{marvosym}
\usepackage{wasysym}
\usepackage{amssymb}
\usepackage{hyperref}
\tolerance=1000
\usepackage{minted}
\author{Derek Harter}
\date{CSci 515 Spring 2015 \textit{<2015-02-18 Wed>}}
\title{Lecture 07 Notes}
\hypersetup{
  pdfkeywords={},
  pdfsubject={Lecture 07 Notes.},
  pdfcreator={Emacs 24.3.1 (Org mode 8.2.4)}}
\begin{document}

\maketitle

\section{First Session (11 - 11:40)}
\label{sec-1}
\subsection{Multidimensional Arrays}
\label{sec-1-1}

\begin{itemize}
\item Two-dimensional arrays represent \textbf{TABLES} of values
\item Tables are a very fundamental data structure/concept.  Represent
the basic type of a relational database, for example.
\item Information arranged in rows and columns (we have seen this before
in our lab where we looked at DSV separated data).
\item By convention, the first identifier for a 2-D array specifies the
table's row, and the second identifier specifies the column.
\item This makes sense, as the row is the \textbf{record} or \textbf{trial} of a table
e.g. it is the set of related information.
\item The column is a particular attribute or piece of information about
a record or trial.
\item So, to initialize a 2-D array:
\end{itemize}

\begin{minted}[linenos=true,frame=lines,numbersep=5pt,fontsize=\footnotesize]{c}
const int NUM_RECORDINGS = 5; // number of rows in table
const int NUM_DIMENSIONS = 3; // number of columns in table

// Each recording records a 3-dimensional position of a particle in
// our experiment
float experimentPositions[NUM_RECORDS][NUM_DIMENSIONS];
\end{minted}

\begin{itemize}
\item For 1-D arrays, the size of the array was not required by compiler
when passing array (by reference) to a function.  Though we always
specified the size of the array as the next parameter for our functions.

\item For 2-D arrays, the number of rows (the first dimension) is still not
required by compiler, though again we will almost always pass this
information in to the function because it is needed to process the
2-D array.
\item However, the compiler needs the number of columns to be specified, 
because reasons.
\begin{itemize}
\item reasons: arrays are stored using row major ordering, thus we need to
know the number of columns in order to correctly calculate item
position.
\end{itemize}
\item 
\end{itemize}
\section{Second Session (11:45 - 12:30)}
\label{sec-2}
\subsection{More array Operations}
\label{sec-2-1}
\begin{itemize}
\item Shifting elements in an array
\item Add element to an array
\item Remove element from an array
\end{itemize}

\section{Third Session (12:40 - 1:40)}
\label{sec-3}
% Emacs 24.3.1 (Org mode 8.2.4)
\end{document}
