% Created 2015-02-13 Fri 12:49
\documentclass[11pt]{article}
\usepackage[utf8]{inputenc}
\usepackage[T1]{fontenc}
\usepackage{fixltx2e}
\usepackage{graphicx}
\usepackage{longtable}
\usepackage{float}
\usepackage{wrapfig}
\usepackage{soul}
\usepackage{textcomp}
\usepackage{marvosym}
\usepackage{wasysym}
\usepackage{latexsym}
\usepackage{amssymb}
\usepackage{hyperref}
\tolerance=1000
\usepackage{minted}
\providecommand{\alert}[1]{\textbf{#1}}

\title{Lecture 05 Notes}
\author{Derek Harter}
\date{2015-02-13}
\hypersetup{
  pdfkeywords={},
  pdfsubject={Lecture 05 Notes.},
  pdfcreator={Emacs Org-mode version 7.9.3f}}

\begin{document}

\maketitle


\section{First Session (11 - 11:40)}
\label{sec-1}
\subsection{Scope Rules}
\label{sec-1-1}

\begin{itemize}
\item Variables declared in functions are local to the function.
\item Variables (and constants) declared outside of functions are global.
\item You probably should NOT be using global variables.  Global variables allow
  state information to leak between functions.
\item Global constants, however, are often a good idea and useful.
\item Understanding the scope of variables is important.  Limiting scope of
  variables is important.
\end{itemize}
\subsection{Passing Parameters by Reference}
\label{sec-1-2}

\begin{itemize}
\item Variables are passed to functions by value by default.
\item This means, the value is copied, and if you change value in the
  function, the value is not changed in the caller.
\item The only way to get a value back to the caller, is to return it
  as the return result from the function.
\item However, sometimes we need to return more than 1 value, or sometimes
  for efficiency reasons (e.g. need to return 1 million values, we probably
  don't really want to copy them all back).
\item In this case, we can pass in values by reference.
\item A reference parameter, if changed in the function, will be changed
  for the caller who provided it.  In effect, a reference parameter is
  not a copy, but it IS the actual variable from the caller, so
  changes to it will be accessible to the caller when the function
  returns.
\end{itemize}
\subsection{Random Number Generation}
\label{sec-1-3}

\begin{itemize}
\item \verb~rand()~ and \verb~srand()~ functions
\item Included in the C standard library \verb~<cstdlib>~ \verb~<stdlib.h>~
\item \verb~RAND_MAX~
\item \verb~rand()~ generates an integer from 0 to max
\item \verb~srand()~ setting the seed.
\item Random number generator function is \textbf{pseudo random}, a sequence
\item Can repeat a sequence of random numbers.
\item How do we flip a coin sided die?
\item Write a function that flips a coin.
\item Write a function that returns random number in arbitrary range.
\item How do we generate a floating point number randomly?
\item Write a function generate a float over range 0.0 to 1.0 with uniform probability.
\end{itemize}
\section{Second Session (11:45 - 12:30)}
\label{sec-2}
\subsection{Recursion}
\label{sec-2-1}

\begin{itemize}
\item A function that calls itself (either directly or indirectly).
\item Some problems are much more easily (succinctly) stated or solved as a
  recursive relationship.
\item \textbf{base case(s)} are extremely important.  All recursive functions must have 
  1 or more base cases.  The base cases are the cases that are directly solvable.
  These are the pieces we know the answer to, or can determine easily.
\item The recursive cases then are those that we don't know directly, but that we solve
  usually by breaking apart the problem, and specifying/solving in terms of easier
  subparts.  We can call our function itself on these smaller subproblems, and combine
  the answer in some way to solve the more complex problem.  This is the
  \textbf{recursive call} or the \textbf{recursion step}.
\item Example, calculate the factorial of a number
\item Write an iterative function to calculate factorial first.
\item Write recursive function to calculate factorial.
\item Example, the Fibonacci sequence
\item Write an iterative function implementation first.
\item Write a recursive function to calculate Fibonacci sequence
\end{itemize}
\section{Third Session (12:40 - 1:40)}
\label{sec-3}
\subsection{Function Call Stack}
\label{sec-3-1}

\begin{itemize}
\item Example, functions A(), B() and C()
\item Recursive
\item Some advice/examples on using the Visual Studio debugger.
\end{itemize}

\end{document}
