% Created 2015-01-23 Fri 12:01
\documentclass[11pt]{article}
\usepackage[utf8]{inputenc}
\usepackage[T1]{fontenc}
\usepackage{fixltx2e}
\usepackage{graphicx}
\usepackage{longtable}
\usepackage{float}
\usepackage{wrapfig}
\usepackage{soul}
\usepackage{textcomp}
\usepackage{marvosym}
\usepackage{wasysym}
\usepackage{latexsym}
\usepackage{amssymb}
\usepackage{hyperref}
\tolerance=1000
\usepackage{minted}
\providecommand{\alert}[1]{\textbf{#1}}

\title{Lecture 02 Notes}
\author{Derek Harter}
\date{2015-01-20}
\hypersetup{
  pdfkeywords={},
  pdfsubject={Lecture 02 Notes.},
  pdfcreator={Emacs Org-mode version 7.9.3f}}

\begin{document}

\maketitle


\section*{Part 1 (11 - 11:40)}
\label{sec-1}
\subsection*{Memory Concepts, Variables, Representation}
\label{sec-1-1}


\begin{itemize}
\item When you declare a variable, it is assigned a location in memory.
  All references to store or access the named variable are translated
  by the C compiler using a lookup table.
\item The datatype is how pattern of bits in memory are stored and
  interpreted to mean a number, a character, etc.
\item Int32 representation (\href{http://en.wikipedia.org/wiki/Integer_%28computer_science%29}{http://en.wikipedia.org/wiki/Integer\_\%28computer\_science\%29})
\item Debug $\rightarrow$ QuickWatch
\item Debug $\rightarrow$ Windows $\rightarrow$ Memory
\item Representation of integer 1 vs. -1
\end{itemize}
\subsection*{Arithmetic and Precedence}
\label{sec-1-2}

\begin{itemize}
\item Rules of precedence
\item Integer division vs. real division
\item Modulus operator
\end{itemize}
\subsection*{Relational Comparisons, and the bool type}
\label{sec-1-3}

\begin{itemize}
\item Making decisions, we need be able to perform alternative actions
  based on making some sort of decision.
\item bool should be used instead of an int for boolean flags in a program in modern c.
\end{itemize}
\section*{Part 2 (11:45 - 12:30)}
\label{sec-2}
\subsection*{Algorithm}
\label{sec-2-1}

An algorithm is a series of unambiguous explicit steps that can be taken to solve a problem.
An algorithm is a procedure for solving a problem in terms of

\begin{enumerate}
\item the actions to execute and
\item the order in which the actions execute
\end{enumerate}

\end{document}
