% Created 2015-01-23 Fri 20:17
\documentclass[11pt]{article}
\usepackage[utf8]{inputenc}
\usepackage[T1]{fontenc}
\usepackage{fixltx2e}
\usepackage{graphicx}
\usepackage{longtable}
\usepackage{float}
\usepackage{wrapfig}
\usepackage{soul}
\usepackage{textcomp}
\usepackage{marvosym}
\usepackage{wasysym}
\usepackage{latexsym}
\usepackage{amssymb}
\usepackage{hyperref}
\tolerance=1000
\providecommand{\alert}[1]{\textbf{#1}}

\title{Lecture 02 Notes}
\author{Derek Harter}
\date{2015-01-20}
\hypersetup{
  pdfkeywords={},
  pdfsubject={Lecture 02 Notes.},
  pdfcreator={Emacs Org-mode version 7.9.3f}}

\begin{document}

\maketitle


\section*{Part 1 (11 - 11:40)}
\label{sec-1}
\subsection*{Memory Concepts, Variables, Representation}
\label{sec-1-1}


\begin{itemize}
\item When you declare a variable, it is assigned a location in memory.
  All references to store or access the named variable are translated
  by the C compiler using a lookup table.
\item The datatype is how pattern of bits in memory are stored and
  interpreted to mean a number, a character, etc.
\item Int32 representation (\href{http://en.wikipedia.org/wiki/Integer_%28computer_science%29}{http://en.wikipedia.org/wiki/Integer\_\%28computer\_science\%29})
\item Debug $\rightarrow$ QuickWatch
\item Debug $\rightarrow$ Windows $\rightarrow$ Memory
\item Representation of integer 1 vs. -1
\end{itemize}
\subsection*{Arithmetic and Precedence}
\label{sec-1-2}

\begin{itemize}
\item Rules of precedence
\item Integer division vs. real division
\item Modulus operator
\end{itemize}
\subsection*{Relational Comparisons, and the bool type}
\label{sec-1-3}

\begin{itemize}
\item Making decisions, we need be able to perform alternative actions
  based on making some sort of decision.
\item bool should be used instead of an int for boolean flags in a program in modern c.
\end{itemize}
\section*{Part 2 (11:45 - 12:30)}
\label{sec-2}
\subsection*{Algorithm}
\label{sec-2-1}

An algorithm is a series of unambiguous explicit steps that can be taken to solve a problem.
An algorithm is a procedure for solving a problem in terms of:

\begin{enumerate}
\item the actions to execute and
\item the order in which the actions execute
\end{enumerate}
\subsection*{Control Structures}
\label{sec-2-2}

Normally statements in a program execute one after the other in the
order in which they are written.  This is sequential execution.

\begin{itemize}
\item The sequence structure is built into the C language normal operation.
  Question: what/where is the first instruction executed in a C program?
\item Selection statements: single if, double if / else, multiple switch
\item Repetition statements: for, while (0 or more) and do..while (1 or more)
\end{itemize}
\subsection*{Multiple Selection with if..else}
\label{sec-2-3}

\begin{itemize}
\item can make 1 of multiple possible decisions using nested if statements.
\end{itemize}
\subsection*{Basic loop}
\label{sec-2-4}

\begin{itemize}
\item while loop repeats a set of statements 0 or more times, checking for
  a condition before beginning and before each repetition of the
  statements in the body of the loop.
\item An infinite loop occurs when you don't properly change values so
  that the condition of the loop is eventually false.
\end{itemize}
\subsection*{Different types of repetition}
\label{sec-2-5}

\begin{itemize}
\item Our first while loop is a kind of purely logical controlled repetition.  We keep repeating as long as the condition
\end{itemize}
is true, and once it becomes false we stop.
\begin{itemize}
\item A common variation is, Counter-Controlled repetition, when you know you need to process X specific items (we will see a better way of doing
\end{itemize}
this with a for loop next.)
\begin{itemize}
\item Discuss an off-by-one error in a counter-controlled loop.
\item Sentenial-controlled repetition means we keep repeating until we see
  some condition.  Our first example is really a sentenial-controlled
  loop, though a more typical example is to have some explicit
  sentenial/flag.
\end{itemize}
\subsection*{Increment, decrement and assignment}
\label{sec-2-6}

\begin{itemize}
\item C provides several assignment operators for abbreviating
  assignment since it is very common to need to manipulate a value and
  assign it back to the variable. (+=, -=, *=, etc.)
\item increment and decrement are uniary operators.  We can increment and
  decrement using pre or post increment, which means either we first
  incrmeent the value then use that in the expression, or we first use
  current value in expression then increment after.
\end{itemize}
\section*{Part 3 (12:40 - 1:40)}
\label{sec-3}
\subsection*{for Loop for Counter-Controlled Repetition}
\label{sec-3-1}

\begin{itemize}
\item Counter-controlled loops are so commonly needed (especially when
  processing arrays of elements later), that C provides a special
  construct for implementing them.
\item Parts of a for loop, initialization, test, manipulation.
\item Give examples of counting up and counting down.
\item Give example of counting with a step size.
\end{itemize}
\subsection*{do..while Repetition}
\label{sec-3-2}

\begin{itemize}
\item If we need to ensure some statements are always executed at least
  once, use do..while construct.  Useful so we can avoid duplicating
  some code.
\end{itemize}
\subsection*{switch statement}
\label{sec-3-3}

\begin{itemize}
\item You may run across the switch statement.  Switch provides multiple-selection (like nested if).
\item It is limited, can only compare the value of a variable to a
  constant integral expression (e.g. our example of grades in ranges
  can't be accomplished using switch, we still need to use nested if
  for them.)
\item switch statement requires use of break statements.
\item break and continue statements can be useful sometimes in loops, to
  avoid some unnecessary repetitions of the loop.
\item For example, in the prime number algorithm, we don't need to perform
  any more loops after we find first negative example.
\item Can use a break statement.  An equivalent effect can be achieved by using a flag.
\item continue can be used to skip any remaining statemens in current
  iteration, and then start immediately with the next iteration of the
  loop.
\end{itemize}
\subsection*{Visual Studio Tips}
\label{sec-3-4}

\begin{itemize}
\item Turn on expert settings
\item Format selection Ctrl-k Ctrl-f
\end{itemize}

\end{document}
