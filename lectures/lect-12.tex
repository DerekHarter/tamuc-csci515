% Created 2015-04-22 Wed 10:24
\documentclass[11pt]{article}
\usepackage[utf8]{inputenc}
\usepackage[T1]{fontenc}
\usepackage{fixltx2e}
\usepackage{graphicx}
\usepackage{longtable}
\usepackage{float}
\usepackage{wrapfig}
\usepackage{soul}
\usepackage{textcomp}
\usepackage{marvosym}
\usepackage{wasysym}
\usepackage{latexsym}
\usepackage{amssymb}
\usepackage{hyperref}
\tolerance=1000
\usepackage{minted}
\providecommand{\alert}[1]{\textbf{#1}}

\title{Lecture 12 Notes}
\author{Derek Harter}
\date{2015-04-21}
\hypersetup{
  pdfkeywords={},
  pdfsubject={Lecture 12 Notes.},
  pdfcreator={Emacs Org-mode version 7.9.3f}}

\begin{document}

\maketitle


\section{First Session (11 - 11:40)}
\label{sec-1}
\subsection{Relationship Between Pointers and Arrays}
\label{sec-1-1}

\begin{itemize}
\item Arrays and pointers are intimately related in C++ and may be used (almost) interchangeably.
\item An array name can be thought of as a constant pointer.
\item Pointers can be used to do any operation involving array sub
  scripting (using pointer arithmetic, see next).
\end{itemize}


\begin{verbatim}
int b[5]; // create 5-element array of integers
int* bPtr;
\end{verbatim}

\begin{itemize}
\item Because the array name (without a subscript) is a (constant) pointer
  to the first element of the array, we can set bPtr to the address of the
  first element in array b
\end{itemize}


\begin{verbatim}
bPtr = b;
\end{verbatim}

\begin{itemize}
\item equivalent to assignment the address of first element
\end{itemize}


\begin{verbatim}
bPtr = &b[0];
\end{verbatim}
\subsection{Pointer Arithmetic}
\label{sec-1-2}

\begin{itemize}
\item Pointers are valid operands in arithmetic expressions. (e.g. +/-)
\item Not all operators used in these expressions are valid with pointer variables (/? \%?)
\item Can increment or decrement a pointer.  Basically like iterating through index of an array.
\item Can add or subtract to a pointer, and can subtract one pointer from another.
\item Pointer arithmetic is usually only sensible in context of processing arrays in memory.
\item For array \verb~b~ and \verb~bPtr~, equivalent ways to access the third element
\end{itemize}


\begin{verbatim}
b[3];
*(bPtr + 3);
\end{verbatim}

\begin{itemize}
\item The 3 is the offset to the pointer.  Both expressions calculate the
  address that is 3 (integers) past where \verb~b~ / \verb~bPtr~ are pointing,
  and dereference that value.
\item The address of the 3rd element, in eqivalent notation
\end{itemize}


\begin{verbatim}
&b[3];
bPtr + 3;
\end{verbatim}
\section{Second Session (11:45 - 12:30)}
\label{sec-2}
\subsection{Character Arrays and Pointer-Based String Processing}
\label{sec-2-1}

\begin{itemize}
\item Old style (legacy) string processing
\item Should probably prefer to use string type whenever possible for new problems
\item However, helps to understand arrays and pointers, and helps if you have old code using old c-style strings
\item A string literal ``blue'' can be assigned to a string type, and it can
  be assigned to initialize an array of char for old-style C string
  processing.
\item If you don't specify array size, the size allocated is inferred from the string literal.
\end{itemize}


\begin{verbatim}
char color[] = "blue";
const char* colorPtr = "blue";
\end{verbatim}

\begin{itemize}
\item each initialize a variable to a string ``blue''.
\item Since arrays of characters and a pointer to the beginning of such an
  array can be used interchangeably, you can do either to create an
  array of characters.
\item old style C-strings use an implicit flag character, the null character \~{}'\0'\~{} as string terminator.
\item For use when passing strings to old-style string processing functions, like \verb~strlen~ and \verb~strcpy~.
\item NOTE: compare this method to method you have been taught for passing arrays to functions.
\end{itemize}
\section{Third Session (12:40 - 1:40)}
\label{sec-3}
\subsection{Arrays of Pointers}
\label{sec-3-1}

\begin{itemize}
\item Arrays may contain pointers (e.g. can have an array of pointers).
\item Useful for dynamically sized structures, or efficiency reasons.
\item A common usage is an array of char*, e.g. an array of old c-style strings
\end{itemize}


\begin{verbatim}
char* colors[] = {"red", "orange", "yellow", "green", "blue", "indigo", "violet" };
\end{verbatim}
\subsection{Command Line Arguments}
\label{sec-3-2}

\begin{itemize}
\item \verb~main()~ function is a \textbf{function}.  Takes parameters as input and returns result.
\item command line arguments given by user at a terminal when execute program.
\item Passed in through the \verb~argc~ and \verb~argv~ parameters to \verb~main~.
\item \verb~argv~ is an array of \verb~char*~.  Each pointer points to a string of a parsed command line argument.
\end{itemize}


\begin{verbatim}
int main(int argc, char* argv[])
\end{verbatim}
\subsection{Function Pointers}
\label{sec-3-3}

\begin{itemize}
\item A pointer to a function contains a function's address in memory.
\item Can pass pointers to functions as parameters to other functions.
\item Can return function pointers as results from other functions.
\item A useful technique, used extensively in functional programming.
\item For example, can specify a order function for sorting routines, so
  can change sorting order (ascending, descending, sort by different keys, etc.)
  without changing logic of the sort function.
\item Used for example in the C standard libraries.
\item stdlib.h qsort function and bsearch function use a compare function, return -1, 0 or 1
\end{itemize}

\end{document}
