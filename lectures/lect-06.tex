% Created 2015-02-23 Mon 18:56
\documentclass[11pt]{article}
\usepackage[utf8]{inputenc}
\usepackage[T1]{fontenc}
\usepackage{fixltx2e}
\usepackage{graphicx}
\usepackage{longtable}
\usepackage{float}
\usepackage{wrapfig}
\usepackage{rotating}
\usepackage[normalem]{ulem}
\usepackage{amsmath}
\usepackage{textcomp}
\usepackage{marvosym}
\usepackage{wasysym}
\usepackage{amssymb}
\usepackage{hyperref}
\tolerance=1000
\usepackage{minted}
\author{Derek Harter}
\date{CSci 515 Spring 2015 \textit{<2015-02-18 Wed>}}
\title{Lecture 06 Notes}
\hypersetup{
  pdfkeywords={},
  pdfsubject={Lecture 06 Notes.},
  pdfcreator={Emacs 24.3.1 (Org mode 8.2.4)}}
\begin{document}

\maketitle

\section{First Session (11 - 11:40)}
\label{sec-1}
\subsection{Introduction to Arrays}
\label{sec-1-1}
An array is a consecutive group of memory locations that all have the
same type.  When you declare an array, C will allocate a block of memory
large enough to hold all of the values of the type you ask for.  You access
the elements of the array by using a subscript or index.

\begin{itemize}
\item Declaring an array
\end{itemize}

\begin{minted}[linenos=true,frame=lines,numbersep=5pt,fontsize=\footnotesize]{c}
type arrayName[arraySize];
\end{minted}

\begin{itemize}
\item Indexing an array
\item Arrays in C are indexed \textbf{STARTING AT 0} (0 based indexing)
\item The size of the array like \verb~int c[10]~ is 10 items, indexed from 0 to 9.
\item When accessing all of the elements of an array, always use an indexed
controlled for loop, and always index the loop from 0 up to N, like this:
\end{itemize}

\begin{minted}[linenos=true,frame=lines,numbersep=5pt,fontsize=\footnotesize]{c}
int c[10];

for (int i = 0; i < 10; i++)
{
}
\end{minted}

\begin{itemize}
\item It is good practice to declare a constant, and use it wherever you are
referencing your array.
\end{itemize}

\begin{minted}[linenos=true,frame=lines,numbersep=5pt,fontsize=\footnotesize]{c}
const int ARRAY_SIZE = 10;
int c[ARRAY_SIZE];

for (int i = 0; i < ARRAY_SIZE; i++)
{
}
\end{minted}

\begin{itemize}
\item This way, if you need to change the size of the array and code
processing the array elements, you change 1 single location, and all
of the code will work correctly.
\item And when I say good practice, read that to mean you SHOULD ALWAYS
be doing this for assignments for this class.
\end{itemize}
\section{Second Session (11:45 - 12:30)}
\label{sec-2}
\subsection{Examples of Using Arrays}
\label{sec-2-1}
\begin{itemize}
\item Initializing with a loop
\item Initialize to random values.
\item Display all the values in an array.
\item Finding the minimum of the values of an array
\item Sum the values of an array
\end{itemize}
\section{Third Session (12:40 - 1:40)}
\label{sec-3}
\subsection{Passing Arrays to Functions}
\label{sec-3-1}
\begin{itemize}
\item Arrays are passed to function by reference (by default)
\item \textbf{ALWAYS} pass the array and the size to function to process an
array.  Makes function self-contained, does not depend on any
globals.
\end{itemize}
\subsection{Advanced Array Processing}
\label{sec-3-2}
\begin{itemize}
\item Revisit our dice game example from a previous lab.
\item Randomly initialize to range [1, 6]
\item Example, pass (2) arrays as input and an array to hold the result.
\item Count frequency of values.
\item In statistics, combinations of uniformly occurring randomness cause 
normally distributed probabilities.  This is an example (central limit
theorem).
\item A histogram is simply a bar chart of frequencies of occurring
outcomes in some experiment.  In our experiment, the possible
outcomes are sums from 2 to 12.
\end{itemize}
% Emacs 24.3.1 (Org mode 8.2.4)
\end{document}
