% Created 2015-04-07 Tue 14:46
\documentclass[11pt]{article}
\usepackage[utf8]{inputenc}
\usepackage[T1]{fontenc}
\usepackage{fixltx2e}
\usepackage{graphicx}
\usepackage{longtable}
\usepackage{float}
\usepackage{wrapfig}
\usepackage{soul}
\usepackage{textcomp}
\usepackage{marvosym}
\usepackage{wasysym}
\usepackage{latexsym}
\usepackage{amssymb}
\usepackage{hyperref}
\tolerance=1000
\usepackage{minted}
\providecommand{\alert}[1]{\textbf{#1}}

\title{Lecture 10 Notes}
\author{Derek Harter}
\date{2015-04-07}
\hypersetup{
  pdfkeywords={},
  pdfsubject={Lecture 10 Notes.},
  pdfcreator={Emacs Org-mode version 7.9.3f}}

\begin{document}

\maketitle


\section{First Session (11 - 11:40)}
\label{sec-1}
\subsection{Structures in C}
\label{sec-1-1}

\begin{itemize}
\item \verb~struct~ are \textbf{aggregrate} data types - they can be built using
  elements of primitive types.
\item \verb~struct~ are basically a \textbf{record}, a row of a database or table.
\item \verb~struct~ are an example of a \textbf{user defined type}.  Like \verb~enum~ they
  allow.
\item In this course we don't cover object oriented analysis and design,
  but basically classes in C++ are \verb~struct~ user defined types with
  associated methods that operate on the new type.
\end{itemize}
\subsection{Defining a struct}
\label{sec-1-2}



\begin{verbatim}
struct Trial
{
   string name;
   string gender;
   float reactionTime; // ms
   int numberOfPresses;
}; // don't forget the semicolon
\end{verbatim}

\begin{itemize}
\item By convention, always use an initial upper case letter for user
  defined types.  In the previous example, \verb~Trial~ is a user defined
  type (an experimental trial record for a single participant).
\item Usually a \verb~struct~ definition should be done globally, it usually
  doesn't make sense to have a new data type that is only defined
  inside of a single local function scope.
\end{itemize}
\subsection{Allocating and accessing a struct}
\label{sec-1-3}

\begin{itemize}
\item Declare a variable (or array) of the new type, just as you would any
  built-in type.
\item Use \verb~.~ with name of field to access fields for reading or writing to fields in
  an allocated structure.
\item Can use a comma separated list initializer, as we with arrays.
\end{itemize}
\subsection{Good Programming Practice enum}
\label{sec-1-4}

\begin{itemize}
\item We haven't used it a lot, but we actually have seen an example of
  user defined type before, using enum (coin, Heads or Tails).
\item Gender is a qualitative variable (it takes on named values, not
  numbers).  Qualitative variables always only take on a discrete set
  of allowable values.  You should always define an enum type when
  using a discrete qualitative variable, for readability and to reduce
  errors.
\end{itemize}
\section{Second Session (11:45 - 12:30)}
\label{sec-2}
\subsection{Using structures with functions}
\label{sec-2-1}

\begin{itemize}
\item Structure elements can be accessed individually if needed, and used or passed
  to function separately.
\item However, a \verb~struct~ is a new data type, so we can create functions that take
  the \verb~struct~ as a parameter.
\item We can even have functions that return \verb~struct~ as result, thus this is one
  way to create functions that return more than 1 value (e.g. to return a
  whole record of values).
\item \textbf{NOTE}: \verb~struct~ can potentially be very large, so can be inefficient to
  pass stucture by value (which is copied).  So sometimes might want to
  pass by reference instead when large.
\end{itemize}
\section{Third Session (12:40 - 1:40)}
\label{sec-3}
\subsection{Arrays of structures}
\label{sec-3-1}

\begin{itemize}
\item It is useful to create arrays of structures, this represents a basic
  table of records that we can process.
\item You can create an array of any user defined type, just as you would an
  array of integers, or of floats, etc.
\item Likewise, you can pass arrays of structures to functions as we have done
  with other types.  The arrays will be passed by reference, just as
  arrays of built-in types are passed by reference.
\end{itemize}

\end{document}
