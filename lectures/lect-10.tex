% Created 2015-04-07 Tue 12:50
\documentclass[11pt]{article}
\usepackage[utf8]{inputenc}
\usepackage[T1]{fontenc}
\usepackage{fixltx2e}
\usepackage{graphicx}
\usepackage{longtable}
\usepackage{float}
\usepackage{wrapfig}
\usepackage{soul}
\usepackage{textcomp}
\usepackage{marvosym}
\usepackage{wasysym}
\usepackage{latexsym}
\usepackage{amssymb}
\usepackage{hyperref}
\tolerance=1000
\usepackage{minted}
\providecommand{\alert}[1]{\textbf{#1}}

\title{Lecture 10 Notes}
\author{Derek Harter}
\date{2015-04-07}
\hypersetup{
  pdfkeywords={},
  pdfsubject={Lecture 10 Notes.},
  pdfcreator={Emacs Org-mode version 7.9.3f}}

\begin{document}

\maketitle


\section{First Session (11 - 11:40)}
\label{sec-1}
\subsection{Structures in C}
\label{sec-1-1}


\begin{itemize}
\item \verb~struct~ are \textbf{aggregrate} data types - they can be built using
  elements of primitive types.
\item \verb~struct~ are basically a \textbf{record}, a row of a database or table.
\item \verb~struct~ are an example of a \textbf{user defined type}.  Like \verb~enum~ they
  allow.
\item In this course we don't cover object oriented analysis and design,
  but basically classes in C++ are \verb~struct~ user defined types with
  associated methods that operate on the new type.
\end{itemize}
\subsection{Defining a \verb~struct~}
\label{sec-1-2}
\section{Second Session (11:45 - 12:30)}
\label{sec-2}
\subsection{Another header}
\label{sec-2-1}
\section{Third Session (12:40 - 1:40)}
\label{sec-3}

\end{document}
