% Created 2015-04-14 Tue 14:55
\documentclass[11pt]{article}
\usepackage[utf8]{inputenc}
\usepackage[T1]{fontenc}
\usepackage{fixltx2e}
\usepackage{graphicx}
\usepackage{longtable}
\usepackage{float}
\usepackage{wrapfig}
\usepackage{soul}
\usepackage{textcomp}
\usepackage{marvosym}
\usepackage{wasysym}
\usepackage{latexsym}
\usepackage{amssymb}
\usepackage{hyperref}
\tolerance=1000
\usepackage{minted}
\providecommand{\alert}[1]{\textbf{#1}}

\title{Lecture 11 Notes}
\author{Derek Harter}
\date{2015-04-07}
\hypersetup{
  pdfkeywords={},
  pdfsubject={Lecture 10 Notes.},
  pdfcreator={Emacs Org-mode version 7.9.3f}}

\begin{document}

\maketitle


\section{First Session (11 - 11:40)}
\label{sec-1}
\subsection{Refresher on memory concepts}
\label{sec-1-1}

\begin{itemize}
\item Memory consists of a sequence of numbered addresses (each holding 1 byte of information)
\item Addresses can be accessed randomly (Random Access Memory).  It
  takes the same amount of time to access any particular address in
  primary memory (O(1), or constant time).
\item When you declare a variable in C, you allocate memory at some
  particular address in memory to hold the value you assign to the
  variable.
\item The type of the variable determines the size, or number of bytes
  of memory, needed to hold the variable.
\item The type also determines the interpretation of the bit pattern in
  memory of those bytes.
\end{itemize}
\subsection{Pointers Introduction}
\label{sec-1-2}

\begin{itemize}
\item A declared variable actually is associated with a memory address.  In C compiler, 
  a look up table translates every variable name reference to the associated address when
  creating the machine code.
\item Pointer variables contain \textbf{memory addresses} as their values.
\item \textbf{Pointer variables contain MEMORY ADDRESSES as their values.}
\item We use pointers indirectly. \textbf{indirect referencing}
\item We dereference a pointer, in order to access the value it is pointing too.
\item Confusing for some people, a pointer variable holds a memory address, but we still
  associate a type with a pointer variable.
\begin{itemize}
\item Because, when we dereference a pointer variable, we need to know
    how to interpret the bit pattern in memory that it is pointing
    too.
\end{itemize}
\end{itemize}
\subsection{Pointer Variables}
\label{sec-1-3}

\begin{itemize}
\item Pointers, like any other variable, must be declared before they can be used.
\end{itemize}

\begin{verbatim}
int count;
int* countPtr;

double x;
double* xPtr;
\end{verbatim}
\begin{itemize}
\item Do not declare multiple pointer variables using single line declaration. Always
  declare pointers on their own line.
\item It is good practice to use Ptr on name of variable, to indicate they
  are pointer variables.
\item Please follow our class convention, and put the * next to type, to indicate it is
  a pointer to the indicated type.
\item Pointers must be initialized before they can be used (like any other variable).
\begin{itemize}
\item It is a common, and \textbf{very bad not to ever be done} error to try and dereference
    an uninitialized pointer.  Bad things will happen, don't do it.
\end{itemize}
\item Pointers can be initialized to 0 or NULL, but \textbf{never dereference a null pointer}.
\end{itemize}
\section{Second Session (11:45 - 12:30)}
\label{sec-2}
\subsection{Pointer Operators: Address of Operator}
\label{sec-2-1}

\begin{itemize}
\item How do we initialize a pointer?
\item Well obviously, pointers hold memory addresses, so we have to initialize with a memory address.
\item How do we found out memory address of interest?
\item use \verb~&~ operator.  Read as the \textbf{address of} operator.
\end{itemize}

\begin{verbatim}
int y = 5;
int* yPtr;

yPtr = &y; // assign address of y to yPtr
\end{verbatim}
\begin{itemize}
\item As we just did to initialize y to an integer value when declaraing, we can also initialize
  a pointer variable in declaration, thus:
\end{itemize}

\begin{verbatim}
int y = 5;
int* yPtr = &y;
\end{verbatim}
\subsection{Pointer Operators: dereference operator}
\label{sec-2-2}

\begin{itemize}
\item Ok we can declare a pointer, and point it at something.  What use is that?
\item When a pointer is validly referencing memory, we can derefrence it
  and use the dereferenced location anywhere we could use a simple variable
  of the given type.
\item Derefernce to read value out and display it.  Operator \verb~*~ is the \textbf{dereference} operator.
\end{itemize}

\begin{verbatim}
cout << *yPtr << endl;
\end{verbatim}
\begin{itemize}
\item Can dereference to read value out, and use it in any arbitray expression
\end{itemize}

\begin{verbatim}
int x = 3;
x = *yPtr + 5;
\end{verbatim}
\begin{itemize}
\item Dereference to write value to reference memory location
\end{itemize}

\begin{verbatim}
*yPtr = 9; // assign value 9 to location referenced in yPtr (e.g. to variable y)
cout << *yPtr << endl;
cout << y << endl;
\end{verbatim}
\begin{itemize}
\item Can also assign value using IOstream input
\end{itemize}

\begin{verbatim}
cin >> *yPtr;
\end{verbatim}
\subsection{Memory Addresses, again}
\label{sec-2-3}

\begin{itemize}
\item Lets look at the actual values in one of our pointer variables
\end{itemize}

\begin{verbatim}
int a;     // a is an integer
int* aPtr; // aPtr will point to the a integer

a = 7;
*aPtr = &a;

cout << "The address of a is " << aPtr << endl;
\end{verbatim}
\begin{itemize}
\item Also, remember that \verb~&~ and \verb~*~ (address of and dereference operators) are inverses 
  of one another.
\end{itemize}
\subsection{sizeof operator}
\label{sec-2-4}

\begin{itemize}
\item We have seen examples of this.  Can be used to determine number of bytes
  a variable is using.
\item Can be used for arrays.
\item In C and typed languages, the type of a variable is tied to its size, each particular
  type needs some number of bytes of memory to represent values of that type.
\item The sizeof a pointer variable will depend on the size used to represent and hold memory
  addresses on the machine you are running.  But typically it is 4 bytes (32 bits) on
  32-bit architectures or 8 bytes (64 bits) on 64-bit architecture.
\end{itemize}
\section{Third Session (12:40 - 1:40)}
\label{sec-3}
\subsection{Pass by Reference}
\label{sec-3-1}

\begin{itemize}
\item Pass by reference used not to be built in to C language.
\item But you can pass by reference by passing pointer parameters.
\item They are equivalent, function parameters declared as pass by reference
  are set to pass in a pointer (a memory reference).  Only difference is
  that a reference variable in a function is dereferenced for you behind the
  scenes.
\end{itemize}
\subsection{Multiple Indirection}
\label{sec-3-2}

\begin{itemize}
\item Pointer dereferences can be chained together, to create multiple levels of
  reference/dereference.
\item This can be useful in many cases, e.g. an array of pointers to structures, that have pointers
  to link to other structures.  We will see more of this when we look at data structures using
  pointers to create links between items.
\item As an example, lets use 2 levels of reference.
\end{itemize}

\begin{verbatim}
int x;
int* xPtr = &x;
int** xPtrPtr = &xPtr;

**xPtrPtr = 42;
cout << x;
\end{verbatim}
\subsection{Pointers to Structs and Struct Members}
\label{sec-3-3}

\begin{itemize}
\item Structs have a special operator used to reference members of a struct pointed to
  by a pointer reference.
\item This was so commonly done, the operator \verb~->~ was developed.  Read as \textbf{pointer to member}.
\end{itemize}

\begin{verbatim}
struct Trial
{
   string name;
   string gender;
   float reactionTime; // ms
   int numberOfPresses;
}; // don't forget the semicolon

Trial t;
Trial* tPtr;

tPtr = &t;
t->name = "Jane Student";
t->gender = "Female";
t->reactionTime = 42;
\end{verbatim}

\end{document}
