% Created 2015-01-30 Fri 13:09
\documentclass[11pt]{article}
\usepackage[utf8]{inputenc}
\usepackage[T1]{fontenc}
\usepackage{fixltx2e}
\usepackage{graphicx}
\usepackage{longtable}
\usepackage{float}
\usepackage{wrapfig}
\usepackage{soul}
\usepackage{textcomp}
\usepackage{marvosym}
\usepackage{wasysym}
\usepackage{latexsym}
\usepackage{amssymb}
\usepackage{hyperref}
\tolerance=1000
\usepackage{minted}
\usepackage{minted}
\usemintedstyle{default}
\providecommand{\alert}[1]{\textbf{#1}}

\title{Lab 02: Calculating PI}
\author{CSci 515 Spring 2015}
\date{2015-01-23}
\hypersetup{
  pdfkeywords={},
  pdfsubject={Lab 02},
  pdfcreator={Emacs Org-mode version 7.9.3f}}

\begin{document}

\maketitle


\section*{Dates:}
\label{sec-1}


\begin{center}
\begin{tabular}{ll}
 Due:  &  In Lab, Wednesday January 28, by 4 pm (lab end time)  \\
\end{tabular}
\end{center}
\section*{Objectives}
\label{sec-2}

\begin{itemize}
\item Practice writing index controlled loops
\item Become more comfortable with using digital computers for calculating mathematical expressions in C.
\item More practice with output formatting.
\item Practice using real valued variables for mathematical calculations.
\item Gain experience in translating formula into algorithmic procedures.
\item Practice using uniary, binary and special assignment operators.
\end{itemize}
\section*{Description}
\label{sec-3}

Calculate an approximate value of $\pi$ from the series:

$$ \pi = 4 - \frac{4}{3} + \frac{4}{5} - \frac{4}{7} + \frac{4}{9} - \frac{4}{11} ... $$

Print a table that shows the approximate value of $\pi$ after each of
the first N terms of this series.  Your program should prompt the user
for the value of N to determine how many values of the table of
approximate values of $\pi$ will be displayed.  A session of using
your program from the terminal should have exactly the following
output:


\begin{verbatim}
What size of table should I compute: 5

N     pi
----- -------
1     4
2     2.66667
3     3.46667
4     2.89524
5     3.33968
\end{verbatim}
\section*{Extra Credit}
\label{sec-4}

If you calculate the approximation of pi correctly using the series,
you will see that using standard cout output it appears that only 6
digits of precision are being calculated (e.g. create a table of size
1000 or bigger).  However, the standard float data type is capable of
representing 16 digits of precision.  The problem is that by default
sending floats to cout only display a few decimal digits.  Find out
how to display all of the calculated digits, using the iostream
library.
\section*{Lab Submission}
\label{sec-5}


An eCollege dropbox has been created for this lab.  You should
upload your version of the lab by the end of lab time to the eCollege
dropbox named \verb~Lab 02 Calculating Pi~.  Work submitted by the end
of lab will be considered, but after the lab ends you may no longer
submit work, so make sure you submit your best effort by the lab end
time in order to receive credit.
\section*{Requirements and Grading Ruberics}
\label{sec-6}
\subsection*{Program Execution, Output and Functional Requirements}
\label{sec-6-1}


\begin{itemize}
\item Your program must compile, run and produce some sort of output to be
  graded.  0 if not satisfied.
\item Your program must correctly respond to the user input when displaying
  the number of items in the series.  20 or more points.
\item Your program must correctly calculate the values in the series. 40 or 
  more points.
\item Your program must correctly format the output as shown in the lab
  description.  20 points
\end{itemize}
\subsection*{Program Style}
\label{sec-6-2}


Your programs must conform to the style and formatting guidelines given for this course.
The following is a list of the guidelines that are required for the lab to be submitted
this week.

\begin{itemize}
\item The file header and function header for your main function must be present, and filled out correctly.
\item You must indent your code correctly and have no embedded tabs in your source code. (Don't forget about the Visual Studio Format Selection command).
\item You must not have any statements that are hacks in order to keep your terminal from closing when your program exits.
\item You must have a single space before and after each binary operator.
\item You must have a single blank line after the end of your declaration
  of variables at the top of a function, before the first code
  statement.
\end{itemize}

Failure to conform to any of these formatting and programming practice
guidelines for this lab will result in a grade of 0 for the lab, and
your program being returned with an indication of which of these items
your program violates.  Failure to follow other class/textbook
programming guidelines may result in a loss of points, especially for
those good programming practices given in chapters 1-5 of our textbook
which you should have read by now.

\end{document}
