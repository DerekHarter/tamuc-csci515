% Created 2015-04-21 Tue 20:29
\documentclass[11pt]{article}
\usepackage[utf8]{inputenc}
\usepackage[T1]{fontenc}
\usepackage{fixltx2e}
\usepackage{graphicx}
\usepackage{longtable}
\usepackage{float}
\usepackage{wrapfig}
\usepackage{rotating}
\usepackage[normalem]{ulem}
\usepackage{amsmath}
\usepackage{textcomp}
\usepackage{marvosym}
\usepackage{wasysym}
\usepackage{amssymb}
\usepackage{hyperref}
\tolerance=1000
\usepackage{minted}
\usepackage{minted}
\usemintedstyle{default}
\author{CSci 515 Spring 2015}
\date{\textit{<2015-04-21 Tue>}}
\title{Lab 12:}
\hypersetup{
  pdfkeywords={},
  pdfsubject={Lab 12},
  pdfcreator={Emacs 24.3.1 (Org mode 8.2.4)}}
\begin{document}

\maketitle

\section*{Dates:}
\label{sec-1}
\begin{center}
\begin{tabular}{ll}
Due: & In Lab, Wednesday April 22, by 4:05 pm (lab end time)\\
\end{tabular}
\end{center}
\section*{Objectives}
\label{sec-2}
\begin{itemize}
\item Understand the relationship between arrays and pointers in C
\item Use pointers to an array of items, and process them.
\item Understand better C-string processing by implementing sentinel based processing functions.
\end{itemize}
\section*{Description}
\label{sec-3}
In this lab we will implement a similar concept to passing arrays of
characters for string processing by defining an array of (positive)
integer type.  Arrays of positive integers can be of varying length.
Since arrays of positive integers must always contain values $v[i] >=
0$ we can use the value of \verb~-1~ as a sentinel character.  Thus we will
not pass array sizes to function that process arrays of positive
integers, instead all such function will expect that a \verb~-1~ terminal
sentinel will be set as the final value for such arrays.  In this lab
you will implement functions to accept a pointer to an array of
positive integers, that implement the calculating the length of such
arrays and comparing if two arrays are equal.

Perform the following tasks:

\begin{enumerate}
\item Create a function called \verb~posintprint()~.  This function should take
a pointer to an array of integers as its first and only argument.
This function is a void function, it doesn't display any result.  
However, this function should cause the values of the list of positive
integers provided to be displayed to standard output as its result.
See the example output for how this output should be displayed.
(As extra credit, try not to have the last unnecessary ',' printed out).

\item Create a function called \verb~posintlen()~.  This function should take
a pointer to an array of integers as its first and only argument.
This function will expect that the array is terminated with the
\verb~-1~ sentinel character correctly.  The function should return
an integer value, which is the number of positive integer values
that is contained in the array.

\item Create a function called \verb~posintcmp()~.  This function should take
a pointer to an array of (positive, \verb~-1~ terminated) integers as
its first and second parameters.  This function should return an
integer.  As with the \verb~strcmp()~ function, this function should
compare all corresponding elements of the two arrays of positive
integers provided as parameters.  It will return a 0 if all of the
elements in both arrays are 0 (you can assume for the lab for today
that the two items will be of the same length), and it will return
-1 or +1 as appropriate, depending on if the first array is less
than or greater than the second at the first position that differs
between the two input parameters.

\item In your \verb~main()~ function make sure you demonstrate your code and
perform the following tests.  Test that your \verb~posintlen()~ function
works for lists of length 5 and of length 0.  Test that your 
\verb~posintcmp()~ function returns 0 for 2 equal lists, and +1 and
-1 for two unequal lists as appropriate.
\end{enumerate}

Example output is shown below (value between the \{ and \} are being displayed
by the \verb~posintprint()~ function for different lists of positive integers): 


\begin{verbatim}
List 1: { 5, 3, 8, 2, 7, }
Length of List 1: 5

List 2: { }
Length of List 2: 0

List 3: { 4, 2, 9, }
List 4: { 4, 2, 9, }
Result of posintcmp(list3, list4): 0

List 3: { 4, 2, 9, }
List 5: { 4, 2, 7, }
Result of posintcmp(list3, list5): 1
Result of posintcmp(list5, list3): -1
\end{verbatim}

\textbf{NOTE}: Now that our programs have more functions than just the
\verb~main()~ function, the use of the function headers becomes meaningful
and required.  Make sure that all of your functions have function
headers preceding them that document the purpose of the functions, and
the input parameters and return value of the function.
\section*{Lab Submission}
\label{sec-4}

An eCollege dropbox has been created for this lab.  You should upload
your version of the lab by the end of lab time to the eCollege dropbox
named \verb~Lab 12 Arrays of Positive Ints~.  Work submitted by the end of
lab will be considered, but after the lab ends you may no longer
submit work, so make sure you submit your best effort by the lab end
time in order to receive credit.
\section*{Requirements and Grading Rubrics}
\label{sec-5}

\subsection*{Program Execution, Output and Functional Requirements}
\label{sec-5-1}

\begin{enumerate}
\item Your program must compile, run and produce some sort of output to be
graded. 0 if not satisfied.
\item 35+ pts.  Your \verb~posintlen()~ function must work correctly and take
a pointer to an array of integers as specified.
\item 40+ pts.  Your \verb~posintcmp()~ function works correctly and takes
two pointers to arrays of positive integers as specified.
\item 25+ pts. You main function demonstrates correctly invoking the functions
and displays the required output from doing so.
\end{enumerate}

\subsection*{Program Style}
\label{sec-5-2}

Your programs must conform to the style and formatting guidelines given for this course.
The following is a list of the guidelines that are required for the lab to be submitted
this week.

\begin{enumerate}
\item The file header for the file with your name and program information
and the function header for your main function must be present, and
filled out correctly.
\item A function header must be present for all functions you define.
You must document the purpose, input parameters and return values
of all functions.  Your function headers must be formatted exactly
as shown in the style guidelines for the class.
\item You must indent your code correctly and have no embedded tabs in
your source code. (Don't forget about the Visual Studio Format
Selection command).
\item You must not have any statements that are hacks in order to keep
your terminal from closing when your program exits (e.g. no calls
to system() ).
\item You must have a single space before and after each binary operator.
\item You must have a single blank line after the end of your declaration
of variables at the top of a function, before the first code
statement.
\item You must have a single blank space after , and \verb~;~ operators used as a
separator in lists of variables, parameters or other control
structures.
\item You must have opening \verb~{~ and closing \verb~}~ for control statement blocks
on their own line, indented correctly for the level of the control
statement block.
\item All control statement blocks (if, for, while, etc.) must have \verb~{~
\verb~}~ enclosing them, even when they are not strictly necessary
(when there is only 1 statement in the block).
\item You should attempt to use meaningful variable and function names in
your program, for program clarity.  Of course, when required, you
must name functions, parameters and variables as specified in the
assignments.  Variable and function names must conform to correct
\verb~camelCaseNameingConvention~ .
\item Put the \verb~*~ for pointer variable declarations next to the
type declaration, with no space between the type and the \verb~*~.
Also please follow the convention of using \verb~Ptr~ at the end of
names for pointer variables.
\end{enumerate}

Failure to conform to any of these formatting and programming practice
guidelines for this lab will result in at least 1/3 of the points (33)
for the assignment being removed for each guideline that is not
followed (up to 3 before getting a 0 for the assignment). Failure to
follow other class/textbook programming guidelines may result in a
loss of points, especially for those programming practices given in
our Deitel textbook that have been in our required reading so far.
% Emacs 24.3.1 (Org mode 8.2.4)
\end{document}
