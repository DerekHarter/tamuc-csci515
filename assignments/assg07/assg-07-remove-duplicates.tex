% Created 2015-03-11 Wed 09:04
\documentclass[11pt]{article}
\usepackage[utf8]{inputenc}
\usepackage[T1]{fontenc}
\usepackage{fixltx2e}
\usepackage{graphicx}
\usepackage{longtable}
\usepackage{float}
\usepackage{wrapfig}
\usepackage{soul}
\usepackage{textcomp}
\usepackage{marvosym}
\usepackage{wasysym}
\usepackage{latexsym}
\usepackage{amssymb}
\usepackage{hyperref}
\tolerance=1000
\usepackage{minted}
\usepackage{minted}
\usemintedstyle{default}
\providecommand{\alert}[1]{\textbf{#1}}

\title{Assg 07: Remove Duplicates from Array\~{}}
\author{CSci 515 Spring 2015}
\date{2015-02-19}
\hypersetup{
  pdfkeywords={},
  pdfsubject={Assg 07},
  pdfcreator={Emacs Org-mode version 7.9.3f}}

\begin{document}

\maketitle


\section*{Dates:}
\label{sec-1}


\begin{center}
\begin{tabular}{ll}
 Due:  &  Tuesday March 24, by Midnight  \\
\end{tabular}
\end{center}
\section*{Objectives}
\label{sec-2}

\begin{itemize}
\item Practice declaring and using arrays.
\item More practice writing functions that take arrays as parameters and
  process them.
\item Look at shifting values in arrays.
\end{itemize}
\section*{Description}
\label{sec-3}

In this assignment, we want to solve the following task.  We need to design
and implement a function that will take an array of integers as input, and
will return a new array with only the unique values from the original
input array (e.g. a new array without any duplicate items).  For example,
if we had the following values in an array of size 5:


\begin{verbatim}
1
3
2
3
2
\end{verbatim}

The result would be a new array of size 3, with only the following unique
values in the array


\begin{verbatim}
1
3
2
\end{verbatim}

The general idea or algorithm that is easiest to implement to solve
this task is as follows.  Iterate through each value in the input
array.  For each value in the input, first see if it already exists in
the output array, and if it doesn't then copy it to the output array.
Of course, if it already exists in the output array, then you will do
nothing.

In order to implement this algorithm, it would be nice to have a
function that takes an array as input and searches the array to
determine if a value is in the array or not.  This function should
return true if the value is in the array, and false otherwise.  This
function should be able to handle an array of size 0, and the answer
when searching for a value in an array with nothing yet in it would be
false, e.g. the value is not yet present in the array.

Given this search function that can handle searching in an empty
array, the implementation of the copying from the input to the
output array, checking for duplicates, becomes relatively
straightforward.  You simply need to start with the output array
being empty (with a size of 0).  For every value in the input array,
you simply perform the search, and if it is not in the output array,
you put it on the end of the array, and increase the size of the array
by 1.  The result, after iterating through the whole input array, should
be to only copy the values the first time they are seen to the output
array.

Perform the following tasks:

\begin{enumerate}
\item Write a function called \verb~searchForValue~.  This function takes
   an array of ints and the size of the array as its parameters.  This
   function also takes a third parameter, an int, which is the value to
   search for.  This function returns a \verb~bool~ result of true if the
   value being searched for is in the given array, and false if the
   value is not in the array.  If you pass in an empty array (e.g. an
   array of size 0), the function should still work.  If the array is
   empty then of course the answer is false, since there are no values
   in the array and so there is no possibility that you can find the
   value that is being asked to search for.
\item Write a function called \verb~findUniqueValues~.  This functions
   takes an input array of int values, and the size of the array
   as its first two parameters.  You should also pass another
   array of int values as a third parameter.  This array should
   be allocated to be the same size as the input array, since
   potentially in the worst case, if there are no duplicates,
   this array will end up being simply a copy of the original
   array.  This function should copy over the values from
   the input array into the output array, but as described above
   it should only copy over unique values (duplicates will
   not be copied to the output array).  The resulting size of
   the array can be smaller than the input array.  Thus this
   function should return the size of the output array as
   its result e.g. the size of the array after removing all
   duplicates.
\item In your main function, declare an array of integers called \verb~values~
   of size 10.  Initialize all of the values in this array to random
   integers in the range from 0 to 4.  Declare another array called
   \verb~uniques~ of 10 integers.  Use your \verb~removeDuplicates~ functions to
   remove any duplicate values, and remember to get the new size of
   the \verb~uniques~ array from the function, after duplicates have been
   removed.
\item Display the \verb~values~ array and the \verb~uniques~ arrays to standard
   output.  An example of the output you should display is given below.
\end{enumerate}

Your program output should look exactly like the following when I run
your program. 


\begin{verbatim}
----- Values array (with duplicates):
[000]     3
[001]     4
[002]     4
[003]     4
[004]     1
[005]     0
[006]     0
[007]     2
[008]     1
[009]     3

----- Uniques array (duplicates removed):
[000]     3
[001]     4
[002]     1
[003]     0
[004]     2
\end{verbatim}


\textbf{NOTE}: Now that our programs have more functions than just the
\verb~main()~ function, the use of the function headers becomes meaningful
and required.  Make sure that all of your functions have function
headers preceding them that document the purpose of the functions, and
the input parameters and return value of the function.
\section*{Assignment Submission}
\label{sec-4}


An eCollege dropbox has been created for this assignment.  You should
upload your version of the assignment to the eCollege dropbox named
\verb~Assg 07 Remove Duplicates~ created for this submission.  Work
submitted by the due date will be considered for evaluation.
\section*{Requirements and Grading Rubrics}
\label{sec-5}
\subsection*{Program Execution, Output and Functional Requirements}
\label{sec-5-1}


\begin{enumerate}
\item Your program must compile, run and produce some sort of output to
   be graded. 0 if not satisfied.
\item 35+ pts.  Your implementation of the \verb~searchForValue~ function must
   be correct and must use the correct parameters and return the
   correct return type as specified above for the assignment.
\item 45+ pts. Your implementation of the \verb~removeDuplicates~ function
   must be correct.  The function should take the stated parameters as
   input.  The function must return an integer value, the size of the
   output array after duplicates were found and removed.  The function
   must perform its task correctly.
\item 20+ pts. You should create the \verb~values~ and \verb~uniques~ arrays in
   your \verb~main()~ function as specified.  Your \verb~values~ array should be
   initialized with random values.  Your output for your program
   should look exactly as shown in the example output.
\end{enumerate}
\subsection*{Program Style}
\label{sec-5-2}


Your programs must conform to the style and formatting guidelines
given for this course.  The following is a list of the guidelines that
are required for the assignment to be submitted this week.

\begin{enumerate}
\item The file header for the file with your name and program information
  and the function header for your main function must be present, and
  filled out correctly.
\item A function header must be present for all functions you define.
   You must document the purpose, input parameters and return values
   of all functions.  Your function headers must be formatted exactly
   as shown in the style guidelines for the class.
\item You must indent your code correctly and have no embedded tabs in
  your source code. (Don't forget about the Visual Studio Format
  Selection command).
\item You must not have any statements that are hacks in order to keep
   your terminal from closing when your program exits (e.g. no calls
   to system() ).
\item You must have a single space before and after each binary operator.
\item You must have a single blank line after the end of your declaration
  of variables at the top of a function, before the first code
  statement.
\item You must have a single blank space after , and \verb~;~ operators used as a
  separator in lists of variables, parameters or other control
  structures.
\item You must have opening \verb~{~ and closing \verb~}~ for control statement blocks
  on their own line, indented correctly for the level of the control
  statement block.
\item All control statement blocks (if, for, while, etc.) must have \verb~{~
   \verb~}~ enclosing them, even when they are not strictly necessary
   (when there is only 1 statement in the block).
\item You should attempt to use meaningful variable and function names in
   your program, for program clarity.  Of course, when required, you
   must name functions, parameters and variables as specified in the
   assignments.  Variable and function names must conform to correct
   \verb~camelCaseNameingConvention~ .
\end{enumerate}

Failure to conform to any of these formatting and programming practice
guidelines for this assignment will result in at least 1/3 of the
points (33) for the assignment being removed for each guideline that
is not followed (up to 3 before getting a 0 for the
assignment). Failure to follow other class/textbook programming
guidelines may result in a loss of points, especially for those
programming practices given in our Deitel textbook that have been in
our required reading so far.

\end{document}
