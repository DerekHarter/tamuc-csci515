% Created 2015-01-24 Sat 10:00
\documentclass[11pt]{article}
\usepackage[utf8]{inputenc}
\usepackage[T1]{fontenc}
\usepackage{fixltx2e}
\usepackage{graphicx}
\usepackage{longtable}
\usepackage{float}
\usepackage{wrapfig}
\usepackage{soul}
\usepackage{textcomp}
\usepackage{marvosym}
\usepackage{wasysym}
\usepackage{latexsym}
\usepackage{amssymb}
\usepackage{hyperref}
\tolerance=1000
\usepackage{minted}
\usepackage{minted}
\usemintedstyle{default}
\providecommand{\alert}[1]{\textbf{#1}}

\title{Lab 02: Find Average of a Group of Numbers}
\author{CSci 515 Spring 2015}
\date{2015-01-23}
\hypersetup{
  pdfkeywords={},
  pdfsubject={Lab 02},
  pdfcreator={Emacs Org-mode version 7.9.3f}}

\begin{document}

\maketitle


\section*{Dates:}
\label{sec-1}


\begin{center}
\begin{tabular}{ll}
 Due:  &  Tuesday February 3, by Midnight  \\
\end{tabular}
\end{center}
\section*{Objectives}
\label{sec-2}

\begin{itemize}
\item Practice writing sentinel-controlled loops.
\item Become familiar with basic if..else decisions, and simple boolean logic.
\item Practice with arithmetic and type conversion in C.
\end{itemize}
\section*{Description}
\label{sec-3}

Determine the average, and some other properties for an arbitrary
number of integer values.  The user should be prompted to enter
numbers from the terminal, and to indicate when they are done entering
numbers by using a sentinel (use \verb~-9999~ as your sentinel value).

Your program should report the following values after the user has
indicated they are finished entering numbers.  The sum of the values,
the average of the values, the minimum value entered, the maximum
value entered, and the total number of values the user entered.  The
output of a session with a user should be formatted to look exactly
like this:


\begin{verbatim}
Enter integer values, when done enter -9999.
Enter next value: 3
Enter next value: 6
Enter next value: 13
Enter next value: -3
Enter next value: 8
Enter next value: -2
Enter next value: -9999

Number of Values Entered: 6
Sum of values: 25
Maximum value: 13
Minimum value: -3
Average value: 4.16667
\end{verbatim}
\section*{Assignment Submission}
\label{sec-4}


An eCollege dropbox has been created for this assignment.  You should
upload your version of the out of class assignment by the end of
Tuesday 2/3 (midnight) to the dropbox named \verb~Assg 02 Average of Values~.
Late submissions will not be graded.
\section*{Requirements}
\label{sec-5}

Your programs must conform to the style and formatting guidelines given for this course.
The following is a list of the guidelines that are required for the lab to be submitted
this week.

\begin{itemize}
\item The file header and function header for your main function must be present, and filled out correctly.
\item You must indent your code correctly and have no embedded tabs in your source code. (Don't forget about the Visual Studio Format Selection command).
\item You must not have any statements that are hacks in order to keep your terminal from closing when your program exits.
\item You must have a single space before and after each binary operator.
\item You must have a single blank line after the end of your declaration
  of variables at the top of a function, before the first code
  statement.
\end{itemize}

Failure to conform to any of these formatting and programming practice
guidelines for this lab will result in a grade of 0 for the lab, and
your program being returned with an indication of which of these items
your program violates.  Failure to follow other class/textbook
programming guidelines may result in a loss of points, especially for
those good programming practices given in chapters 1-5 of our textbook
which you should have read by now.

\end{document}
