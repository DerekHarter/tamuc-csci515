% Created 2015-03-25 Wed 14:44
\documentclass[11pt]{article}
\usepackage[utf8]{inputenc}
\usepackage[T1]{fontenc}
\usepackage{fixltx2e}
\usepackage{graphicx}
\usepackage{longtable}
\usepackage{float}
\usepackage{wrapfig}
\usepackage{soul}
\usepackage{textcomp}
\usepackage{marvosym}
\usepackage{wasysym}
\usepackage{latexsym}
\usepackage{amssymb}
\usepackage{hyperref}
\tolerance=1000
\usepackage{minted}
\usepackage{minted}
\usemintedstyle{default}
\providecommand{\alert}[1]{\textbf{#1}}

\title{Assg 08: Selection Sort}
\author{CSci 515 Spring 2015}
\date{2015-02-19}
\hypersetup{
  pdfkeywords={},
  pdfsubject={Assg 08},
  pdfcreator={Emacs Org-mode version 7.9.3f}}

\begin{document}

\maketitle


\section*{Dates:}
\label{sec-1}


\begin{center}
\begin{tabular}{ll}
 Due:  &  Tuesday March 31, by Midnight  \\
\end{tabular}
\end{center}
\section*{Objectives}
\label{sec-2}

\begin{itemize}
\item Implement a sorting algorithm
\item Practice using arrays.
\item Implement and use a swap function
\item More practice with passing arrays to functions.
\end{itemize}
\section*{Description}
\label{sec-3}

A selection sort searches an array looking for the smallest element.
Then the smallest element is swapped with the first element of the
array.  The process is repeated for the subarray beginning with the
second element of the array on the second pass.  Each pass of the
array results in one element being swapped into its proper location.
This algorithm is also an $O(n^2)$ algorithm, for an array of $n$
elements, $n - 1$ passes must be made, and for each subarray $n - 1$
comparisons must be made to find the smallest value.

In this assignment you are to implement the selection sort algorithm.
To support the selection sort, you will also (re)implement two helper
functions that work on arrays (and which we have seen examples of
before in class).  In particular, you need to create a function that,
given an array and two indexes will swap the values between the two
indexes.  And you need to implement a function that will search an
array, from some beginning start position, and return the location of
the minimum value found in the array after that position.  With these
two functions implemented, you should be able to write the selection
sort function that calls these two helper functions, in order to find
the minimum value and then swap it into place.



Perform the following tasks:

\begin{enumerate}
\item Write a function called \verb~findMinimumInSubarray~.  This function
   takes an array of ints as its first parameter and an integer
   parameter indicating the size of the array.  The third parameter is
   an integer position which indicates the location to start searching
   within the array.  This function should return a single integer.
   The returned value should be the index location of the minimum
   value in the subarray that was searched.  You will use this
   location in your sorting function to swap the minimum value into
   its proper place.
\item Write a function called \verb~swapArrayLocations~.  This functions takes
   an array of ints.  It should then take two integer parameters,
   which represent two (valid) indexes into the array.  The function
   should swap the values in the two indicated locations as a result
   of calling the function.  This function does not return an explicit
   result (it is a void function) but as a side effect, the indicated
   locations will be swapped after this function is called.
\item Write an implementation of the selection sorting algorithm.  Name
   your function \verb~selectionSort~.  This function takes an array of
   integers and the size of the array as its first two parameters.
   This function is a void function.  This function should implement
   the selection sort algorithm as described above.  The result of
   calling this function is that the array of integers should be
   sorted in ascending order after calling the function.
\item In your \verb~main~ function, create an array of size 20.  Initialize
   the array with value in the range from 1 to 100 (inclusive).  You
   may (re)use previous functions from class we have created to
   initialize an array of random values for this task.  Call your
   sort function to sort the values in the array, and then display the contents of 
   your array on standard output.
\end{enumerate}

Your program output should look exactly like the following when I run
your program. 



\textbf{NOTE}: Now that our programs have more functions than just the
\verb~main()~ function, the use of the function headers becomes meaningful
and required.  Make sure that all of your functions have function
headers preceding them that document the purpose of the functions, and
the input parameters and return value of the function.
\section*{Assignment Submission}
\label{sec-4}


An eCollege dropbox has been created for this assignment.  You should
upload your version of the assignment to the eCollege dropbox named
\verb~Assg 08 Selection Sort~ created for this submission.  Work
submitted by the due date will be considered for evaluation.
\section*{Requirements and Grading Rubrics}
\label{sec-5}
\subsection*{Program Execution, Output and Functional Requirements}
\label{sec-5-1}


\begin{enumerate}
\item Your program must compile, run and produce some sort of output to
   be graded. 0 if not satisfied.
\item 50+ pts.  Your implementation of the two helper functions must be
   correct, and the functions must work as described.  This is half of
   your grade, so if you are having trouble getting the sort to work,
   make sure you at least have these simple helper functions written
   and working correctly.
\item 40+ pts. Your selection sort implementation must work.  The
   selection sort algorithm must be implemented as described in our
   assignment description.
\item 10+ pts. Your main function should create and initialize the
   desired array of random integers as described.  After calling your
   sort function, your program should display the contents of the
   array of integers to standard output.
\end{enumerate}
\subsection*{Program Style}
\label{sec-5-2}


Your programs must conform to the style and formatting guidelines
given for this course.  The following is a list of the guidelines that
are required for the assignment to be submitted this week.

\begin{enumerate}
\item The file header for the file with your name and program information
  and the function header for your main function must be present, and
  filled out correctly.
\item A function header must be present for all functions you define.
   You must document the purpose, input parameters and return values
   of all functions.  Your function headers must be formatted exactly
   as shown in the style guidelines for the class.
\item You must indent your code correctly and have no embedded tabs in
  your source code. (Don't forget about the Visual Studio Format
  Selection command).
\item You must not have any statements that are hacks in order to keep
   your terminal from closing when your program exits (e.g. no calls
   to system() ).
\item You must have a single space before and after each binary operator.
\item You must have a single blank line after the end of your declaration
  of variables at the top of a function, before the first code
  statement.
\item You must have a single blank space after , and \verb~;~ operators used as a
  separator in lists of variables, parameters or other control
  structures.
\item You must have opening \verb~{~ and closing \verb~}~ for control statement blocks
  on their own line, indented correctly for the level of the control
  statement block.
\item All control statement blocks (if, for, while, etc.) must have \verb~{~
   \verb~}~ enclosing them, even when they are not strictly necessary
   (when there is only 1 statement in the block).
\item You should attempt to use meaningful variable and function names in
   your program, for program clarity.  Of course, when required, you
   must name functions, parameters and variables as specified in the
   assignments.  Variable and function names must conform to correct
   \verb~camelCaseNameingConvention~ .
\end{enumerate}

Failure to conform to any of these formatting and programming practice
guidelines for this assignment will result in at least 1/3 of the
points (33) for the assignment being removed for each guideline that
is not followed (up to 3 before getting a 0 for the
assignment). Failure to follow other class/textbook programming
guidelines may result in a loss of points, especially for those
programming practices given in our Deitel textbook that have been in
our required reading so far.

\end{document}
