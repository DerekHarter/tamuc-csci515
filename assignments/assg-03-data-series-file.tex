% Created 2015-01-30 Fri 11:39
\documentclass[11pt]{article}
\usepackage[utf8]{inputenc}
\usepackage[T1]{fontenc}
\usepackage{fixltx2e}
\usepackage{graphicx}
\usepackage{longtable}
\usepackage{float}
\usepackage{wrapfig}
\usepackage{soul}
\usepackage{textcomp}
\usepackage{marvosym}
\usepackage{wasysym}
\usepackage{latexsym}
\usepackage{amssymb}
\usepackage{hyperref}
\tolerance=1000
\usepackage{minted}
\usepackage{minted}
\usemintedstyle{default}
\providecommand{\alert}[1]{\textbf{#1}}

\title{Assg 03: Process a File of Scientific Series of Data}
\author{CSci 515 Spring 2015}
\date{2015-01-30}
\hypersetup{
  pdfkeywords={},
  pdfsubject={Assg 03},
  pdfcreator={Emacs Org-mode version 7.9.3f}}

\begin{document}

\maketitle


\section*{Dates:}
\label{sec-1}


\begin{center}
\begin{tabular}{ll}
 Due:  &  Tuesday February 10, by Midnight  \\
\end{tabular}
\end{center}
\section*{Objectives}
\label{sec-2}

\begin{itemize}
\item Be able to open a serial text file for reading and writing.
\item Process more complex data items from a file of delimiter separated values.
\item More practice with I/O stream formatting manipulators.
\item Gain more practice using C control structures for implementing
  algorithms.
\item Implement formula calculations into a typical data processing task
  written in C.
\end{itemize}
\section*{Description}
\label{sec-3}

In this assignment, we will be writing a filter, that will filter a file of
delimiter separated values of data gathered from an experiment, perform
some simple data analysis on the data, and save the results to a new file.

The input file you will be given has the following format (truncated,
the real files will have many more items than 7 to process):


\begin{verbatim}
STRANGE
trial          x               y               z        class
00001    4.23169         4.68996        -8.86438        STRANGE
00002    5.43040        -3.59577        -4.71896        UP
00003    0.37792         3.34626        -0.22265        CHARM
00004    5.35208         3.96738         1.77813        UP
00005    0.90207        -0.38525         4.66088        CHARM
00006   -4.67474        -4.18064         1.65754        UP
00007    4.27666         4.56251        -8.56897        STRANGE
...
\end{verbatim}

The first line represents a filter class upon which the data is to be
filtered.  More on this below.

The next line after the filter is a header for the columns/features of
the data.  After the header are the actual data trials in the
experiment.  Column 1 is simply the trial number for the row of data.
Columns 2, 3 and 4 represent x, y and z measured positions of a
physical component in some experiment.  Column 5 represents a feature
category, and will be a string.

Your task is to process this file.  We are only interested in experimental
trials that are of a particular class.  In the above example, the first
line indicates that we need to process the \verb~STRANGE~ TRIALS.  \textbf{NOTE}: I
can and will use your code on different input files, that will filter
on a different class, so you need to read in this first line from the
file, and only process subsequent experimental trials of that type.

Each trial of the experiment has recorded the x, y and z position.  We need
to calculate the distance traveled between successive trials of the
target filter class, using the standard euclidean distance:

$d = \sqrt{(x_1 - x_2)^2 + (y_1 - y_2)^2 + (z_1 - z_2)^}$
\section*{Assignment Submission}
\label{sec-4}


An eCollege dropbox has been created for this assignment.  You should
upload your version of the out of class assignment by the end of
Tuesday 2/3 (midnight) to the dropbox named \verb~Assg 03 Scientific Data File~.
Late submissions will not be graded.
\section*{Requirements}
\label{sec-5}

Your programs must conform to the style and formatting guidelines
given for this course.  The following is a list of the guidelines that
are required for the assignment to be submitted this week.

\begin{itemize}
\item The file header and function header for your main function must be present, and filled out correctly.
\item You must indent your code correctly and have no embedded tabs in your source code. (Don't forget about the Visual Studio Format Selection command).
\item You must not have any statements that are hacks in order to keep your terminal from closing when your program exits.
\item You must have a single space before and after each binary operator.
\item You must have a single blank line after the end of your declaration
  of variables at the top of a function, before the first code
  statement.
\end{itemize}

Failure to conform to any of these formatting and programming practice
guidelines for this assignment will result in a grade of 0 for the
assignment, and your program being returned with an indication of
which of these items your program violates.  Failure to follow other
class/textbook programming guidelines may result in a loss of points,
especially for those good programming practices given in chapters 1-5
of our textbook which you should have read by now.

\end{document}
