% Created 2015-01-30 Fri 20:40
\documentclass[11pt]{article}
\usepackage[utf8]{inputenc}
\usepackage[T1]{fontenc}
\usepackage{fixltx2e}
\usepackage{graphicx}
\usepackage{longtable}
\usepackage{float}
\usepackage{wrapfig}
\usepackage{rotating}
\usepackage[normalem]{ulem}
\usepackage{amsmath}
\usepackage{textcomp}
\usepackage{marvosym}
\usepackage{wasysym}
\usepackage{amssymb}
\usepackage{hyperref}
\tolerance=1000
\usepackage{minted}
\usepackage{minted}
\usepackage{amsmath}
\usemintedstyle{default}
\author{CSci 515 Spring 2015}
\date{CSci 515 Spring 2014 \textit{<2015-01-30 Fri>}}
\title{Assg 03: Process a File of Scientific Series of Data}
\hypersetup{
  pdfkeywords={},
  pdfsubject={Assg 03},
  pdfcreator={Emacs 24.3.1 (Org mode 8.2.4)}}
\begin{document}

\maketitle

\section*{Dates:}
\label{sec-1}
\begin{center}
\begin{tabular}{ll}
Due: & Tuesday February 10, by Midnight\\
\end{tabular}
\end{center}
\section*{Objectives}
\label{sec-2}
\begin{itemize}
\item Be able to open a serial text file for reading and writing.
\item Process more complex data items from a file of delimiter separated values.
\item More practice with I/O stream formatting manipulators.
\item Gain more practice using C control structures for implementing
algorithms.
\item Implement formula calculations into a typical data processing task
written in C.
\end{itemize}
\section*{Description}
\label{sec-3}
In this assignment, we will be writing a filter, that will filter a file of
delimiter separated values of data gathered from an experiment, perform
some simple data analysis on the data, and save the results to a new file.

The input file you will be given has the following format (truncated,
the real files will have many more items than 10 to process):

\begin{verbatim}
STRANGE
trial	       x	       y	       z	class
00001	 4.23169	 4.68996	-8.86438	STRANGE
00002	 5.43040	-3.59577	-4.71896	UP
00003	 0.37792	 3.34626	-0.22265	CHARM
00004	 5.35208	 3.96738	 1.77813	UP
00005	 0.90207	-0.38525	 4.66088	CHARM
00006	-4.67474	-4.18064	 1.65754	UP
00007	 4.27666	 4.56251	-8.56897	STRANGE
00008	 4.17155	 4.33587	-8.61905	STRANGE
00009	 2.52960	-3.98237	 8.60114	DOWN
00010	 0.41158	 5.61738	 4.29349	UP
...
\end{verbatim}

The first line represents a filter class upon which the data is to be
filtered.  More on this below.

The next line after the filter is a header for the columns/features of
the data.  After the header are the actual data trials in the
experiment.  Column 1 is simply the trial number for the row of data.
Columns 2, 3 and 4 represent x, y and z measured positions of a
physical component in some experiment.  Column 5 represents a feature
category, and will be a string.

Your task is to process this file.  We are only interested in experimental
trials that are of a particular class.  In the above example, the first
line indicates that we need to process the \verb~STRANGE~ trials.  \textbf{NOTE}: I
can and will use your code on different input files, that will filter
on a different class, so you need to read in this first line from the
file, and only process subsequent experimental trials of that type.

Each trial of the experiment has recorded the x, y and z position.  We need
to calculate the distance traveled between successive trials of the
target filter class, using the standard euclidean distance:

\begin{equation}
\begin{split}
d = \sqrt{(x_2 - x_1)^2 + (y_2 - y_1)^2 + (z_2 - z_1)^}
\end{split}
\end{equation}

So for example, there are only 3 trials (1, 7 and 8) in our example data of the
target \verb~STRANGE~ class.  The distance traveled for the \verb~STRANGE~ item between
trials 1 and 7 would be:

\begin{equation}
\begin{split}
d &= \sqrt{(4.27666 - 4.23169)^2 + (4.56251 - 4.68996)^2 + (-8.56897 - -8.86438)^2} \\\\
  &= 0.32486
\end{split}
\end{equation}

In addition to processing the file, you will open up a new file and save the results
of the filtering and data processing to this file.  The file I will give you to test with
for input will be called \verb~assg-03-data-in.dsv~.  You should put the results into a file
you open and write to called \verb~assg-03-data-out.dsv~.  Your results need to be formatted
to look exactly like this output (using IOStream manipulators to format the output
as required):

\begin{verbatim}
00007   0.32486
00008   0.25480
\end{verbatim}

The first column is simply the trial identifier of the $x_2, y_2, z_2$
from the input (e.g.  the second pair of coordinates used in the
distance calculation), and the second column is the calculated
distance.  If you input file has $N$ trials of the class being
filtered for, then your output file will end up with $N-1$ distance
values being calculated.  The first column should be in a field width
of 5, with leading 0's (as in the input).  The second column is in a
field width of 10, with 5 decimal places of precision and in fixed
notation.

In summary you need to perform the following tasks for this assignment:

\begin{itemize}
\item Open the input file, and read in the trial class to filter on (and
skip over the header).
\item Open the input file and process it one line at a time.
\item Open an output file, and output the formatted results of your
distance calculations after filtering for successive trials.
\item Calculate distance between successive trials of the trial class
of interest.
\end{itemize}
\section*{Assignment Submission}
\label{sec-4}

An eCollege dropbox has been created for this assignment.  You should
upload your version of the out of class assignment by the end of
Tuesday 2/10 (midnight) to the dropbox named \verb~Assg 03 Time Series File~.
Late submissions will not be graded.
\section*{Requirements and Grading Rubrics}
\label{sec-5}

\subsection*{Program Execution, Output and Functional Requirements}
\label{sec-5-1}

\begin{enumerate}
\item Your program must compile, run and produce some sort of output to be
graded. 0 if not satisfied.
\item 10+ pts. Your program must successfully open the file from the current
working directory.
\item 10+ pts. Your program must successfully open the output file for writing, and
write some contents to it.
\item 20 pts. Your program must successfully read the lines from the file in the
correct order and attempt to process them.
\item 10+ pts. Your program must use I/O manipulators to achieve the correct output
format to the output file.
\item 25+ pts. Your program must correctly filter the input, and only calculate and
output results for the trials of the correct class.
\item 25+ pts. Your program must produce the correct distance values between successive
trials.
\end{enumerate}


\subsection*{Program Style}
\label{sec-5-2}

Your programs must conform to the style and formatting guidelines given for this course.
The following is a list of the guidelines that are required for the lab to be submitted
this week.

\begin{enumerate}
\item The file header for the file with your name and program information
and the function header for your main function must be present, and
filled out correctly.
\item You must indent your code correctly and have no embedded tabs in
your source code. (Don't forget about the Visual Studio Format
Selection command).
\item You must not have any statements that are hacks in order to keep
your terminal from closing when your program exits.
\item You must have a single space before and after each binary operator.
\item You must have a single blank line after the end of your declaration
of variables at the top of a function, before the first code
statement.
\item You must have a single blank space after , and \verb~;~ operators used as a
separator in lists of variables, parameters or other control
structures.
\item You must have opening \verb~{~ and closing \verb~}~ for control statement blocks
on their own line, indented correctly for the level of the control
statement block.
\end{enumerate}

Failure to conform to any of these formatting and programming practice
guidelines for this lab will result in at least 1/3 of the points (33)
for the assignment being removed.  Failure to follow other
class/textbook programming guidelines may result in a loss of points,
especially for those programming practices given in our Deitel
textbook that have been in our required reading so far.
% Emacs 24.3.1 (Org mode 8.2.4)
\end{document}
