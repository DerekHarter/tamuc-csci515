% Created 2015-02-06 Fri 15:44
\documentclass[11pt]{article}
\usepackage[utf8]{inputenc}
\usepackage[T1]{fontenc}
\usepackage{fixltx2e}
\usepackage{graphicx}
\usepackage{longtable}
\usepackage{float}
\usepackage{wrapfig}
\usepackage{soul}
\usepackage{textcomp}
\usepackage{marvosym}
\usepackage{wasysym}
\usepackage{latexsym}
\usepackage{amssymb}
\usepackage{hyperref}
\tolerance=1000
\usepackage{minted}
\usepackage{minted}
\usemintedstyle{default}
\providecommand{\alert}[1]{\textbf{#1}}

\title{Lab 04: Temperature Conversion Functions}
\author{CSci 515 Spring 2015}
\date{2015-01-23}
\hypersetup{
  pdfkeywords={},
  pdfsubject={Lab 04},
  pdfcreator={Emacs Org-mode version 7.9.3f}}

\begin{document}

\maketitle


\section*{Dates:}
\label{sec-1}


\begin{center}
\begin{tabular}{ll}
 Due:  &  In Lab, Wednesday February 11, by 4:10 pm (lab end time)  \\
\end{tabular}
\end{center}
\section*{Objectives}
\label{sec-2}

\begin{itemize}
\item Write functions in C that take 1 or more parameters and return a result.
\item Learn about breaking up a larger problem into smaller sub-parts.
\item Create simple functions that use an input parameter.
\item Learn how to return results using the return values from a function.
\item Practice with input and output.
\end{itemize}
\section*{Description}
\label{sec-3}

In this lab we will write two simple functions.  The functions will
take a single parameter as input, and return a single value as their
result.  You are to write functions that convert temperatures
on the Fahrenheit scale into the Celsius scale, and vice verse.

The formula for converting a temperature from Fahrenheit to Celsius is
given by:

$$
C = (F - 32) \cdot \frac{5}{9}
$$

And to convert in the other direction, from Celsius to Fahrenheit:

$$
F = C \cdot \frac{9}{5} + 32
$$

Perform the following tasks:

\begin{enumerate}
\item Write a function named \verb~fahrenheitToCelsius()~.  The function takes
   a single float value as input, and returns a single float value
   as its result.  The input of course represents a temperature in
   degrees Fahrenheit, and your program should use the above formula
   to convert the value to a temperature on the Celsius scale.
\item Write a function named \verb~celsiusToFahrenheit()~.  The function
   takes a single float values as input, and returns a single float
   value as its result.  As you can guess, this function performs
   the other operation, to convert from the Celsius scale back to
   the Fahrenheit scale.
\item In your \verb~main()~ function, prompt the user for a temperature on
   the Fahrenheit scale.  Use the first function to convert this value
   to the Celsius scale and display this to the user.  Then 
   prompt the user for a temperature in degrees Centigrade, and use your
   second function to convert it and display your results.
\end{enumerate}

Your program output should look exactly like the following when I
run your program:


\begin{verbatim}
Enter a value in degrees Fahrenheit, and I will convert it to the Celsius scale: 6
6 degrees Fahrenheit is equal to -14.4444 degress Celsius

Enter a value in degrees Celsius, and I will convert it to the Fahrenheit scale: 22
22 degrees Celsius is equal to 71.6 degress Fahrenheit
\end{verbatim}

\textbf{NOTE}: Now that our programs have more functions than just the
\verb~main()~ function, the use of the function headers becomes meaningful
and required.  Make sure that all of your functions (\verb~main~,
\verb~fahrenheitToCelsius~, \verb~celsiusToFahrenheit~) have function headers
before them that document the purpose of the functions, and the input
values and return value of the function.
\section*{Lab Submission}
\label{sec-4}


An eCollege dropbox has been created for this lab.  You should
upload your version of the lab by the end of lab time to the eCollege
dropbox named \verb~Lab 04 Temperature Conversion~.  Work submitted by the end
of lab will be considered, but after the lab ends you may no longer
submit work, so make sure you submit your best effort by the lab end
time in order to receive credit.
\section*{Requirements and Grading Rubrics}
\label{sec-5}
\subsection*{Program Execution, Output and Functional Requirements}
\label{sec-5-1}


\begin{enumerate}
\item Your program must compile, run and produce some sort of output to be
  graded. 0 if not satisfied.
\item 40+ pts.  Your program must have the 2 required named functions, that 
   accept the required input parameters and return the required values
   (if any).
\item 30+ pts. Your functions must convert temperatures correctly, according to the
   formula given for you for converting between the two scales.
\item 30+ pts. You should prompt the user for values in your \verb~main()~ function, and
   display the results.  The interaction with your program should be as shown
   in the example output above.
\end{enumerate}
\subsection*{Program Style}
\label{sec-5-2}


Your programs must conform to the style and formatting guidelines given for this course.
The following is a list of the guidelines that are required for the lab to be submitted
this week.

\begin{enumerate}
\item The file header for the file with your name and program information
  and the function header for your main function must be present, and
  filled out correctly.
\item A function header must be present for all functions you define.
  You must document the purpose, input parameters and return values
  of all functions.
\item You must indent your code correctly and have no embedded tabs in
  your source code. (Don't forget about the Visual Studio Format
  Selection command).
\item You must not have any statements that are hacks in order to keep
  your terminal from closing when your program exits.
\item You must have a single space before and after each binary operator.
\item You must have a single blank line after the end of your declaration
  of variables at the top of a function, before the first code
  statement.
\item You must have a single blank space after , and \verb~;~ operators used as a
  separator in lists of variables, parameters or other control
  structures.
\item You must have opening \verb~{~ and closing \verb~}~ for control statement blocks
  on their own line, indented correctly for the level of the control
  statement block.
\end{enumerate}

Failure to conform to any of these formatting and programming practice
guidelines for this lab will result in at least 1/3 of the points (33)
for the assignment being removed.  Failure to follow other
class/textbook programming guidelines may result in a loss of points,
especially for those programming practices given in our Deitel
textbook that have been in our required reading so far.

\end{document}
