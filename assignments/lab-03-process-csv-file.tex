% Created 2015-01-29 Thu 20:08
\documentclass[11pt]{article}
\usepackage[utf8]{inputenc}
\usepackage[T1]{fontenc}
\usepackage{fixltx2e}
\usepackage{graphicx}
\usepackage{longtable}
\usepackage{float}
\usepackage{wrapfig}
\usepackage{rotating}
\usepackage[normalem]{ulem}
\usepackage{amsmath}
\usepackage{textcomp}
\usepackage{marvosym}
\usepackage{wasysym}
\usepackage{amssymb}
\usepackage{hyperref}
\tolerance=1000
\usepackage{minted}
\usepackage{minted}
\usemintedstyle{default}
\author{CSci 515 Spring 2015}
\date{CSci 515 Spring 2014 \textit{<2015-01-23 Fri>}}
\title{Lab 03: Processing Data Files}
\hypersetup{
  pdfkeywords={},
  pdfsubject={Lab 02},
  pdfcreator={Emacs 24.3.1 (Org mode 8.2.4)}}
\begin{document}

\maketitle

\section*{Dates:}
\label{sec-1}
\begin{center}
\begin{tabular}{ll}
Due: & In Lab, Wednesday February 4, by 4 pm (lab end time)\\
\end{tabular}
\end{center}
\section*{Objectives}
\label{sec-2}
\begin{itemize}
\item Be able to Open a serial text file for reading.
\item Be able to process a simple text file of comma separated values.
\item Use I/O formatting manipulators, for reading and writing formatted data.
\item Use loops to read and process line oriented files.
\item Get some more practice implementing mathematical formula for data processing tasks.
\end{itemize}
\section*{Description}
\label{sec-3}
Plain text files containing tables of information are very common,
minimal representations of data sets needed for processing.  The
simplest type of formatting of a table of numbers, is to separate the
columns or features of the table using spaces, tabs or commas. In
general these are known as delimiter separated value files (DSV).  The
most common is to use commas (known as a comma separated value file or
CSV files).  The C++ IOStream operators make it trivial to process
space separated and tab separated value files, since it uses
whitespace (spaces and tabs) by default as the delimiter when breaking
apart a stream to automatically parse and convert into input
variables. In a space or tab separated file, the values separated by
space are the features, and each row or line is an individual record
or trial in an experiment.

For this lab you are to read in records from a space separated file, and
do some processing of the data.  I will give you a space separated file
to use.  The file you are to open and process looks like this:

\begin{verbatim}
 1.03925	-0.0466664	 6.78488	
 5.733837	-7.16068	 -2.792169	
-4.5309	-0.1812	6.10121759	
  1.437688	2.798135755	  4.12021	
1.22328	 -1.5012	  8.99924	
-1.53216	 -5.395889	  4.0939	
  8.6847	  5.44601	 -7.70818	
-4.76181	8.362	 -0.389249517	
 2.298	4.148659	 -0.757	
  3.3200104	  3.953700223	-5.8364352
\end{verbatim}

You need to perform the following tasks:

\begin{itemize}
\item Open the file and read it one line at a time.
\item Output the original contents of the file, but with cleaned up
formatting.  You will output each column in a field of width 15
characters (\verb~setw()~).  All numbers will be printed with exactly 5
decimal point digits of precision (\verb~setprecision()~).  All of the
floating point numbers should be displayed using scientific notation
(\verb~scientific~).  See the example output to get details on what
this should look like.
\item You will add an index as a new first column of each line, in a field
of width 5.
\item You will add a header to the original data, as shown in the example
output.
\item In addition to cleaning up and formatting the data, you will do some
processing of the data.  You need to determine the minimum value of
all values seen in column 1, the maximum of the values in column 2
and the average of the values in column 3.  You will report these
values after displaying the cleaned contents of the file, with some
others.  Again see the example output to determine exactly which
summary information you must produce and in what format.
\end{itemize}

Here is an example of the correct, cleaned up and summarized data for
the previously shown input file:

\begin{verbatim}
Trial      Feature-1      Feature-2      Feature-3
    1    1.03925e+00   -4.66664e-02    6.78488e+00
    2    5.73384e+00   -7.16068e+00   -2.79217e+00
    3   -4.53090e+00   -1.81200e-01    6.10122e+00
    4    1.43769e+00    2.79814e+00    4.12021e+00
    5    1.22328e+00   -1.50120e+00    8.99924e+00
    6   -1.53216e+00   -5.39589e+00    4.09390e+00
    7    8.68470e+00    5.44601e+00   -7.70818e+00
    8   -4.76181e+00    8.36200e+00   -3.89250e-01
    9    2.29800e+00    4.14866e+00   -7.57000e-01
   10    3.32001e+00    3.95370e+00   -5.83644e+00

        Number of Trials:        10
    Minimum of Feature 1:  -4.76181
    Maximum of Feature 2:   8.36200
    Average of Feature 3:   1.26164
\end{verbatim}

\section*{Lab Submission}
\label{sec-4}

An eCollege dropbox has been created for this lab.  You should
upload your version of the lab by the end of lab time to the eCollege
dropbox named \verb~Lab 03 Process DSV File~.  Work submitted by the end
of lab will be considered, but after the lab ends you may no longer
submit work, so make sure you submit your best effort by the lab end
time in order to receive credit.
\section*{Requirements and Grading Rubrics}
\label{sec-5}

\subsection*{Program Execution, Output and Functional Requirements}
\label{sec-5-1}

\begin{itemize}
\item Your program must compile, run and produce some sort of output to be
graded. 0 if not satisfied.
\item Your program must successfully open the file from the current
working directory. 10 or more points.
\item Your program must successfully read the lines from the file in the
correct order and attempt to process them. 20 points.
\item Your program must use I/O manipulators to achieve the correct output
format of the cleaned up original data.  30 or more points.
\item Your program must include a header for the original data, properly
formatted. 10 points.
\item Your program must produce the correct summary information values. 30
or more points.
\item Your program must format the summary information as required, again
using I/O formatting manipulators.  20 or more points.
\end{itemize}

\subsection*{Program Style}
\label{sec-5-2}

Your programs must conform to the style and formatting guidelines given for this course.
The following is a list of the guidelines that are required for the lab to be submitted
this week.

\begin{itemize}
\item The file header for the file with your name and program information
and the function header for your main function must be present, and
filled out correctly.
\item You must indent your code correctly and have no embedded tabs in
your source code. (Don't forget about the Visual Studio Format
Selection command).
\item You must not have any statements that are hacks in order to keep
your terminal from closing when your program exits.
\item You must have a single space before and after each binary operator.
\item You must have a single blank line after the end of your declaration
of variables at the top of a function, before the first code
statement.
\item You must have a single blank space after , and ; operators used as a
separator in lists of variables, parameters or other control
structures.
\end{itemize}

Failure to conform to any of these formatting and programming practice
guidelines for this lab will result at least 1/3 of the points (33)
for the assignment being removed.  Failure to follow other
class/textbook programming guidelines may result in a loss of points,
especially for those programming practices given in our Deitel
textbook that have been in our required reading so far.

\section*{Grading Rubrics}
\label{sec-6}
% Emacs 24.3.1 (Org mode 8.2.4)
\end{document}
