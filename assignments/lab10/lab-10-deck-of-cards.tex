% Created 2015-04-07 Tue 15:34
\documentclass[11pt]{article}
\usepackage[utf8]{inputenc}
\usepackage[T1]{fontenc}
\usepackage{fixltx2e}
\usepackage{graphicx}
\usepackage{longtable}
\usepackage{float}
\usepackage{wrapfig}
\usepackage{soul}
\usepackage{textcomp}
\usepackage{marvosym}
\usepackage{wasysym}
\usepackage{latexsym}
\usepackage{amssymb}
\usepackage{hyperref}
\tolerance=1000
\usepackage{minted}
\usepackage{minted}
\usemintedstyle{default}
\providecommand{\alert}[1]{\textbf{#1}}

\title{Lab 10: Deck of Cards}
\author{CSci 515 Spring 2015}
\date{2015-04-07}
\hypersetup{
  pdfkeywords={},
  pdfsubject={Lab 10},
  pdfcreator={Emacs Org-mode version 7.9.3f}}

\begin{document}

\maketitle


\section*{Dates:}
\label{sec-1}


\begin{center}
\begin{tabular}{ll}
 Due:  &  In Lab, Wednesday April 8, by 4:05 pm (lab end time)  \\
\end{tabular}
\end{center}
\section*{Objectives}
\label{sec-2}

\begin{itemize}
\item Get practice in declaring structs and enums user defined
  types.
\item Become familiar with defining arrays of structs.
\item Learn how to pass arrays of structs to functions and process
  the structs.
\end{itemize}
\section*{Description}
\label{sec-3}

In this lab we will create a basic data structure to represent
an array of standard playing cards.  A deck of playing
cards contains 52 cards.  There are 4 different suits
in a standard deck of cards (HEARTS, DIAMONDS, CLUBS, SPADES).
For each suite there is one card each of 13 possible face
values (ACE, DEUCE, THREE, FOUR, FIVE, SIX, SEVEN, EIGHT, NINE,
TEN, JACK, QUEEN, KING).  You will define enumerated types
to represent the possible suits and face values that cards
can have.  Then you need to define a \verb~struct~ to represent
a single playing card. Each individual playing card has
a single suit, and is of some particular face value.

Once you have defined your user defined types to represent a card, we
will simulate the creation of a deck of cards.  You will write a
function that initializes an array of 52 Card structures to a standard
deck of cards.

Perform the following tasks:

\begin{enumerate}
\item Define enumerated types called Suit and Face.  Define each \verb~enum~
   using the list of discrete possible values shown above.
\item Declare a \verb~struct~ called Card.  Using your Suit and Face
   enumerated type, define the Card type to represent
   a single card.
\item Write a function called \verb~initDeckOfCards~.  This function will
   expect an array of Card structures as input.  It will assume that
   the array has been initialized to hold exactly 52 cards, thus you
   do not need to pass in the size of this array to the function.  The
   function should initialize the array of 52 cards
   correctly. (e.g. one of each possible Face/Suit combination in the
   deck).
\item In your main function, show an example of declaring an array of
   52 Card structures.  Demonstrate calling your init function to
   initialize your array of cards.
\item If you have time, and for extra credit, in main display the
   52 resulting cards from your deck.  Or even better, create
   a function called \verb~displayDeckOfCards~ that takes an array
   of cards and displays them.
\end{enumerate}



\textbf{NOTE}: Now that our programs have more functions than just the
\verb~main()~ function, the use of the function headers becomes meaningful
and required.  Make sure that all of your functions have function
headers preceding them that document the purpose of the functions, and
the input parameters and return value of the function.
\section*{Lab Submission}
\label{sec-4}


An eCollege dropbox has been created for this lab.  You should upload
your version of the lab by the end of lab time to the eCollege dropbox
named \verb~Lab 10 Deck of Cards~.  Work submitted by the end of
lab will be considered, but after the lab ends you may no longer
submit work, so make sure you submit your best effort by the lab end
time in order to receive credit.
\section*{Requirements and Grading Rubrics}
\label{sec-5}
\subsection*{Program Execution, Output and Functional Requirements}
\label{sec-5-1}


\begin{enumerate}
\item Your program must compile, run and produce some sort of output to be
  graded. 0 if not satisfied.
\item 33+ pts.  You must declare the enumerated and struct types as
   specified.
\item 33+ pts.  Your program must have the required initialization named
   function, that accepts the required input parameters and return the
   required values (if any).  The function must initialize the
   array of cards correctly.
\item 33+ pts. You should demonstrate creating an array of your Card
   structures, and calling your initialization function in \verb~main~.
\item 5+ extra credit pts.  Display the resulting deck of cards on
   standard output.  Use a function that takes the array of cards
   for the full 5 points, and displays strings for the face/suit
   (not integer values).
\end{enumerate}
\subsection*{Program Style}
\label{sec-5-2}


Your programs must conform to the style and formatting guidelines given for this course.
The following is a list of the guidelines that are required for the lab to be submitted
this week.

\begin{enumerate}
\item The file header for the file with your name and program information
  and the function header for your main function must be present, and
  filled out correctly.
\item A function header must be present for all functions you define.
   You must document the purpose, input parameters and return values
   of all functions.  Your function headers must be formatted exactly
   as shown in the style guidelines for the class.
\item You must indent your code correctly and have no embedded tabs in
  your source code. (Don't forget about the Visual Studio Format
  Selection command).
\item You must not have any statements that are hacks in order to keep
   your terminal from closing when your program exits (e.g. no calls
   to system() ).
\item You must have a single space before and after each binary operator.
\item You must have a single blank line after the end of your declaration
  of variables at the top of a function, before the first code
  statement.
\item You must have a single blank space after , and \verb~;~ operators used as a
  separator in lists of variables, parameters or other control
  structures.
\item You must have opening \verb~{~ and closing \verb~}~ for control statement blocks
  on their own line, indented correctly for the level of the control
  statement block.
\item All control statement blocks (if, for, while, etc.) must have \verb~{~
   \verb~}~ enclosing them, even when they are not strictly necessary
   (when there is only 1 statement in the block).
\item You should attempt to use meaningful variable and function names in
   your program, for program clarity.  Of course, when required, you
   must name functions, parameters and variables as specified in the
   assignments.  Variable and function names must conform to correct
   \verb~camelCaseNameingConvention~ .
\end{enumerate}

Failure to conform to any of these formatting and programming practice
guidelines for this lab will result in at least 1/3 of the points (33)
for the assignment being removed for each guideline that is not
followed (up to 3 before getting a 0 for the assignment). Failure to
follow other class/textbook programming guidelines may result in a
loss of points, especially for those programming practices given in
our Deitel textbook that have been in our required reading so far.

\end{document}
