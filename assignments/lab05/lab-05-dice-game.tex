% Created 2015-02-11 Wed 20:41
\documentclass[11pt]{article}
\usepackage[utf8]{inputenc}
\usepackage[T1]{fontenc}
\usepackage{fixltx2e}
\usepackage{graphicx}
\usepackage{longtable}
\usepackage{float}
\usepackage{wrapfig}
\usepackage{rotating}
\usepackage[normalem]{ulem}
\usepackage{amsmath}
\usepackage{textcomp}
\usepackage{marvosym}
\usepackage{wasysym}
\usepackage{amssymb}
\usepackage{hyperref}
\tolerance=1000
\usepackage{minted}
\usepackage{minted}
\usemintedstyle{default}
\author{CSci 515}
\date{Spring 2015 \textit{<2015-02-11 Wed>}}
\title{Lab 05: Dice Game}
\hypersetup{
  pdfkeywords={},
  pdfsubject={Lab 05},
  pdfcreator={Emacs 24.3.1 (Org mode 8.2.4)}}
\begin{document}

\maketitle

\section*{Dates:}
\label{sec-1}
\begin{center}
\begin{tabular}{ll}
Due: & In Lab, Wednesday February 18, by 4:05 pm (lab end time)\\
\end{tabular}
\end{center}
\section*{Objectives}
\label{sec-2}
\begin{itemize}
\item More practice writing C functions
\item Use standard library rand() functions to implement random simulations
\item Practice creating user defined functions that break problems into smaller pieces
\item Learn about rescaling ranges of values.
\end{itemize}
\section*{Description}
\label{sec-3}
In this lab we will write a program to answer a simple question from
probability.  The question is, if I roll 2 dices, what is the probability
of rolling a sum of either a 7 or 11 if I add up the faces of the 2 dice
rolled.

You will first of all create a function that will return a random
number between 1 and 6 when called.  Then you need to write another
function to implement you simulation.  The simulation should roll N
pairs of dice, and keep track of the number of times M out of the N
rolls that the sum of the two dice was either a 7 or an 11.  The
function will return this result M/N as a floating point ratio of the
result (make sure you avoid integer division problems when returning
the result).  For example, if we roll 1000 pairs of dice, and it comes
up with a 7 or an 11 on 676 out of those 1000 trials, then the
function will return 676 / 1000 = 0.676


Perform the following tasks:

\begin{enumerate}
\item Write a function named \verb~rollDice()~.  This function takes no
parameters as input.  It will return a single integer value in the
range from 1 to 6.  The function should use the rand()
system call to generate a random number, and then scale the
number to the appropriate range [1,6] and return it.

\item Write a function call \verb~simulateDiceGame()~.  This function will
take a single integer called \verb~numTrials~ as the input.  This is the
number of pairs of dice it is to simulate rolling.  You should
implement rolling \verb~numTrials~ pairs of dice by writing a for loop
that uses an index that ranges from 1 up to \verb~numTrials~.  You need
to call your \verb~rollDice()~ function two times, to simulate rolling 2
pairs of dice.  You then need to sum up the values of the two
random dice rolls you receive.  You need to keep track of the
number of trials whose sum is either 7 or 11, by for example
declaring a variable called \verb~successes~ that is initially 0, and
that is incremented by 1 each time the sum of the two dice rolls is
as indicated.  The result of your \verb~simulateDiceGame()~ function should
be a single floating point number, between [0.0, 1.0], that is the
result of dividing the number of successes, by the total number of
simulated dice throwing trials that were performed.

\item In your \verb~main()~ function, prompt the user to enter two values.
First of all, ask the user to enter a random number to use as the
seed for the srand() function.  Use the value entered to seed the
random number generator.  Then ask the user to enter the number of
trials to run. Using this number of trials values, call your
\verb~simulateDiceGame()~ with this number of trials, and get back the
resulting ratio.  This ratio should represent an estimate of the
probability of throwing a 7 or an 11 when rolling 2 dice.
\end{enumerate}

Your program output should look something close to the following when I
run your program:

\begin{verbatim}
Enter a seed with which to initialize the random number generator: 42

I will simulate rolling a pair of dice and estimate the probability
of rolling a 7 or an 11.  Enter the number of trials to run: 1000000

I tried 1000000 experiments.  The estimated probability
of rolling a 7 or an 11 is: 0.2224999964237213
\end{verbatim}

If you implement your program correctly, then giving the same value
for the seed will end up resulting in the same estimated probability
on subsequent runs of your program.  However different seeds will give
slightly different results.  The true probability of getting a sum of
7 or 11 when rolling 2 dice is $\frac{8}{36} \approx 0.22222222$, so
any experiment you run should end up with a value somewhere around
that true probability.

\textbf{NOTE}: Now that our programs have more functions than just the
\verb~main()~ function, the use of the function headers becomes meaningful
and required.  Make sure that all of your functions (\verb~main~,
\verb~rollDice~, \verb~simulateDiceGame~) have function headers preceding them
that document the purpose of the functions, and the input values and
return value of the function.
\section*{Lab Submission}
\label{sec-4}

An eCollege dropbox has been created for this lab.  You should
upload your version of the lab by the end of lab time to the eCollege
dropbox named \verb~Lab 05 Dice Game~.  Work submitted by the end
of lab will be considered, but after the lab ends you may no longer
submit work, so make sure you submit your best effort by the lab end
time in order to receive credit.
\section*{Requirements and Grading Rubrics}
\label{sec-5}

\subsection*{Program Execution, Output and Functional Requirements}
\label{sec-5-1}

\begin{enumerate}
\item Your program must compile, run and produce some sort of output to be
graded. 0 if not satisfied.
\item 40+ pts.  Your program must have the 2 required named functions,
that accept the required input parameters and return the required
values (if any).
\item 20+ pts. Your dice rolling function must return a random value within the
correct range each time it is called.
\item 20+ pts. Your dice simulation function must correctly perform the number of indicated
trials, and count up the successful trials from all of the trials performed,
and return the correct probability ratio.  Your ratio must be correct.
\item 20+ pts. You should prompt the user for the number of trials to
perform in your \verb~main()~ function, and display the results.  The
interaction with your program should be as shown in the example
output above.
\end{enumerate}

\subsection*{Program Style}
\label{sec-5-2}

Your programs must conform to the style and formatting guidelines given for this course.
The following is a list of the guidelines that are required for the lab to be submitted
this week.

\begin{enumerate}
\item The file header for the file with your name and program information
and the function header for your main function must be present, and
filled out correctly.
\item A function header must be present for all functions you define.
You must document the purpose, input parameters and return values
of all functions.
\item You must indent your code correctly and have no embedded tabs in
your source code. (Don't forget about the Visual Studio Format
Selection command).
\item You must not have any statements that are hacks in order to keep
your terminal from closing when your program exits.
\item You must have a single space before and after each binary operator.
\item You must have a single blank line after the end of your declaration
of variables at the top of a function, before the first code
statement.
\item You must have a single blank space after , and \verb~;~ operators used as a
separator in lists of variables, parameters or other control
structures.
\item You must have opening \verb~{~ and closing \verb~}~ for control statement blocks
on their own line, indented correctly for the level of the control
statement block.
\end{enumerate}

Failure to conform to any of these formatting and programming practice
guidelines for this lab will result in at least 1/3 of the points (33)
for the assignment being removed for each guideline that is not
followed (up to 3 before getting a 0 for the assignment). Failure to
follow other class/textbook programming guidelines may result in a
loss of points, especially for those programming practices given in
our Deitel textbook that have been in our required reading so far.
% Emacs 24.3.1 (Org mode 8.2.4)
\end{document}
