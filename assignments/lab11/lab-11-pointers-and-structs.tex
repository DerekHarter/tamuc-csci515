% Created 2015-04-14 Tue 15:52
\documentclass[11pt]{article}
\usepackage[utf8]{inputenc}
\usepackage[T1]{fontenc}
\usepackage{fixltx2e}
\usepackage{graphicx}
\usepackage{longtable}
\usepackage{float}
\usepackage{wrapfig}
\usepackage{soul}
\usepackage{textcomp}
\usepackage{marvosym}
\usepackage{wasysym}
\usepackage{latexsym}
\usepackage{amssymb}
\usepackage{hyperref}
\tolerance=1000
\usepackage{minted}
\usepackage{minted}
\usemintedstyle{default}
\providecommand{\alert}[1]{\textbf{#1}}

\title{Lab 11: Pointer Paramters and Struct Pointers}
\author{CSci 515 Spring 2015}
\date{2015-04-14}
\hypersetup{
  pdfkeywords={},
  pdfsubject={Lab 11},
  pdfcreator={Emacs Org-mode version 7.9.3f}}

\begin{document}

\maketitle


\section*{Dates:}
\label{sec-1}


\begin{center}
\begin{tabular}{ll}
 Due:  &  In Lab, Wednesday April 15, by 4:05 pm (lab end time)  \\
\end{tabular}
\end{center}
\section*{Objectives}
\label{sec-2}

\begin{itemize}
\item Become familiar with creating and using pointer variables.
\item Pass pointers to a function.
\item Get practice using structs with pointers.
\end{itemize}
\section*{Description}
\label{sec-3}

In this lab you will practice using pointers.  We will create
some functions that you will pass pointer parameters too
in order to perform some tasks.

Perform the following tasks:

\begin{enumerate}
\item Create a function called \verb~maxInFirst~.  This function will take two
   float pointer variables as its only parameters.  This is a void
   function, so it doesn't return anything explicitly.  The purpose of
   this function is to determine which of the two values pointed to is
   larger, and make sure that the largest of the two is at the memory
   location pointed to by the first parameter.  So if the value
   pointed to by the first parameter is already the largest, then this
   function does nothing.  But if the second pointer value referenced
   is the larger, then this function needs to swap these two values.
\item Declare a \verb~struct~ called \verb~Employee~.  The Employee \verb~struct~ should
   have a name field, of type string, and a salary field, of type
   float.
\item Write a function called \verb~initializeEmployee~.  This function
   will take a pointer to an Employee \verb~struct~ as its only parameter.
   This will be a void function, so it will not return anything
   explicitly.  Have this function prompt the user interactively
   (inside of the function) for an employee name and a floating
   point salary.  You should initialize the fields of the \verb~struct~
   that is passed in as a pointer parameter using the values
   input by the user.
\item In your main function show an example of calling each of your two 
   functions.  For the \verb~maxInFirst~ function, make sure you demonstrate
   calling it two times, one time demonstrating when the values need
   to be swapped, and one time when they done.  Also create an
   Employee variable instance in your main function, and initialize it
   by calling your \verb~initializeEmployee~ function.  Display the two
   fields after the function returns from initializing them.
\end{enumerate}

Example output is shown below.  You should prompt for the employee
name inside of your function, and display the resulting fields after
returning from calling the initialization function.



\begin{verbatim}
Before calling maxInFirst: f1 = 42.42 f2 = 18.18
After  calling maxInFirst: f1 = 42.42 f2 = 18.18
Before calling maxInFirst: f1 = 9.9 f2 = 23.23
After  calling maxInFirst: f1 = 23.23 f2 = 9.9

Employee Name  : Derek Harter
Employee Salary: 32567.15
After initialization Name: Derek Harter salary: 32567.15
\end{verbatim}

\textbf{NOTE}: Now that our programs have more functions than just the
\verb~main()~ function, the use of the function headers becomes meaningful
and required.  Make sure that all of your functions have function
headers preceding them that document the purpose of the functions, and
the input parameters and return value of the function.
\section*{Lab Submission}
\label{sec-4}


An eCollege dropbox has been created for this lab.  You should upload
your version of the lab by the end of lab time to the eCollege dropbox
named \verb~Lab 11 Pointers and Structs~.  Work submitted by the end of
lab will be considered, but after the lab ends you may no longer
submit work, so make sure you submit your best effort by the lab end
time in order to receive credit.
\section*{Requirements and Grading Rubrics}
\label{sec-5}
\subsection*{Program Execution, Output and Functional Requirements}
\label{sec-5-1}


\begin{enumerate}
\item Your program must compile, run and produce some sort of output to be
  graded. 0 if not satisfied.
\item 33+ pts.  You must declare the enumerated and struct types as
   specified.
\item 33+ pts.  Your program must have the required initialization named
   function, that accepts the required input parameters and return the
   required values (if any).  The function must initialize the
   array of cards correctly.
\item 33+ pts. You should demonstrate creating an array of your Card
   structures, and calling your initialization function in \verb~main~.
\item 5+ extra credit pts.  Display the resulting deck of cards on
   standard output.  Use a function that takes the array of cards
   for the full 5 points, and displays strings for the face/suit
   (not integer values).
\end{enumerate}
\subsection*{Program Style}
\label{sec-5-2}


Your programs must conform to the style and formatting guidelines given for this course.
The following is a list of the guidelines that are required for the lab to be submitted
this week.

\begin{enumerate}
\item The file header for the file with your name and program information
  and the function header for your main function must be present, and
  filled out correctly.
\item A function header must be present for all functions you define.
   You must document the purpose, input parameters and return values
   of all functions.  Your function headers must be formatted exactly
   as shown in the style guidelines for the class.
\item You must indent your code correctly and have no embedded tabs in
  your source code. (Don't forget about the Visual Studio Format
  Selection command).
\item You must not have any statements that are hacks in order to keep
   your terminal from closing when your program exits (e.g. no calls
   to system() ).
\item You must have a single space before and after each binary operator.
\item You must have a single blank line after the end of your declaration
  of variables at the top of a function, before the first code
  statement.
\item You must have a single blank space after , and \verb~;~ operators used as a
  separator in lists of variables, parameters or other control
  structures.
\item You must have opening \verb~{~ and closing \verb~}~ for control statement blocks
  on their own line, indented correctly for the level of the control
  statement block.
\item All control statement blocks (if, for, while, etc.) must have \verb~{~
   \verb~}~ enclosing them, even when they are not strictly necessary
   (when there is only 1 statement in the block).
\item You should attempt to use meaningful variable and function names in
   your program, for program clarity.  Of course, when required, you
   must name functions, parameters and variables as specified in the
   assignments.  Variable and function names must conform to correct
   \verb~camelCaseNameingConvention~ .
\item Put the \verb~*~ for pointer variable declarations next to the
   type declaration, with no space between the type and the \verb~*~.
   Also please follow the convention of using \verb~Ptr~ at the end of
   names for pointer variables.
\end{enumerate}

Failure to conform to any of these formatting and programming practice
guidelines for this lab will result in at least 1/3 of the points (33)
for the assignment being removed for each guideline that is not
followed (up to 3 before getting a 0 for the assignment). Failure to
follow other class/textbook programming guidelines may result in a
loss of points, especially for those programming practices given in
our Deitel textbook that have been in our required reading so far.

\end{document}
