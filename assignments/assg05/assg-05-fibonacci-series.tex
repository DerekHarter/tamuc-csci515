% Created 2015-02-13 Fri 10:30
\documentclass[11pt]{article}
\usepackage[utf8]{inputenc}
\usepackage[T1]{fontenc}
\usepackage{fixltx2e}
\usepackage{graphicx}
\usepackage{longtable}
\usepackage{float}
\usepackage{wrapfig}
\usepackage{soul}
\usepackage{textcomp}
\usepackage{marvosym}
\usepackage{wasysym}
\usepackage{latexsym}
\usepackage{amssymb}
\usepackage{hyperref}
\tolerance=1000
\usepackage{minted}
\usepackage{minted}
\usemintedstyle{default}
\providecommand{\alert}[1]{\textbf{#1}}

\title{Assg 05: Fibonacci Series}
\author{CSci 515 Spring 2015}
\date{2015-02-11}
\hypersetup{
  pdfkeywords={},
  pdfsubject={Assg 05},
  pdfcreator={Emacs Org-mode version 7.9.3f}}

\begin{document}

\maketitle


\section*{Dates:}
\label{sec-1}


\begin{center}
\begin{tabular}{ll}
 Due:  &  Tuesday February 24, by Midnight  \\
\end{tabular}
\end{center}
\section*{Objectives}
\label{sec-2}

\begin{itemize}
\item More practice writing C functions
\item Learn about implementing algorithms using iteration and recursion.
\item Practice writing a recursive function.
\item Become more familiar with implementing mathematical series on a computer.
\end{itemize}
\section*{Description}
\label{sec-3}

In this assignment we will write two versions of functions that will
compute the Fibonacci series.  The Fibonacci series is a well studied
series that displays exponential growth.  You may have run across this
series before, the initial values in the Fibonacci series are:

$$
0, 1, 1, 2, 3, 5, 8, 13, 21, ...
$$

where each value is the sum of the proceeding two values in the series
(and the first two values represent the base cases).  Mathematically
we can represent the series as a recursive relationship:

$$
x_n = x_{n-1} + x_{n-2}, \quad x_0 = 0, \quad x_1 = 1
$$

This recursive expression basically says that we can calculate the
$n^{th}$ value of the Fibonacci series by adding the $n-1$ and $n-2$
values in the sequence together, and that the base cases are when $n=0$
then the value of the series is $0$ and when $n=1$ the value of the
series is $1$.

In this assignment you need to write two functions that calculate
and return the $nth$ value of the Fibonacci series.  The functions
will take a single integer parameter $n$ as input.  If $n$ is 0, you
should return 0, and if $n$ is 1 you should return 1.  For all other
values of $n>1$ you should calculate the value to return as given
above and return it.  Your functions will return a single integer
as their result, which will be the $n^{th}$ Fibonacci number they
were asked to calculate.

Perform the following tasks:

\begin{enumerate}
\item Write a function called \verb~nthFibonacciIterative()~.  This version of
   the function takes a single integer as input, and returns an
   integer as its result, as already mentioned.  You should implement
   this function using a loop (when needed) to calculate the $n^{th}$
   Fibonacci number.
\item Write a second function that will also calculate the $n^{th}$ term
   in the Fibonacci sequence.  But your second function will do this
   using a recursive implementation.  Your recursive implementation
   should be called \verb~nthFibonacciRecursive()~.  You should not use any
   loops in this version of the function, you will calculate the
   $n^{th}$ Fibonacci number by recursively calling the
   \verb~nthFibonacciRecursive()~ function itself in the appropriate manner
   in order to calculate the correct term of the sequence.
\item In your \verb~main()~ function, prompt the user to enter a value for $n$.
   You should then call each of your two different implementations, and
   display the result of their calculation of the $n^{th}$ term of the
   Fibonacci sequence, which if you implement the two different versions
   correctly, should always give the same result.  Be careful of using
   large values of $n$, as the recursive implementation especially may
   take quite a while to calculate in cases of large $n$.
\end{enumerate}

Your program output should look something close to the following when I
run your program:



\textbf{NOTE}: Now that our programs have more functions than just the
\verb~main()~ function, the use of the function headers becomes meaningful
and required.  Make sure that all of your functions (\verb~main~,
\verb~nthFibonacciIterative~, \verb~nthFibonacciRecursive~) have function
headers preceding them that document the purpose of the functions, and
the input parameters and return value of the function.
\section*{Assignment Submission}
\label{sec-4}


An eCollege dropbox has been created for this assignment.  You should
upload your version of the assignment to the eCollege dropbox named
\verb~Assg 05 Fibonacci Sequence~ created for this submission.  Work
submitted by the due date will be considered for evaluation.
\section*{Requirements and Grading Rubrics}
\label{sec-5}
\subsection*{Program Execution, Output and Functional Requirements}
\label{sec-5-1}


\begin{enumerate}
\item Your program must compile, run and produce some sort of output to be
  graded. 0 if not satisfied.
\item 25+ pts.  Your program must have the 2 required named functions,
   that accept the required input parameters and return the required
   values (if any).
\item 25+ pts. Your iterative implementation must use loops/iteration to implement
   its calculation.  The function must of course correctly compute the $n^{th}$
   term of the series.
\item 40+ pts. Your recursive implementation must perform its calculation using
   recursion.  You must have the correct base cases defined.  Your function must
   of course correctly compute the $n^{th}$ term of the series.
   trials, and count up the successful trials from all of the trials performed,
   and return the correct probability ratio.  Your ratio must be correct.
\item 10+ pts. You must prompt the user for $n$ in main, and correctly display
   the returned results form your function as shown.
\end{enumerate}
\subsection*{Program Style}
\label{sec-5-2}


Your programs must conform to the style and formatting guidelines
given for this course.  The following is a list of the guidelines that
are required for the assignment to be submitted this week.

\begin{enumerate}
\item The file header for the file with your name and program information
  and the function header for your main function must be present, and
  filled out correctly.
\item A function header must be present for all functions you define.
  You must document the purpose, input parameters and return values
  of all functions.
\item You must indent your code correctly and have no embedded tabs in
  your source code. (Don't forget about the Visual Studio Format
  Selection command).
\item You must not have any statements that are hacks in order to keep
  your terminal from closing when your program exits.
\item You must have a single space before and after each binary operator.
\item You must have a single blank line after the end of your declaration
  of variables at the top of a function, before the first code
  statement.
\item You must have a single blank space after , and \verb~;~ operators used as a
  separator in lists of variables, parameters or other control
  structures.
\item You must have opening \verb~{~ and closing \verb~}~ for control statement blocks
  on their own line, indented correctly for the level of the control
  statement block.
\end{enumerate}

Failure to conform to any of these formatting and programming practice
guidelines for this assignment will result in at least 1/3 of the
points (33) for the assignment being removed for each guideline that
is not followed (up to 3 before getting a 0 for the
assignment). Failure to follow other class/textbook programming
guidelines may result in a loss of points, especially for those
programming practices given in our Deitel textbook that have been in
our required reading so far.

\end{document}
