% Created 2015-02-13 Fri 12:15
\documentclass[11pt]{article}
\usepackage[utf8]{inputenc}
\usepackage[T1]{fontenc}
\usepackage{fixltx2e}
\usepackage{graphicx}
\usepackage{longtable}
\usepackage{float}
\usepackage{wrapfig}
\usepackage{soul}
\usepackage{textcomp}
\usepackage{marvosym}
\usepackage{wasysym}
\usepackage{latexsym}
\usepackage{amssymb}
\usepackage{hyperref}
\tolerance=1000
\usepackage{minted}
\usepackage{minted}
\usemintedstyle{default}
\providecommand{\alert}[1]{\textbf{#1}}

\title{Assg 05: Binomial Coefficient.}
\author{CSci 515 Spring 2015}
\date{2015-02-11}
\hypersetup{
  pdfkeywords={},
  pdfsubject={Assg 05},
  pdfcreator={Emacs Org-mode version 7.9.3f}}

\begin{document}

\maketitle


\section*{Dates:}
\label{sec-1}


\begin{center}
\begin{tabular}{ll}
 Due:  &  Tuesday February 24, by Midnight  \\
\end{tabular}
\end{center}
\section*{Objectives}
\label{sec-2}

\begin{itemize}
\item More practice writing C functions
\item Learn about implementing algorithms using iteration and recursion.
\item Practice writing a recursive function.
\item Become more familiar with implementing mathematical series on a computer.
\end{itemize}
\section*{Description}
\label{sec-3}

In this assignment we will write a recursive function to calculate
what is known as the binomial coefficient.  The binomial coefficient
is a very useful quantity, it allows us to count the number of ways of
selecting $i$ items out of a set of $n$ elements.  For example, if we
have $3$ items \verb~A, B, C~, there are 3 ways to choose 1 element from
the items: choose A, or choose B or choose C.  There are also 3 ways
to choose 2 elements from the items: AB, AC, BC.  There is only 1 way
to choose 3 elements form a set of 3 items: ABC. When we choose
2 elements from a set of 3 items, we normally speak of this as
3 select 2, and mathematically we write this as a binomial coefficient

$$
{3 \choose 2} = 3
$$

Where the result of the binomial coefficient is to count up the number of
combinations we will have for $n$ items when we select $i$ elements.  As
another example, just to make this clear, if we have a set of 4 items, and
we choose 2 elements, we get: AB, AC, AD, BC, BD, CD = 6:

$$
{4 \choose 2} = 6
$$

Mathematically we can compute directly the number of combinations for
$n$ choose $i$ using factorials:

$$
{n \choose i} = \frac{n!}{i! (n - i)!}
$$

Where $!$ represents the factorial of a number, as we discussed in
class.

However, another way of computing the number of combinations is by
defining a recursive relationship:

$$
{n \choose i} = {n-1 \choose i-1} + {n-1 \choose i}
$$

You can think of this as a recursive function that takes two parameters
$n$ and $i$, and computes the answer recursively by adding together
two smaller subproblems.  For this recursive definition of the
binomial coefficient, the base cases are:

$$
{n \choose 0} = {n \choose n} = 1
$$

We have already seen why $n$ items choose $n$ elements will always
be 1.  The other base case is used by definition, and simply means
that there is only 1 way of choosing no items from a set (e.g. you
don't choose).

In this assignment you will write 3 functions.  You need to first of
all write a function to compute the number of combinations of $n$
items choosing $i$ elements directly using our first formula
(involving factorials).  You should also (re)create the factorial
function (that we developed in class) that you will call from this
function to compute factorials of numbers.  You will also implement a
recursive version of calculating the number of calculations, using the
recursive definition of the binomial coefficient and the base cases
given above.


Perform the following tasks:

\begin{enumerate}
\item Write a function called \verb~factorial()~.  This function should take a
   single integer as its parameter, and it will compute the factorial
   $n!$ of the value and return it as its result.
\item Write a function called \verb~countCombinationsDirectly()~.  This
   function will compute the number of combinations that result from
   choosing $i$ elements from a set of $n$ items.  The function should
   take two integer values as its parameters $n$ and $i$, which will
   be integers >= 0.  The function should directly compute the number
   of combinations, using the equation involving factorials of $n$ and
   $i$.  The function will return a single integer as its result.
\item Write a second function called \verb~countCombinationsRecursive()~.
   This function will also count the number of combinations of
   choosing $i$ elements from a set of $n$ items.  However, you need
   to implement this calculation as a recursive function, using the
   recursive definition and two base cases given above.  Your function
   will take the same two integer parameters $n$ and $i$ with values
   >= 0, and will return an integer result, the count of the number of
   combinations from $n$ choose $i$ items.
\item In your \verb~main()~ function, prompt the user to enter a value for $n$
   the number of items in a set to choose from, and $i$, the number of
   items to select from that set.  You will calculate the number of
   combinations using both of your methods, and display the result.
\end{enumerate}

Your program output should look something close to the following when I
run your program.  I have run the program multiple times, so that you
can also see some examples of correct output for some different values
of $n$ and $i$:


\begin{verbatim}
$ assg05
Enter n (an integer >= 0), and I will calculate the n^th
Fibonacci term for you using two different methods: 0

0 term of the Fibonacci series, using iterative method: 0
0 term of the Fibonacci series, using recursive method: 0

$ assg05
Enter n (an integer >= 0), and I will calculate the n^th
Fibonacci term for you using two different methods: 1

1 term of the Fibonacci series, using iterative method: 1
1 term of the Fibonacci series, using recursive method: 1

$ assg05
Enter n (an integer >= 0), and I will calculate the n^th
Fibonacci term for you using two different methods: 8

8 term of the Fibonacci series, using iterative method: 21
8 term of the Fibonacci series, using recursive method: 21

$ assg05
Enter n (an integer >= 0), and I will calculate the n^th
Fibonacci term for you using two different methods: 10

10 term of the Fibonacci series, using iterative method: 55
10 term of the Fibonacci series, using recursive method: 55

$ assg05
Enter n (an integer >= 0), and I will calculate the n^th
Fibonacci term for you using two different methods: 35

35 term of the Fibonacci series, using iterative method: 9227465
35 term of the Fibonacci series, using recursive method: 9227465
\end{verbatim}


\textbf{NOTE}: Now that our programs have more functions than just the
\verb~main()~ function, the use of the function headers becomes meaningful
and required.  Make sure that all of your functions (\verb~main~,
\verb~nthFibonacciIterative~, \verb~nthFibonacciRecursive~) have function
headers preceding them that document the purpose of the functions, and
the input parameters and return value of the function.
\section*{Assignment Submission}
\label{sec-4}


An eCollege dropbox has been created for this assignment.  You should
upload your version of the assignment to the eCollege dropbox named
\verb~Assg 05 Fibonacci Sequence~ created for this submission.  Work
submitted by the due date will be considered for evaluation.
\section*{Requirements and Grading Rubrics}
\label{sec-5}
\subsection*{Program Execution, Output and Functional Requirements}
\label{sec-5-1}


\begin{enumerate}
\item Your program must compile, run and produce some sort of output to be
  graded. 0 if not satisfied.
\item 25+ pts.  Your program must have the 2 required named functions,
   that accept the required input parameters and return the required
   values (if any).
\item 25+ pts. Your iterative implementation must use loops/iteration to implement
   its calculation.  The function must of course correctly compute the $n^{th}$
   term of the series.
\item 40+ pts. Your recursive implementation must perform its calculation using
   recursion.  You must have the correct base cases defined.  Your function must
   of course correctly compute the $n^{th}$ term of the series.
   trials, and count up the successful trials from all of the trials performed,
   and return the correct probability ratio.  Your ratio must be correct.
\item 10+ pts. You must prompt the user for $n$ in main, and correctly display
   the returned results form your function as shown.
\end{enumerate}
\subsection*{Program Style}
\label{sec-5-2}


Your programs must conform to the style and formatting guidelines
given for this course.  The following is a list of the guidelines that
are required for the assignment to be submitted this week.

\begin{enumerate}
\item The file header for the file with your name and program information
  and the function header for your main function must be present, and
  filled out correctly.
\item A function header must be present for all functions you define.
  You must document the purpose, input parameters and return values
  of all functions.
\item You must indent your code correctly and have no embedded tabs in
  your source code. (Don't forget about the Visual Studio Format
  Selection command).
\item You must not have any statements that are hacks in order to keep
  your terminal from closing when your program exits.
\item You must have a single space before and after each binary operator.
\item You must have a single blank line after the end of your declaration
  of variables at the top of a function, before the first code
  statement.
\item You must have a single blank space after , and \verb~;~ operators used as a
  separator in lists of variables, parameters or other control
  structures.
\item You must have opening \verb~{~ and closing \verb~}~ for control statement blocks
  on their own line, indented correctly for the level of the control
  statement block.
\end{enumerate}

Failure to conform to any of these formatting and programming practice
guidelines for this assignment will result in at least 1/3 of the
points (33) for the assignment being removed for each guideline that
is not followed (up to 3 before getting a 0 for the
assignment). Failure to follow other class/textbook programming
guidelines may result in a loss of points, especially for those
programming practices given in our Deitel textbook that have been in
our required reading so far.

\end{document}
