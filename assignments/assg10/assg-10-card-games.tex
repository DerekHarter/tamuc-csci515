% Created 2015-04-07 Tue 16:15
\documentclass[11pt]{article}
\usepackage[utf8]{inputenc}
\usepackage[T1]{fontenc}
\usepackage{fixltx2e}
\usepackage{graphicx}
\usepackage{longtable}
\usepackage{float}
\usepackage{wrapfig}
\usepackage{soul}
\usepackage{textcomp}
\usepackage{marvosym}
\usepackage{wasysym}
\usepackage{latexsym}
\usepackage{amssymb}
\usepackage{hyperref}
\tolerance=1000
\usepackage{minted}
\usepackage{minted}
\usemintedstyle{default}
\providecommand{\alert}[1]{\textbf{#1}}

\title{Assg 10: Card Games}
\author{CSci 515 Spring 2015}
\date{2015-04-07}
\hypersetup{
  pdfkeywords={},
  pdfsubject={Assg 10: Card Games},
  pdfcreator={Emacs Org-mode version 7.9.3f}}

\begin{document}

\maketitle


\section*{Dates:}
\label{sec-1}


\begin{center}
\begin{tabular}{ll}
 Due:  &  Tuesday April 14, by Midnight  \\
\end{tabular}
\end{center}
\section*{Objectives}
\label{sec-2}

\begin{itemize}
\item Get more practice defining and using structures.
\item Practice using arrays of structs as basic tables.
\item Learn to process arrays of structs.
\end{itemize}
\section*{Description}
\label{sec-3}

In this assignment you will extend your lab 10 work.  If you did not
complete the structure and enums defining your Card data type, or the
function to initialize the deck of cards in the lab, you should finish
those parts from the lab first.

You will add the following functionality to your deck of cards
simulation.  You will add a display function that can be used to
display the cards in an array of cards.  You will also add a shuffle
function, to shuffle up the deck of cards.  Finally you will add
a function that will play a simple game with the deck of cards.

Perform the following tasks:

\begin{enumerate}
\item Create a function called \verb~displayDeckOfCards~.  This function
   should take an array of your Cards, and a begin and end index.  The
   function should display the cards in the indicated index range to
   standard output.  The begin index is inclusive, but the end index
   is not inclusive, so if asked to display cards from begin index 0
   to end index 5, it will display the cards at indexes 0, 1, 2, 3
   and 4.  See the example output below for how the output should be
   displayed.
\item Create a function called \verb~shuffleDeckOfCards~.  This functions
   should take an array of 52 Card structures.  It will expect
   the cards to have already been created/initialized.  This function
   should shuffle up the cards randomly, using the following
   algorithm.  Iterate through all of the cards from index 0 up to
\begin{enumerate}
\item For each index, choose another index at random in the range
\end{enumerate}
from 0 up to 52.  Then swap the Card at the current index of the
   cards array with the Card at the randomly generated index.  For
   example, the first time through the loop the current index will be
\begin{enumerate}
\item Lets say you choose random index 15 to swap with.  Then you
\end{enumerate}
will swap Card 0 with Card 15 (make sure you use a temporary
   variable to perform the swap).  The result of doing this swap with
   a random card for every index in the array should be a well and
   randomly shuffled deck of cards.
\item Create a function called \verb~highestDrawFour~.  This function will
   take an array of (randomly shuffled) cards as its input.  The
   task is to simulate 4 players drawing 4 cards from the desk.  
   Player 0 is given the card at index 0, Player 1 the card at
   index 1, etc.  You determine the winner in this manner.  The
   suit of the card is the most important attribute.  
   HEARTS beats DIAMONDS beats CLUBS beats SPADES.  For example,
   if none of the 4 players has a HEARTS, but only 1 player has
   a DIAMONDS, then the player with DIAMONDS wins.  If 2 or more
   players have equivalent best suits (for example, two players had
   DIAMONDS in the previous example), then you need to fall back
   on the face of the cards.  In this case an ACE beats KING beats
   QUEEN beats JACK etc all the way down to a DEUCE (ACE is highest
   card, followed by the face cards, then the number cards by rank).
   Your function should display a message to standard output about
   which player of the 4 (Player 0, Player 1, Player 2 or Player 3)
   is the winner.
\item In your main function, create a deck of cards and initialize it.
   Then use your shuffle function to shuffle the deck of cards, and
   use your display function to display the 4 top cards from the
   shuffled deck.  Finally call your \verb~highestDrawFour~ function
   to play a game with the top 4 cards, which should cause
   the winning player of the game to be displayed.
\end{enumerate}

Here is an example output from running a correct implementation of
this assignment.


\textbf{HINT}: Finding the highest card in the game described can be a bit
tricky.  Here is my suggestion on how to do it.  Create a function
that simply compares two cards, passed in as a parameter to the
function.  The function should return a \verb~bool~ of True if the first card
wins over the second card according to the rules, and a False if the
second card is the winner.  With this function you can first compare
Player 0 and Player 1 card to see which one wins, the compare the
winner of this came to Player 2, etc.

\textbf{NOTE}: Now that our programs have more functions than just the
\verb~main()~ function, the use of the function headers becomes meaningful
and required.  Make sure that all of your functions have function
headers preceding them that document the purpose of the functions, and
the input parameters and return value of the function.
\section*{Assignment Submission}
\label{sec-4}


An eCollege dropbox has been created for this assignment.  You should
upload your version of the assignment to the eCollege dropbox named
\verb~Assg 10 Card Games~ created for this submission.  Work
submitted by the due date will be considered for evaluation.
\section*{Requirements and Grading Rubrics}
\label{sec-5}
\subsection*{Program Execution, Output and Functional Requirements}
\label{sec-5-1}


\begin{enumerate}
\item Your program must compile, run and produce some sort of output to
   be graded. 0 if not satisfied.
\item 20+ pts. For the correct implementation of the function to display the
   cards as described.
\item 30+ pts. For correctly implementing the deck shuffling function.
\item 40+ pts. For correctly implementing the draw four game as described.
\item 10+ pts. For displaying the output of the game as described.
\end{enumerate}
\subsection*{Program Style}
\label{sec-5-2}


Your programs must conform to the style and formatting guidelines
given for this course.  The following is a list of the guidelines that
are required for the assignment to be submitted this week.

\begin{enumerate}
\item The file header for the file with your name and program information
  and the function header for your main function must be present, and
  filled out correctly.
\item A function header must be present for all functions you define.
   You must document the purpose, input parameters and return values
   of all functions.  Your function headers must be formatted exactly
   as shown in the style guidelines for the class.
\item You must indent your code correctly and have no embedded tabs in
  your source code. (Don't forget about the Visual Studio Format
  Selection command).
\item You must not have any statements that are hacks in order to keep
   your terminal from closing when your program exits (e.g. no calls
   to system() ).
\item You must have a single space before and after each binary operator.
\item You must have a single blank line after the end of your declaration
  of variables at the top of a function, before the first code
  statement.
\item You must have a single blank space after , and \verb~;~ operators used as a
  separator in lists of variables, parameters or other control
  structures.
\item You must have opening \verb~{~ and closing \verb~}~ for control statement blocks
  on their own line, indented correctly for the level of the control
  statement block.
\item All control statement blocks (if, for, while, etc.) must have \verb~{~
   \verb~}~ enclosing them, even when they are not strictly necessary
   (when there is only 1 statement in the block).
\item You should attempt to use meaningful variable and function names in
   your program, for program clarity.  Of course, when required, you
   must name functions, parameters and variables as specified in the
   assignments.  Variable and function names must conform to correct
   \verb~camelCaseNameingConvention~ .
\end{enumerate}

Failure to conform to any of these formatting and programming practice
guidelines for this assignment will result in at least 1/3 of the
points (33) for the assignment being removed for each guideline that
is not followed (up to 3 before getting a 0 for the
assignment). Failure to follow other class/textbook programming
guidelines may result in a loss of points, especially for those
programming practices given in our Deitel textbook that have been in
our required reading so far.

\end{document}
