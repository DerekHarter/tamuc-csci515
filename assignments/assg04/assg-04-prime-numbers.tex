% Created 2015-02-06 Fri 15:08
\documentclass[11pt]{article}
\usepackage[utf8]{inputenc}
\usepackage[T1]{fontenc}
\usepackage{fixltx2e}
\usepackage{graphicx}
\usepackage{longtable}
\usepackage{float}
\usepackage{wrapfig}
\usepackage{soul}
\usepackage{textcomp}
\usepackage{marvosym}
\usepackage{wasysym}
\usepackage{latexsym}
\usepackage{amssymb}
\usepackage{hyperref}
\tolerance=1000
\usepackage{minted}
\usepackage{minted}
\usemintedstyle{default}
\providecommand{\alert}[1]{\textbf{#1}}

\title{Assg 04: Finding Prime Numbers}
\author{CSci 515 Spring 2015}
\date{2015-01-23}
\hypersetup{
  pdfkeywords={},
  pdfsubject={Lab 04},
  pdfcreator={Emacs Org-mode version 7.9.3f}}

\begin{document}

\maketitle


\section*{Dates:}
\label{sec-1}


\begin{center}
\begin{tabular}{ll}
 Due:  &  Tuesday February 17, by Midnight  \\
\end{tabular}
\end{center}
\section*{Objectives}
\label{sec-2}

\begin{itemize}
\item Write functions in C that take 1 or more parameters and return a result.
\item Learn and practice breaking up a larger problem into smaller sub-parts.
\item More practice using control structures for looping and conditional
  execution in C.
\item Practice with mathematical operations and concepts, like the modulus
  operator.
\end{itemize}
\section*{Description}
\label{sec-3}

As you should know, an integer is said to be \emph{prime} if it is
divisible by only 1 and itself.  For example, 2, 3, 5 and 7 are
prime, but 4, 6, 8 and 9 are not.

Your overall task is to write a function, called \verb~findPrimes()~ that
will find all of the prime numbers in a range from M to N.  To
solve this task, you will break down the big problem into two
sub-problems:

\begin{enumerate}
\item Write a function called \verb~isPrimeNumber()~ that determines whether
   some particular number is a prime number or not.
\item Write a function, called \verb~findPrimes()~ that will use the first
   function to search for primes in a range of values.
\end{enumerate}

If you recall we gave an example of a brute force method for determining
if a number is a prime or not previously.  A simple method to
determine if a number X is a prime is to test all possible
divisors from 2 to X/2.  If any number is found that can divide
the number X evenly, then the number can not be a prime.  If
no such divisor is found, then you have proven the number is
a prime.  Recall that you can use the modulus operator `\%'
in C to test if 2 numbers are evenly divisible (with no
remainder).  Your \verb~isPrimeNumber()~ function should take
a single integer value as its input parameter.  The function
will return a \verb~bool~ Boolean value as its result.  Your function
should return \verb~true~ if the number passed into it in its
parameter is a prime, and it should return \verb~false~ otherwise.

The second function you should write will use your first function
to search for prime numbers in a range of numbers.  Your
second function should be called \verb~findPrimes()~.  This function
will take 2 integer parameters as input, the \verb~beginRange~ and
\verb~endRange~ of the range that the function should search
within (inclusive) for prime numbers.  The function
will not return any result, so it will be a \verb~void~ function.
The function will communicate its results by displaying
prime numbers that it finds to standard output.  

Your program should initially prompt the user for the minimum and
maximum values that should be used for the range that is to be
searched for primes.  This prompt and input of values should be done
in your \verb~main()~ function.  The output from running your program
with the described functions should look exactly like this:


\begin{verbatim}
This program finds all of the prime numbers within a range from M to N.
At what value should we begin searching: 2
At what value should we end searching: 100
2 is a prime number
3 is a prime number
5 is a prime number
7 is a prime number
11 is a prime number
13 is a prime number
17 is a prime number
19 is a prime number
23 is a prime number
29 is a prime number
31 is a prime number
37 is a prime number
41 is a prime number
43 is a prime number
47 is a prime number
53 is a prime number
59 is a prime number
61 is a prime number
67 is a prime number
71 is a prime number
73 is a prime number
79 is a prime number
83 is a prime number
89 is a prime number
97 is a prime number
\end{verbatim}

\textbf{NOTE}: Now that our programs have more functions than just the \verb~main()~
function the use of the function headers becomes meaningful.  Make sure
that all of your functions (\verb~main~, \verb~findPrimes~, \verb~isPrimeNumber~)
have function headers before them that document the purpose of the
functions, and the input values and return value of the function. 
\section*{Lab Submission}
\label{sec-4}


An eCollege dropbox has been created for this lab.  You should
upload your version of the lab by the end of lab time to the eCollege
dropbox named \verb~Lab 04 Finding Primes~.  Work submitted by the end
of lab will be considered, but after the lab ends you may no longer
submit work, so make sure you submit your best effort by the lab end
time in order to receive credit.
\section*{Requirements and Grading Rubrics}
\label{sec-5}
\subsection*{Program Execution, Output and Functional Requirements}
\label{sec-5-1}


\begin{enumerate}
\item Your program must compile, run and produce some sort of output to be
  graded. 0 if not satisfied.
\item 40+ pts.  Your program must have the 2 required named functions, that 
   accept the required input parameters and return the required values
   (if any).
\item 30+ pts. Your algorithm for determining if a number is a prime in the
   \verb~isPrimeNumber()~ function must work correctly.
\item 30+ pts. Likewise the \verb~findPrimes()~ function must work, and produce
   the output as shown for the assignment.
\end{enumerate}
\subsection*{Program Style}
\label{sec-5-2}


Your programs must conform to the style and formatting guidelines given for this course.
The following is a list of the guidelines that are required for the lab to be submitted
this week.

\begin{enumerate}
\item The file header for the file with your name and program information
  and the function header for your main function must be present, and
  filled out correctly.
\item A function header must be present for all functions you define.
  You must document the purpose, input parameters and return values
  of all functions.
\item You must indent your code correctly and have no embedded tabs in
  your source code. (Don't forget about the Visual Studio Format
  Selection command).
\item You must not have any statements that are hacks in order to keep
  your terminal from closing when your program exits.
\item You must have a single space before and after each binary operator.
\item You must have a single blank line after the end of your declaration
  of variables at the top of a function, before the first code
  statement.
\item You must have a single blank space after , and \verb~;~ operators used as a
  separator in lists of variables, parameters or other control
  structures.
\item You must have opening \verb~{~ and closing \verb~}~ for control statement blocks
  on their own line, indented correctly for the level of the control
  statement block.
\end{enumerate}

Failure to conform to any of these formatting and programming practice
guidelines for this lab will result in at least 1/3 of the points (33)
for the assignment being removed for each guideline this is not being
met (up to 3 total before getting a 0).  Failure to follow other
class/textbook programming guidelines may result in a loss of points,
especially for those programming practices given in our Deitel
textbook that have been in our required reading so far.

\end{document}
