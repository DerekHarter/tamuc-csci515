% Created 2015-01-23 Fri 13:29
\documentclass[11pt]{article}
\usepackage[utf8]{inputenc}
\usepackage[T1]{fontenc}
\usepackage{fixltx2e}
\usepackage{graphicx}
\usepackage{longtable}
\usepackage{float}
\usepackage{wrapfig}
\usepackage{soul}
\usepackage{textcomp}
\usepackage{marvosym}
\usepackage{wasysym}
\usepackage{latexsym}
\usepackage{amssymb}
\usepackage{hyperref}
\tolerance=1000
\usepackage{minted}
\usepackage{minted}
\usemintedstyle{default}
\providecommand{\alert}[1]{\textbf{#1}}

\title{Lab 02: Find Average of a Group of Numbers}
\author{CSci 515 Spring 2015}
\date{2015-01-23}
\hypersetup{
  pdfkeywords={},
  pdfsubject={Lab 02},
  pdfcreator={Emacs Org-mode version 7.9.3f}}

\begin{document}

\maketitle

\section*{Objective}
\label{sec-1}

Determine the average, and some other properties for an arbitrary
number of floating point values.  The user should be prompted to enter
numbers and to indicate when they are done entering numbers by using a
sentinel (use -9999 as your sentinel value).

Your program should report the following values after the user has
indicated they are finished entering numbers.  The average of the
values, the minimum value entered, the maximum value entered, and the
total number of values the user entered.  Your output should be formatted
to the terminal to look exactly like this:

\+begin$_{\mathrm{example}}$
Number of Values Entered: 5
Maximum Value: 28
Minimum Value: -32
Average Value: 18.5236
\+end$_{\mathrm{example}}$

\end{document}
