% Created 2015-04-01 Wed 09:41
\documentclass[11pt]{article}
\usepackage[utf8]{inputenc}
\usepackage[T1]{fontenc}
\usepackage{fixltx2e}
\usepackage{graphicx}
\usepackage{longtable}
\usepackage{float}
\usepackage{wrapfig}
\usepackage{soul}
\usepackage{textcomp}
\usepackage{marvosym}
\usepackage{wasysym}
\usepackage{latexsym}
\usepackage{amssymb}
\usepackage{hyperref}
\tolerance=1000
\usepackage{minted}
\usepackage{minted}
\usemintedstyle{default}
\providecommand{\alert}[1]{\textbf{#1}}

\title{Lab 09: Improving Bubble Sort}
\author{CSci 515 Spring 2015}
\date{2015-02-18}
\hypersetup{
  pdfkeywords={},
  pdfsubject={Lab 09 Improving Bubble Sort},
  pdfcreator={Emacs Org-mode version 7.9.3f}}

\begin{document}

\maketitle


\section*{Dates:}
\label{sec-1}


\begin{center}
\begin{tabular}{ll}
 Due:  &  In Lab, Wednesday April 1, by 4:05 pm (lab end time)  \\
\end{tabular}
\end{center}
\section*{Objectives}
\label{sec-2}

\begin{itemize}
\item Practice using arrays.
\item Become familiar with details of sorting algorithms.
\item Learn more about analyzing the time complexity of algorithms.
\end{itemize}
\section*{Description}
\label{sec-3}

There are several improvements / tweaks that can be done to the basic
bubble sort sorting algorithm we looked at last week that improve it
somewhat.  Start this lab by copying the final version of the bubble
sort (P03-bubble-sort.cpp) from last weeks lectures.  Then make the
following two improvements to the basic bubble sort.

First of all, the best case behavior of the basic bubble sort is
$O(n^2)$.  Basically, even if the values are already sorted, as
currently implemented, the outer and inner loops will both always run
approximately $n$ times.  However, it is unnecessary to continue
running the outer loop if/when the array of values has become sorted.
An easy way to determine if the values are sorted is to keep track in
the inner loop of weather or not any swaps occur.  In other words in
the inner loop simply keep a flag or count of the number of swaps.  If
it ever happens that no swaps occur, then it must mean that the values
are sorted, and we can stop.  In the best case, after adding this
improvement, if you give bubble sort an already sorted list, it will
simply go through the inner loop 1 time, see the values are sorted,
and then stop.

For your second improvement you will perform something related, but a
little bit trickier.  It can happen that large portions of the end of
the array have become sorted during the bubble sort.  Our current
implementation does take advantage of the fact that on the first pass
the largest item will be bubbled up to the last position, so it only
performs the necessary iterations in the inner loop keeping track of
which items have been sorted.  But, as we did before, we can try and
keep track of the location where the last swap occurs during the inner
loop.  For example, let \verb~values[idx]~ and \verb~values[idx+1]~ be the 
last two values swapped on a pass.  Then surely the values
\verb~values[idx+1]~ to \verb~values[size-1]~ are already in their final
positions.  In the next pass we only need to consider numbers
\verb~values[0]~ to \verb~values[idx-1]~.  If you keep track of the last
swap location, you can use this to be the maximum location to
iterate up to on the next pass.

Perform the following tasks:

\begin{enumerate}
\item Implement the first optimization described.  Start by copying the
   first version of the \verb~bubbleSort()~ function to a new function
   named \verb~bubbleSortV1()~.  In your \verb~bubbleSortV1()~ function, implement
   the optimization that causes the sorting to stop anytime the inner
   loop detects it makes a pass without any swaps occurring.  As a
   hint, and to prepare to do the next task, you might want to try
   counting the number of swaps that occur, and/or keep track of the
   last item swapped.  You would start this count/location out at 0 in
   both cases, and if it is 0 after the inner loop iteration, it means
   no swaps occurred.
\item Implement the second optimization described.  Start by copying the
   clean first version of the \verb~bubbleSort()~ function to a new
   function named \verb~bubbleSortV2()~ (don't use your modification for
   part 1 here, do this optimization from scratch).  Implement the
   idea that you keep track of the location where the last swap
   occurs, and on the next pass in the inner loop you only bubble up
   to this location.  If you check when this location becomes 0, to
   get out of the outer loop you will then also have the equivalent of
   the first optimization you worked on.
\end{enumerate}

In your \verb~main~ function, show an example of calling your versions of
the \verb~bubbleSort~ functions.  To test that your optimizations are
working, demonstrate passing in a small, already sorted array, and
check using the debugger that the function is performing as you
expect.

An example of output you might generate.  In both cases I used static
initialization of an array of 8 integers to create an almost sorted
array.  Then passed this array to the sorting algorithms.  I have some
output statements inside of the \verb~bubbleSortV1()~ and \verb~bubbleSortV2()~
functions, displaying information about the passes, swaps done, etc.

\begin{verbatim}
$ ./lab09
Before calling bubbleSortV1:
000:   3
001:   5
002:  13
003:   9
004:  18
005:  11
006:  10
007:  15

bubbleSortV1() pass:0 number of swaps performed:4
bubbleSortV1() pass:1 number of swaps performed:2
bubbleSortV1() pass:2 number of swaps performed:1
bubbleSortV1() pass:3 number of swaps performed:0

After calling bubbleSortV1:
000:   3
001:   5
002:   9
003:  10
004:  11
005:  13
006:  15
007:  18

Before calling bubbleSortV2:
000:   7
001:   5
002:   3
003:   9
004:  10
005:  11
006:  15
007:  18

bubbleSortV2() pass:0 last swap location:1
bubbleSortV2() pass:1 last swap location:0
bubbleSortV2() pass:2 last swap location:-1

After calling bubbleSortV2:
000:   3
001:   5
002:   7
003:   9
004:  10
005:  11
006:  15
007:  18
\end{verbatim}


\textbf{NOTE}: Now that our programs have more functions than just the
\verb~main()~ function, the use of the function headers becomes meaningful
and required.  Make sure that all of your functions have function
headers preceding them that document the purpose of the functions, and
the input parameters and return value of the function.
\section*{Lab Submission}
\label{sec-4}


An eCollege dropbox has been created for this lab.  You should upload
your version of the lab by the end of lab time to the eCollege dropbox
named \verb~Lab 09 Improving Bubble Sort~.  Work submitted by the end of
lab will be considered, but after the lab ends you may no longer
submit work, so make sure you submit your best effort by the lab end
time in order to receive credit.
\section*{Requirements and Grading Rubrics}
\label{sec-5}
\subsection*{Program Execution, Output and Functional Requirements}
\label{sec-5-1}


\begin{enumerate}
\item Your program must compile, run and produce some sort of output to be
  graded. 0 if not satisfied.
\item 50+ pts.  Your program must have the required 2 named function,
   that accepts the required input parameters and return the required
   values (if any).
\item 50+ pts. The functions must implement the described improvements
   to the bubble sort algorithm and work.
\end{enumerate}
\subsection*{Program Style}
\label{sec-5-2}


Your programs must conform to the style and formatting guidelines given for this course.
The following is a list of the guidelines that are required for the lab to be submitted
this week.

\begin{enumerate}
\item The file header for the file with your name and program information
  and the function header for your main function must be present, and
  filled out correctly.
\item A function header must be present for all functions you define.
   You must document the purpose, input parameters and return values
   of all functions.  Your function headers must be formatted exactly
   as shown in the style guidelines for the class.
\item You must indent your code correctly and have no embedded tabs in
  your source code. (Don't forget about the Visual Studio Format
  Selection command).
\item You must not have any statements that are hacks in order to keep
   your terminal from closing when your program exits (e.g. no calls
   to system() ).
\item You must have a single space before and after each binary operator.
\item You must have a single blank line after the end of your declaration
  of variables at the top of a function, before the first code
  statement.
\item You must have a single blank space after , and \verb~;~ operators used as a
  separator in lists of variables, parameters or other control
  structures.
\item You must have opening \verb~{~ and closing \verb~}~ for control statement blocks
  on their own line, indented correctly for the level of the control
  statement block.
\item All control statement blocks (if, for, while, etc.) must have \verb~{~
   \verb~}~ enclosing them, even when they are not strictly necessary
   (when there is only 1 statement in the block).
\item You should attempt to use meaningful variable and function names in
   your program, for program clarity.  Of course, when required, you
   must name functions, parameters and variables as specified in the
   assignments.  Variable and function names must conform to correct
   \verb~camelCaseNameingConvention~ .
\end{enumerate}

Failure to conform to any of these formatting and programming practice
guidelines for this lab will result in at least 1/3 of the points (33)
for the assignment being removed for each guideline that is not
followed (up to 3 before getting a 0 for the assignment). Failure to
follow other class/textbook programming guidelines may result in a
loss of points, especially for those programming practices given in
our Deitel textbook that have been in our required reading so far.

\end{document}
