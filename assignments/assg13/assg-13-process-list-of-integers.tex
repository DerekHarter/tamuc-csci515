% Created 2015-04-29 Wed 14:23
\documentclass[11pt]{article}
\usepackage[utf8]{inputenc}
\usepackage[T1]{fontenc}
\usepackage{fixltx2e}
\usepackage{graphicx}
\usepackage{longtable}
\usepackage{float}
\usepackage{wrapfig}
\usepackage{soul}
\usepackage{textcomp}
\usepackage{marvosym}
\usepackage{wasysym}
\usepackage{latexsym}
\usepackage{amssymb}
\usepackage{hyperref}
\tolerance=1000
\usepackage{minted}
\usepackage{minted}
\usemintedstyle{default}
\providecommand{\alert}[1]{\textbf{#1}}

\title{Assg 13: Process List of Integers}
\author{CSci 515 Spring 2015}
\date{2015-04-21}
\hypersetup{
  pdfkeywords={},
  pdfsubject={Assg 13: Process List of Integers},
  pdfcreator={Emacs Org-mode version 7.9.3f}}

\begin{document}

\maketitle


\section*{Dates:}
\label{sec-1}


\begin{center}
\begin{tabular}{ll}
 Due:  &  Tuesday May 5, by Midnight  \\
\end{tabular}
\end{center}
\section*{Objectives}
\label{sec-2}

\begin{itemize}
\item More practice processing linked lists
\item More practice with dynamic memory allocation
\item More practice with pointers
\end{itemize}
\section*{Description}
\label{sec-3}

In this lab you will create a couple of functions to process linked lists
of integer values.  You should use your generation function from the lab
in order to create randomly generated linked lists to test these functions.

Perform the following tasks:

\begin{enumerate}
\item Implement a function called \verb~displayList()~.  This function should
   take a list of integers as its input and display the list on
   standard output.  Use this function to demonstrate your
   implementations of the following functions.
\item Create a function called \verb~averageList()~.  This function should
   take a pointer to the head node in a linked list of integers as its
   only parameter.  This function should process all elements in the
   list, to find their average value and return the result.  Use
   floating point arithmetic to sum up and average the values, and
   return a floating point value as the result of this function.  In
   your \verb~main()~ function, create a randomly generated list with 10
   elements, display the list, and demonstrate calling this function
   to find the average of the 10 values in your list.
\item Implement a function called \verb~concatenateLists()~.  This function
   will take two linked lists as its input parameters.  This function
   should cause the second linked list that is passed in to be
   concatenated on to the end of the first linked list.  To do this,
   you need to find the last \verb~Node~ in the first linked list, then set
   its next pointer to point to the head of the second linked list.
   If I were writing this, I might create a small helper function that
   returns the last node in a linked list, called something like
   \verb~findLastNode()~, and use this result to implement the concatenate
   operation.  The result of calling this function is that the first
   list will be modified.  Thus this function does not return a new
   result nor create a new list.  But after calling the function, the
   pointer to the first list will now point to the concatenated result
   of appending the second list.  In your \verb~main()~ function, create
   two lists of integers of size 4 and 3, display the two lists.  Then
   use the concatentate function and show that the resulting lists are
   concatenated again by displaying the first list.
\item Implement a function called \verb~reverseList()~.  This function will
   take a list of integers as input (a singly linked list), and it
   will return a new list.  This function should construct a new list
   that will have all of the items of the original list, but in
   reverse order.  For example, given the list \verb~5 -> 3 -> 2 -> NULL~ 
   this function would return a new list \verb~2 -> 3 -> 5 -> NULL~.  This
   function should not destroy the original list, it needs to dynamically
   create new \verb~Node~ items and create a new list, and copy the
   integer values from the original list nodes to the new list
   nodes.  This function will return the pointer to the head of this
   new list as the result of calling this function.  In your \verb~main~
   function create a list of 8 integers, reverse it, and display
   the original and the reversed list.
\item Demonstrate all of your functions in your \verb~main()~ function.  Make
   sure you adequately test and demonstrate the correct working of all
   of your functions.
\end{enumerate}


Here is an example output from running a correct implementation of
this assignment:


\begin{verbatim}
List of 10 integers:
4 -> 7 -> 18 -> 16 -> 14 -> 16 -> 7 -> 13 -> 10 -> 2 -> EOL
Average of list: 10.7

List 1:
3 -> 8 -> 11 -> 20 -> EOL
List 2:
4 -> 7 -> 1 -> EOL
After concatenating: 
3 -> 8 -> 11 -> 20 -> 4 -> 7 -> 1 -> EOL

List before reverse:
7 -> 13 -> 17 -> 12 -> 9 -> 8 -> 10 -> 3 -> EOL
List after reverse:
3 -> 10 -> 8 -> 9 -> 12 -> 17 -> 13 -> 7 -> EOL
\end{verbatim}

\textbf{NOTE}: Now that our programs have more functions than just the
\verb~main()~ function, the use of the function headers becomes meaningful
and required.  Make sure that all of your functions have function
headers preceding them that document the purpose of the functions, and
the input parameters and return value of the function.
\section*{Assignment Submission}
\label{sec-4}


An eCollege dropbox has been created for this assignment.  You should
upload your version of the assignment to the eCollege dropbox named
\verb~Assg 13 Process List of Integers~ created for this submission.  Work
submitted by the due date will be considered for evaluation.
\section*{Requirements and Grading Rubrics}
\label{sec-5}
\subsection*{Program Execution, Output and Functional Requirements}
\label{sec-5-1}


\begin{enumerate}
\item Your program must compile, run and produce some sort of output to
   be graded. 0 if not satisfied.
\item 10+ pts. For correct implementation of the \verb~displayList()~ function.
\item 20+ pts. For correct implementation of the \verb~averageList()~ function.
\item 30+ pts. For correctly implementing the \verb~concatenateLists()~ function.
\item 30+ pts. For correctly implementing the \verb~reverseList()~ function.
\item 10+ pts. For demonstrating your functions adequately in your \verb~main()~ function.
\end{enumerate}
\subsection*{Program Style}
\label{sec-5-2}


Your programs must conform to the style and formatting guidelines
given for this course.  The following is a list of the guidelines that
are required for the assignment to be submitted this week.

\begin{enumerate}
\item The file header for the file with your name and program information
  and the function header for your main function must be present, and
  filled out correctly.
\item A function header must be present for all functions you define.
   You must document the purpose, input parameters and return values
   of all functions.  Your function headers must be formatted exactly
   as shown in the style guidelines for the class.
\item You must indent your code correctly and have no embedded tabs in
  your source code. (Don't forget about the Visual Studio Format
  Selection command).
\item You must not have any statements that are hacks in order to keep
   your terminal from closing when your program exits (e.g. no calls
   to system() ).
\item You must have a single space before and after each binary operator.
\item You must have a single blank line after the end of your declaration
  of variables at the top of a function, before the first code
  statement.
\item You must have a single blank space after , and \verb~;~ operators used as a
  separator in lists of variables, parameters or other control
  structures.
\item You must have opening \verb~{~ and closing \verb~}~ for control statement blocks
  on their own line, indented correctly for the level of the control
  statement block.
\item All control statement blocks (if, for, while, etc.) must have \verb~{~
   \verb~}~ enclosing them, even when they are not strictly necessary
   (when there is only 1 statement in the block).
\item You should attempt to use meaningful variable and function names in
   your program, for program clarity.  Of course, when required, you
   must name functions, parameters and variables as specified in the
   assignments.  Variable and function names must conform to correct
   \verb~camelCaseNameingConvention~ .
\item Put the \verb~*~ for pointer variable declarations next to the
   type declaration, with no space between the type and the \verb~*~.
   Also please follow the convention of using \verb~Ptr~ at the end of
   names for pointer variables.
\end{enumerate}

Failure to conform to any of these formatting and programming practice
guidelines for this assignment will result in at least 1/3 of the
points (33) for the assignment being removed for each guideline that
is not followed (up to 3 before getting a 0 for the
assignment). Failure to follow other class/textbook programming
guidelines may result in a loss of points, especially for those
programming practices given in our Deitel textbook that have been in
our required reading so far.

\end{document}
