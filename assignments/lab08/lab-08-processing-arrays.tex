% Created 2015-03-25 Wed 09:25
\documentclass[11pt]{article}
\usepackage[utf8]{inputenc}
\usepackage[T1]{fontenc}
\usepackage{fixltx2e}
\usepackage{graphicx}
\usepackage{longtable}
\usepackage{float}
\usepackage{wrapfig}
\usepackage{soul}
\usepackage{textcomp}
\usepackage{marvosym}
\usepackage{wasysym}
\usepackage{latexsym}
\usepackage{amssymb}
\usepackage{hyperref}
\tolerance=1000
\usepackage{minted}
\usepackage{minted}
\usemintedstyle{default}
\providecommand{\alert}[1]{\textbf{#1}}

\title{Lab 08: Restricted Binary Search}
\author{CSci 515 Spring 2015}
\date{2015-02-18}
\hypersetup{
  pdfkeywords={},
  pdfsubject={Lab 08 Restricted Binary Search},
  pdfcreator={Emacs Org-mode version 7.9.3f}}

\begin{document}

\maketitle


\section*{Dates:}
\label{sec-1}


\begin{center}
\begin{tabular}{ll}
 Due:  &  In Lab, Wednesday March 4, by 4:05 pm (lab end time)  \\
\end{tabular}
\end{center}
\section*{Objectives}
\label{sec-2}

\begin{itemize}
\item Learn about declaring, accessing and processing array elements
\item Practice passing arrays to and from functions.
\item See examples of using file I/O to read date into arrays for processing
\item Practice I/O formatting
\item Learn about indexing into arrays in ranges.
\end{itemize}
\section*{Description}
\label{sec-3}

In this lab you will be given a function, and an example of how to
call the function, that will read in an array of values from a
file. You will be processing arrays of floating point values.  You
will write one function to display the elements in an array.  In the
function, you will pass in an array, and a begin and end range.  This
function will display the values in the array, formatted on standard
output as specified below.

Perform the following tasks:

\begin{enumerate}
\item Start with the given template code for this lab (download from
   eCollege), which contains one function already written for you, and
   an example of calling that function.  This function reads in an
   array of floating point values from a file.  As we learned in
   class, since arrays are passed by reference by default, specifying
   the array to be read in to the function, causes the values to be
   filled in and returned to the caller in the given array.
\item Write a function called \verb~displayArrayValues~.  This function will
   take an array of floats as its first parameter, and a \verb~beginRange~
   and \verb~endRange~ as its second and third parameters.  These
   parameters are of type int, and they will be used to specify the
   beginning index and end index of values to be displayed from the
   array.  These values should be >= 0, and the \verb~endRange~ should be
   less than the size of the input array.  If \verb~beginRange~ ==
   \verb~endRange~ then only 1 value will be displayed.  If \verb~beginRange~ >
   \verb~endRange~ then no values will be output. Your function does not
   return anything, so it is a void function.  Instead your function
   displays its output to standard out, and it should be formatted to
   look exactly like the output shown below.  \textbf{HINT}: you may need to
   use the \verb~setw()~, \verb~setprecision()~, \verb~setfill()~ and \verb~fixed~ I/O
   manipulators to get the output formatted exactly as shown.
\item In your main program, you should leave the code alone that reads in
   the array values from the file.  After the array is read in, you
   should first prompt the user for a begin and end range, and call
   the \verb~displayArrayValues~ function to display the values in the
   array in that range.
\end{enumerate}

Your program output should look something close to the following when I
run your program:


\begin{verbatim}
I will display a range of values from our array.
Enter index to start at: 5
Enter index to end at: 15
values[005] 0.52743745
values[006] 0.58105505
values[007] 0.16465282
values[008] 0.28310502
values[009] 0.59936452
values[010] 0.87698799
values[011] 0.34905022
values[012] 0.03936137
values[013] 0.22738582
values[014] 0.59080577
values[015] 0.87810069
\end{verbatim}


\textbf{NOTE}: Now that our programs have more functions than just the
\verb~main()~ function, the use of the function headers becomes meaningful
and required.  Make sure that all of your functions have function
headers preceding them that document the purpose of the functions, and
the input parameters and return value of the function.
\section*{Lab Submission}
\label{sec-4}


An eCollege dropbox has been created for this lab.  You should
upload your version of the lab by the end of lab time to the eCollege
dropbox named \verb~Lab 06 Processing Arrays~.  Work submitted by the end
of lab will be considered, but after the lab ends you may no longer
submit work, so make sure you submit your best effort by the lab end
time in order to receive credit.
\section*{Requirements and Grading Rubrics}
\label{sec-5}
\subsection*{Program Execution, Output and Functional Requirements}
\label{sec-5-1}


\begin{enumerate}
\item Your program must compile, run and produce some sort of output to be
  graded. 0 if not satisfied.
\item 40+ pts.  Your program must have the required named function,
   that accepts the required input parameters and return the required
   values (if any).
\item 20+ pts. Your \verb~displayArrayValues~ function must correctly format
   the displayed output on standard output.  Your program should work
   if the begin and end range are equal, and should show no output
   when begin is greater than the end specified.
\item 20+ pts.  You must use I/O formatting to correctly display the
   output index ranges of the arrays as shown.
\item 20+ pts. Your main function must prompt the user as specified, and
   display the output formatted correctly as shown.
\end{enumerate}
\subsection*{Program Style}
\label{sec-5-2}


Your programs must conform to the style and formatting guidelines given for this course.
The following is a list of the guidelines that are required for the lab to be submitted
this week.

\begin{enumerate}
\item The file header for the file with your name and program information
  and the function header for your main function must be present, and
  filled out correctly.
\item A function header must be present for all functions you define.
   You must document the purpose, input parameters and return values
   of all functions.  Your function headers must be formatted exactly
   as shown in the style guidelines for the class.
\item You must indent your code correctly and have no embedded tabs in
  your source code. (Don't forget about the Visual Studio Format
  Selection command).
\item You must not have any statements that are hacks in order to keep
   your terminal from closing when your program exits (e.g. no calls
   to system() ).
\item You must have a single space before and after each binary operator.
\item You must have a single blank line after the end of your declaration
  of variables at the top of a function, before the first code
  statement.
\item You must have a single blank space after , and \verb~;~ operators used as a
  separator in lists of variables, parameters or other control
  structures.
\item You must have opening \verb~{~ and closing \verb~}~ for control statement blocks
  on their own line, indented correctly for the level of the control
  statement block.
\item All control statement blocks (if, for, while, etc.) must have \verb~{~
   \verb~}~ enclosing them, even when they are not strictly necessary
   (when there is only 1 statement in the block).
\item You should attempt to use meaningful variable and function names in
   your program, for program clarity.  Of course, when required, you
   must name functions, parameters and variables as specified in the
   assignments.  Variable and function names must conform to correct
   \verb~camelCaseNameingConvention~ .
\end{enumerate}

Failure to conform to any of these formatting and programming practice
guidelines for this lab will result in at least 1/3 of the points (33)
for the assignment being removed for each guideline that is not
followed (up to 3 before getting a 0 for the assignment). Failure to
follow other class/textbook programming guidelines may result in a
loss of points, especially for those programming practices given in
our Deitel textbook that have been in our required reading so far.

\end{document}
