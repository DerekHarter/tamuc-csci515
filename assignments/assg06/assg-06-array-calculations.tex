% Created 2015-02-19 Thu 18:44
\documentclass[11pt]{article}
\usepackage[utf8]{inputenc}
\usepackage[T1]{fontenc}
\usepackage{fixltx2e}
\usepackage{graphicx}
\usepackage{longtable}
\usepackage{float}
\usepackage{wrapfig}
\usepackage{rotating}
\usepackage[normalem]{ulem}
\usepackage{amsmath}
\usepackage{textcomp}
\usepackage{marvosym}
\usepackage{wasysym}
\usepackage{amssymb}
\usepackage{hyperref}
\tolerance=1000
\usepackage{minted}
\usepackage{minted}
\usemintedstyle{default}
\author{CSci 515 Spring 2015}
\date{\textit{<2015-02-19 Thu>}}
\title{Assg 06: Calculations on Arrays}
\hypersetup{
  pdfkeywords={},
  pdfsubject={Assg 06},
  pdfcreator={Emacs 24.3.1 (Org mode 8.2.4)}}
\begin{document}

\maketitle

\section*{Dates:}
\label{sec-1}
\begin{center}
\begin{tabular}{ll}
Due: & Tuesday March 3, by Midnight\\
\end{tabular}
\end{center}
\section*{Objectives}
\label{sec-2}
\begin{itemize}
\item More practice writing C functions that take arrays as parameters
\item More work on declaring, indexing and processing arrays.
\item Learn about performing calculations over arrays of values.
\item More examples of breaking problems into subproblems using functions.
\end{itemize}
\section*{Description}
\label{sec-3}
In our lab for this week, you were given a template that reads in an array
of floating point values from a file and saves them in to an array for processing.
For this assignment, you need to start with the same template.

For this assignment, I have provided a different file off 100 floating
point values, named \verb~assg-06-float-array.txt~.  Modify the constants and
code in the original template for lab 06 to read in the 100 values
from this file into your array named \verb~values~.

In this assignment you will be writing 4 function to process the data
in your \verb~values~ array.  Your function will find the minimum value in
an array, the maximum value in an array, the average of the values in
an array, and the standard deviation of the values in an array.  All 4
of these function take the array, and an integer parameter indicating
the size of the array, as input parameters to them.  All 4 of these
function return a floating point result, which is the result of
finding/calculating the minimum / maximum / average / standard
deviation of the values in the array.

Perform the following tasks:

\begin{enumerate}
\item Write a function called \verb~findMinimumValue~.  This function takes
an array of floats as its first parameter, and the size of the
array as its second parameter.  This function should search through
all of the values in the array and return the value that is the
smallest (the minimum) in the array.

\item Write a function called \verb~findMaximumValue~.  This function takes an
array of floats as its first parameter, and the size of the array
as its second parameter.  This function should search through all
of the values in the array and return the value that is the largest
(the maximum) in the array.  This function is basically almost an
exact repeat of the previous function, except for one small change.
In general, repeating code like this is a bad idea, but for this
assignment simply repeat your logic and make the one small change
to find the maximum instead of the minimum.  Later in this course
we will look at some ways we can encapsulate this repeated bit of
functionality into a function or abstraction.

\item Write a function called \verb~findSum~.  This function takes an
array of floats as its first parameter, and the size of the array
as its second parameter.  This function again is almost a complete
repeat of the previous two function, but in this case you need to
add a variable to keep track of and calculate the running \verb~sum~ of
the values in the array.  This function returns a float value,
which should be the sum of all of the values in the given input
array.

\item Write a function called \verb~findAverage~.  This function takes
an array of floats as its first parameter, and the size of the array
as its second parameter.  This function will calculate the average
of the values in the input array, and return this average (a float)
as its result.  The average is calculated by taking the sum of the
values and dividing this by the total number of values.  Thus
your implementation of this function \textbf{MUST} use the \verb~findSumValue~
function to calculate the sum, then compute the average from
(reusing) this functions result.  Thus this function will be quite a
bit different from the previous 3 functions, as you are reusing the
sum function as a black box in order to calculate an average.

\item Write a function called \verb~findStandardDeviation~.  As with all of the
previous functions, this function takes an array of floats as its
first parameter, and the size of the array as its second parameter.
This function will calculate the standard deviation of the values
in the given input array.  The formula to calculate the standard deviation
is as follows:
\end{enumerate}
$$
\sigma = \sqrt{\frac{1}{N} \sum_{i=0}^{N-1} (x_i - \mu)^2}
$$
   Here $\mu$ represents the average of the values in the array, and $x_i$ is
   mathematical notation that basically represents the elements of our array
   indexed from $0$ up to $N-1$.

\begin{enumerate}
\item In your \verb~main()~ function, prompt the user to enter a value for $n$.
You should then call each of your two different implementations, and
display the result of their calculation of the $n^{th}$ term of the
Fibonacci sequence, which if you implement the two different versions
correctly, should always give the same result.  Be careful of using
large values of $n$, as the recursive implementation especially may
take quite a while to calculate in cases of large $n$.
\end{enumerate}

Your program output should look something close to the following when I
run your program.  I have run the program multiple times, so that you
can also see some examples of correct output for some larger values
of $n$:

\begin{verbatim}
$ assg05
Enter n (an integer >= 0), and I will calculate the n^th
Fibonacci term for you using two different methods: 0

0 term of the Fibonacci series, using iterative method: 0
0 term of the Fibonacci series, using recursive method: 0

$ assg05
Enter n (an integer >= 0), and I will calculate the n^th
Fibonacci term for you using two different methods: 1

1 term of the Fibonacci series, using iterative method: 1
1 term of the Fibonacci series, using recursive method: 1

$ assg05
Enter n (an integer >= 0), and I will calculate the n^th
Fibonacci term for you using two different methods: 8

8 term of the Fibonacci series, using iterative method: 21
8 term of the Fibonacci series, using recursive method: 21

$ assg05
Enter n (an integer >= 0), and I will calculate the n^th
Fibonacci term for you using two different methods: 10

10 term of the Fibonacci series, using iterative method: 55
10 term of the Fibonacci series, using recursive method: 55

$ assg05
Enter n (an integer >= 0), and I will calculate the n^th
Fibonacci term for you using two different methods: 35

35 term of the Fibonacci series, using iterative method: 9227465
35 term of the Fibonacci series, using recursive method: 9227465
\end{verbatim}


\textbf{NOTE}: Now that our programs have more functions than just the
\verb~main()~ function, the use of the function headers becomes meaningful
and required.  Make sure that all of your functions (\verb~main~,
\verb~nthFibonacciIterative~, \verb~nthFibonacciRecursive~) have function
headers preceding them that document the purpose of the functions, and
the input parameters and return value of the function.
\section*{Assignment Submission}
\label{sec-4}

An eCollege dropbox has been created for this assignment.  You should
upload your version of the assignment to the eCollege dropbox named
\verb~Assg 05 Fibonacci Sequence~ created for this submission.  Work
submitted by the due date will be considered for evaluation.
\section*{Requirements and Grading Rubrics}
\label{sec-5}

\subsection*{Program Execution, Output and Functional Requirements}
\label{sec-5-1}

\begin{enumerate}
\item Your program must compile, run and produce some sort of output to be
graded. 0 if not satisfied.
\item 25+ pts.  Your program must have the 2 required named functions,
that accept the required input parameters and return the required
values (if any).
\item 25+ pts. Your iterative implementation must use loops/iteration to implement
its calculation.  The function must of course correctly compute the $n^{th}$
term of the series.
\item 40+ pts. Your recursive implementation must perform its calculation using
recursion.  You must have the correct base cases defined.  Your function must
of course correctly compute the $n^{th}$ term of the series.
trials, and count up the successful trials from all of the trials performed,
and return the correct probability ratio.  Your ratio must be correct.
\item 10+ pts. You must prompt the user for $n$ in main, and correctly display
the returned results form your function as shown.
\end{enumerate}

\subsection*{Program Style}
\label{sec-5-2}

Your programs must conform to the style and formatting guidelines
given for this course.  The following is a list of the guidelines that
are required for the assignment to be submitted this week.

\begin{enumerate}
\item The file header for the file with your name and program information
and the function header for your main function must be present, and
filled out correctly.
\item A function header must be present for all functions you define.
You must document the purpose, input parameters and return values
of all functions.
\item You must indent your code correctly and have no embedded tabs in
your source code. (Don't forget about the Visual Studio Format
Selection command).
\item You must not have any statements that are hacks in order to keep
your terminal from closing when your program exits.
\item You must have a single space before and after each binary operator.
\item You must have a single blank line after the end of your declaration
of variables at the top of a function, before the first code
statement.
\item You must have a single blank space after , and \verb~;~ operators used as a
separator in lists of variables, parameters or other control
structures.
\item You must have opening \verb~{~ and closing \verb~}~ for control statement blocks
on their own line, indented correctly for the level of the control
statement block.
\end{enumerate}

Failure to conform to any of these formatting and programming practice
guidelines for this assignment will result in at least 1/3 of the
points (33) for the assignment being removed for each guideline that
is not followed (up to 3 before getting a 0 for the
assignment). Failure to follow other class/textbook programming
guidelines may result in a loss of points, especially for those
programming practices given in our Deitel textbook that have been in
our required reading so far.
% Emacs 24.3.1 (Org mode 8.2.4)
\end{document}
