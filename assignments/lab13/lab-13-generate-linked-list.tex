% Created 2015-04-29 Wed 09:08
\documentclass[11pt]{article}
\usepackage[utf8]{inputenc}
\usepackage[T1]{fontenc}
\usepackage{fixltx2e}
\usepackage{graphicx}
\usepackage{longtable}
\usepackage{float}
\usepackage{wrapfig}
\usepackage{soul}
\usepackage{textcomp}
\usepackage{marvosym}
\usepackage{wasysym}
\usepackage{latexsym}
\usepackage{amssymb}
\usepackage{hyperref}
\tolerance=1000
\usepackage{minted}
\usepackage{minted}
\usemintedstyle{default}
\providecommand{\alert}[1]{\textbf{#1}}

\title{Lab 13: Generate Linked List}
\author{CSci 515 Spring 2015}
\date{2015-04-21}
\hypersetup{
  pdfkeywords={},
  pdfsubject={Lab 13 Generate Linked List},
  pdfcreator={Emacs Org-mode version 7.9.3f}}

\begin{document}

\maketitle


\section*{Dates:}
\label{sec-1}


\begin{center}
\begin{tabular}{ll}
 Due:  &  In Lab, Wednesday April 22, by 4:05 pm (lab end time)  \\
\end{tabular}
\end{center}
\section*{Objectives}
\label{sec-2}

\begin{itemize}
\item Use dynamic memory.
\item Manipulate linked lists
\item Practice using pointers
\item Practice processing linked list data structures and following pointers.
\end{itemize}
\section*{Description}
\label{sec-3}

Using the linked list data structure we developed in class today write
the following function.  Write a function that creates
a linked list of randomly generated integers.  This function
will take a single parameter \verb~numNodes~ and generate that number of nodes
in the linked list.  Then you will display the values in the linked
list.

Perform the following tasks:

\begin{enumerate}
\item Create a function called \verb~generateRandomList()~.  This function
   should take a single parameter \verb~numNodes~ as its input.  It
   will generate a linked list with that number of nodes in it.  The
   \verb~Node~ of the linked list will hold a single integer value.  This
   function should generate a random integer in the range from 1 to
   20 and initialize each of the nodes it creates with this random
   value.  This function should return a \verb~Node*~ pointer as its
   result, which will be a pointer to the first node in the
   generated linked list.  The nodes created by the linked list should
   be generated using dynamic memory allocation.
\item In your main function, display the returned values from generating
   a list with 15 nodes in it.  You should use a while loop to
   follow the node links, and process and display the value of each
   \verb~Node~ item in the list to standard output.  See the example 
   output for how your results should be formatted.
\end{enumerate}

Example output is shown below: 



\textbf{NOTE}: Now that our programs have more functions than just the
\verb~main()~ function, the use of the function headers becomes meaningful
and required.  Make sure that all of your functions have function
headers preceding them that document the purpose of the functions, and
the input parameters and return value of the function.
\section*{Lab Submission}
\label{sec-4}


An eCollege dropbox has been created for this lab.  You should upload
your version of the lab by the end of lab time to the eCollege dropbox
named \verb~Lab 13 Genrate Linked List~.  Work submitted by the end of
lab will be considered, but after the lab ends you may no longer
submit work, so make sure you submit your best effort by the lab end
time in order to receive credit.
\section*{Requirements and Grading Rubrics}
\label{sec-5}
\subsection*{Program Execution, Output and Functional Requirements}
\label{sec-5-1}


\begin{enumerate}
\item Your program must compile, run and produce some sort of output to be
  graded. 0 if not satisfied.
\item 40+ Your \verb~generateRandomList()~ function must work correctly and take the
   specified input parameters..
\item 30+ pts.  Your funciton must use dynamic memory allocation to create the
   correct number of nodes, and set up and return the linked list correctly.
\item 30+ pts. You main function demonstrates correctly invoking the functions
   and displays the required output from doing so.
\end{enumerate}
\subsection*{Program Style}
\label{sec-5-2}


Your programs must conform to the style and formatting guidelines given for this course.
The following is a list of the guidelines that are required for the lab to be submitted
this week.

\begin{enumerate}
\item The file header for the file with your name and program information
  and the function header for your main function must be present, and
  filled out correctly.
\item A function header must be present for all functions you define.
   You must document the purpose, input parameters and return values
   of all functions.  Your function headers must be formatted exactly
   as shown in the style guidelines for the class.
\item You must indent your code correctly and have no embedded tabs in
  your source code. (Don't forget about the Visual Studio Format
  Selection command).
\item You must not have any statements that are hacks in order to keep
   your terminal from closing when your program exits (e.g. no calls
   to system() ).
\item You must have a single space before and after each binary operator.
\item You must have a single blank line after the end of your declaration
  of variables at the top of a function, before the first code
  statement.
\item You must have a single blank space after , and \verb~;~ operators used as a
  separator in lists of variables, parameters or other control
  structures.
\item You must have opening \verb~{~ and closing \verb~}~ for control statement blocks
  on their own line, indented correctly for the level of the control
  statement block.
\item All control statement blocks (if, for, while, etc.) must have \verb~{~
   \verb~}~ enclosing them, even when they are not strictly necessary
   (when there is only 1 statement in the block).
\item You should attempt to use meaningful variable and function names in
   your program, for program clarity.  Of course, when required, you
   must name functions, parameters and variables as specified in the
   assignments.  Variable and function names must conform to correct
   \verb~camelCaseNameingConvention~ .
\item Put the \verb~*~ for pointer variable declarations next to the
   type declaration, with no space between the type and the \verb~*~.
   Also please follow the convention of using \verb~Ptr~ at the end of
   names for pointer variables.
\end{enumerate}

Failure to conform to any of these formatting and programming practice
guidelines for this lab will result in at least 1/3 of the points (33)
for the assignment being removed for each guideline that is not
followed (up to 3 before getting a 0 for the assignment). Failure to
follow other class/textbook programming guidelines may result in a
loss of points, especially for those programming practices given in
our Deitel textbook that have been in our required reading so far.

\end{document}
