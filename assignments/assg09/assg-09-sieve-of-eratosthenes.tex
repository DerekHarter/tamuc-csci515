% Created 2015-04-01 Wed 00:09
\documentclass[11pt]{article}
\usepackage[utf8]{inputenc}
\usepackage[T1]{fontenc}
\usepackage{fixltx2e}
\usepackage{graphicx}
\usepackage{longtable}
\usepackage{float}
\usepackage{wrapfig}
\usepackage{rotating}
\usepackage[normalem]{ulem}
\usepackage{amsmath}
\usepackage{textcomp}
\usepackage{marvosym}
\usepackage{wasysym}
\usepackage{amssymb}
\usepackage{hyperref}
\tolerance=1000
\usepackage{minted}
\usepackage{minted}
\usemintedstyle{default}
\author{CSci 515 Spring 2015}
\date{\textit{<2015-02-19 Thu>}}
\title{Assg 09: The Sieve of Eratosthenes}
\hypersetup{
  pdfkeywords={},
  pdfsubject={Assg 09},
  pdfcreator={Emacs 24.3.1 (Org mode 8.2.4)}}
\begin{document}

\maketitle

\section*{Dates:}
\label{sec-1}
\begin{center}
\begin{tabular}{ll}
Due: & Tuesday April 7, by Midnight\\
\end{tabular}
\end{center}
\section*{Objectives}
\label{sec-2}
\begin{itemize}
\item Practice using advanced array logic.
\item More practice with passing arrays to functions.
\end{itemize}
\section*{Description}
\label{sec-3}
We have seen a brute force algorithm or two for testing whether or not
a given integer is a prime or not.  Recall that a prime number is any
integer this is only evenly divisible by itself and 1.  Also recall
that it is handy to use the C language modulus operator \verb~%~ in order
to test whether one number is divisible by another (the modulus
operator returns the remainder of doing the division, and if the
remainder is 0 then the numbers were evenly divisible).

The Sieve of Eratosthenes is a method of finding prime numbers that
is much more efficient than the brute force method we have seen.
The algorithm operates as follows:

\begin{enumerate}
\item Create an array of boolean variables with all elements initialized
to true.  Array elements with prime subscripts will remain true
after the algorithm finishes.  All other array elements will
eventually be set to false.  We will ignore the elements at index 0
and 1 in this exercise.
\item Starting with array subscript 2, every time an array element is
found whose value is true, loop through the remainder of the array
and set to false every element whose subscript is a multiple of the
subscript for the element with a true value.  For array subscript
2, all elements beyond 2 in the array that are multiples of 2 will
be set to false (subscripts 4, 6, 8, 10, etc.); for array subscript 
3, all elements beyond 3 in the array that are multiples of 3 will be
set to false (subscripts 6, 9, 12, 15, etc.); and so on.  After
array subscript 3, when you check array subscript 4 you will find it
is false (since it was set to false by the multiples of 2 pass), thus
you won't set multiples of 4, etc.
\end{enumerate}

When this process is complete, the array elements that are still set
to true indicate that the subscript is a prime number.  This algorithm
is much more efficient at finding all prime numbers in some range from
2 up to $N$ than checking each individual integer using a brute force method
as we have done before.  In fact, it is close to $O(N)$ to check for primes
in the range $2 ... N$.

Perform the following tasks:

\begin{enumerate}
\item Create a function called \verb~initValuesToTrue()~.  This function should
take an array of booleans, and the size of the array as its input
parameters.  It should initialize all of the values in the boolean 
array to \verb~true~ as a result of calling the function.
\item Create a function called \verb~setMultiplesToFalse()~.  This function
performs the part in step 2 of the algorithm above.  This function
should take an array of boolean values, and the size of the array.
It should also take a third parameter, and integer index or
multiple.  As we described above, for example, if the multiple is
2, it shouldn't set the value at index 2 to \verb~false~ in the array,
but it should set all multiples in of 2 in the array to \verb~false~,
e.g. the values at index 4, 6, 8, \ldots{}  up to the size of the
boolean array.
\item Create a function called \verb~sieveOfEratosthenes()~.  This function
will implement the above described algorithm.  It should take
an array of booleans, and the size of the array as its only 2 parameters.
The function should first call the \verb~initValuesToTrue()~ function, to
ensure that it starts out with all \verb~true~ values in the array in
order to perform step 1.  It should then iterate over all the
indexes in the array, from 2 up to the size of the array, and as
described, any index that it finds still marked \verb~true~, it should
use the \verb~setMultiplesToFalse()~ function to mark all subsequent
multiples of that index to be \verb~false~.
\item Create a function called \verb~displayPrimeNumbers()~.  This function
will again take an array of boolean values, and the size of the
array.  It is assumed this is the resulting array obtained after
calling your Sieve of Eratosthenes function.  Display all of
the values that were determined to be primes from 2 to the size
to standard output.  An example of the desired output is shown below.
\item In your main function, create a boolean array of size 10,000.  Call your
\verb~sieveOfEratosthenes()~ function to find the primes in the first 10,000
integers, and display the results using your \verb~displayPrimeNumbers()~
function.
\end{enumerate}


\textbf{NOTE}: Now that our programs have more functions than just the
\verb~main()~ function, the use of the function headers becomes meaningful
and required.  Make sure that all of your functions have function
headers preceding them that document the purpose of the functions, and
the input parameters and return value of the function.
\section*{Assignment Submission}
\label{sec-4}

An eCollege dropbox has been created for this assignment.  You should
upload your version of the assignment to the eCollege dropbox named
\verb~Assg 09 Sieve of Eratosthenes~ created for this submission.  Work
submitted by the due date will be considered for evaluation.
\section*{Requirements and Grading Rubrics}
\label{sec-5}

\subsection*{Program Execution, Output and Functional Requirements}
\label{sec-5-1}

\begin{enumerate}
\item Your program must compile, run and produce some sort of output to
be graded. 0 if not satisfied.
\item 60+ pts. For the correct implementation of the three helper functions.
\item 30+ pts. For implementing the Sieve of Eratosthenes algorithm and functions
correctly, and using the functions as described above.
\item 10+ pts. Your main function should create the array of boolean values
and demonstrate using your functions to find and display the primes in
the first 10,000 integers.
\end{enumerate}

\subsection*{Program Style}
\label{sec-5-2}

Your programs must conform to the style and formatting guidelines
given for this course.  The following is a list of the guidelines that
are required for the assignment to be submitted this week.

\begin{enumerate}
\item The file header for the file with your name and program information
and the function header for your main function must be present, and
filled out correctly.
\item A function header must be present for all functions you define.
You must document the purpose, input parameters and return values
of all functions.  Your function headers must be formatted exactly
as shown in the style guidelines for the class.
\item You must indent your code correctly and have no embedded tabs in
your source code. (Don't forget about the Visual Studio Format
Selection command).
\item You must not have any statements that are hacks in order to keep
your terminal from closing when your program exits (e.g. no calls
to system() ).
\item You must have a single space before and after each binary operator.
\item You must have a single blank line after the end of your declaration
of variables at the top of a function, before the first code
statement.
\item You must have a single blank space after , and \verb~;~ operators used as a
separator in lists of variables, parameters or other control
structures.
\item You must have opening \verb~{~ and closing \verb~}~ for control statement blocks
on their own line, indented correctly for the level of the control
statement block.
\item All control statement blocks (if, for, while, etc.) must have \verb~{~
\verb~}~ enclosing them, even when they are not strictly necessary
(when there is only 1 statement in the block).
\item You should attempt to use meaningful variable and function names in
your program, for program clarity.  Of course, when required, you
must name functions, parameters and variables as specified in the
assignments.  Variable and function names must conform to correct
\verb~camelCaseNameingConvention~ .
\end{enumerate}

Failure to conform to any of these formatting and programming practice
guidelines for this assignment will result in at least 1/3 of the
points (33) for the assignment being removed for each guideline that
is not followed (up to 3 before getting a 0 for the
assignment). Failure to follow other class/textbook programming
guidelines may result in a loss of points, especially for those
programming practices given in our Deitel textbook that have been in
our required reading so far.
% Emacs 24.3.1 (Org mode 8.2.4)
\end{document}
