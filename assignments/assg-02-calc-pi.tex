% Created 2015-01-23 Fri 22:25
\documentclass[11pt]{article}
\usepackage[utf8]{inputenc}
\usepackage[T1]{fontenc}
\usepackage{fixltx2e}
\usepackage{graphicx}
\usepackage{longtable}
\usepackage{float}
\usepackage{wrapfig}
\usepackage{soul}
\usepackage{textcomp}
\usepackage{marvosym}
\usepackage{wasysym}
\usepackage{latexsym}
\usepackage{amssymb}
\usepackage{hyperref}
\tolerance=1000
\usepackage{minted}
\usepackage{minted}
\usemintedstyle{default}
\providecommand{\alert}[1]{\textbf{#1}}

\title{Assg 02: Calculating PI}
\author{CSci 515 Spring 2015}
\date{2015-01-23}
\hypersetup{
  pdfkeywords={},
  pdfsubject={Assg 02},
  pdfcreator={Emacs Org-mode version 7.9.3f}}

\begin{document}

\maketitle


\section*{Dates:}
\label{sec-1}


\begin{center}
\begin{tabular}{ll}
 Due:  &  Tuesday February 3, by Midnight  \\
\end{tabular}
\end{center}
\section*{Objectives}
\label{sec-2}

\begin{itemize}
\item Practice writing index controlled loops
\item Become more comfortable with using digital computers for calculating mathematical expressions in C.
\item Practice with arithmetic and type conversion in C.
\item More practice with output formatting.
\item Practice using real valued variables for mathematical calculations.
\item Gain experience in translating formula into algorithmic procedures.
\end{itemize}
\section*{Description}
\label{sec-3}

Calculate the value of $\pi$ from the finite series:

$$ \pi = 4 - \frac{4}{3} + \frac{4}{5} - \frac{4}{7} + \frac{4}{9} - \frac{4}{11} ... $$

Print a table that shows the approximate value of $\pi$ after each of
the first N terms of this series.  Your program should prompt the user
for the value of N to determine how many values of the table of
approximate values of $\pi$ will be displayed.  A session of using
your program from the terminal should have exactly the following
output:
\section*{Lab Submission}
\label{sec-4}


An eCollege dropbox has been created for this assignment.  You should
upload your version of the lab by the end of Tuesday 2/3 (midnight) to
the dropbox named \verb~Assg 02 Calculating PI~. 
\section*{Requirements}
\label{sec-5}

Your programs must conform to the style and formatting guidelines given for this course.
The following is a list of the guidelines that are required for the lab to be submitted
this week.

\begin{itemize}
\item The file header and function header for your main function must be present, and filled out correctly.
\item You must indent your code correctly and have no embedded tabs in your source code.
\item You must not have any statements that are hacks in order to keep your terminal from closing when your program exits.
\item You must have a single space before and after each binary operator.
\item You must have a single blank line after the end of your declaration
  of variables at the top of a function, before the first code
  statement.
\end{itemize}

Failure to conform to any of these formatting and programming practice
guidelines for this lab will result in a grade of 0 for the lab, and
your program being returned with an indication of which of these items
your program violates.  Failure to follow other class/textbook
programming guidelines may result in a loss of points, especially for
those good programming practices given in chapters 1-5 of our textbook
which you should have read by now.

\end{document}
