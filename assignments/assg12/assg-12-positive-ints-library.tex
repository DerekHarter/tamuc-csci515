% Created 2015-04-22 Wed 09:15
\documentclass[11pt]{article}
\usepackage[utf8]{inputenc}
\usepackage[T1]{fontenc}
\usepackage{fixltx2e}
\usepackage{graphicx}
\usepackage{longtable}
\usepackage{float}
\usepackage{wrapfig}
\usepackage{soul}
\usepackage{textcomp}
\usepackage{marvosym}
\usepackage{wasysym}
\usepackage{latexsym}
\usepackage{amssymb}
\usepackage{hyperref}
\tolerance=1000
\usepackage{minted}
\usepackage{minted}
\usemintedstyle{default}
\providecommand{\alert}[1]{\textbf{#1}}

\title{Assg 12: Positive Ints Library}
\author{CSci 515 Spring 2015}
\date{2015-04-21}
\hypersetup{
  pdfkeywords={},
  pdfsubject={Assg 12: Positive Ints Library},
  pdfcreator={Emacs Org-mode version 7.9.3f}}

\begin{document}

\maketitle


\section*{Dates:}
\label{sec-1}


\begin{center}
\begin{tabular}{ll}
 Due:  &  Tuesday April 28, by Midnight  \\
\end{tabular}
\end{center}
\section*{Objectives}
\label{sec-2}

\begin{itemize}
\item Understand the relationship between arrays and pointers in C
\item Use pointers to an array of items, and process them.
\item Get more practice with more advanced sentinel based processing.
\end{itemize}
\section*{Description}
\label{sec-3}

In this assignment you will extend the basic functionality from your
lab 12 work.  In this assignment, you will be implementing further
functions that can be used to process pointers to arrays of
positive integers that are terminated with a \verb~-1~ sentinel
character.

Perform the following tasks:

\begin{enumerate}
\item Expand the functionality of your \verb~posintcmp()~ function to be able
   to handle arrays of positive integers of differing lengths.  For
   the \verb~strlen()~ function, if the string are equal up to position \verb~n~,
   but one string is longer than the other, the longer string should
   be considered as occurring after shorter one.  For example if
   \verb~int* p1Ptr = {4, 3, -1}~ and \verb~int* p2Ptr = {4, 3, 2, -1}~
   then \verb~posIntCmp(p1Ptr, p2Ptr)~ should return a \verb~-1~, and
   \verb~posIntCmp(p2Ptr, p1Ptr)~ should return a \verb~1~.
\item Implement a \verb~posintcpy()~ function.  This function takes a
   destination as its first parameter and a source as its second
   parameter (both of type pointer to an array of positive integers).
   This function is a void function, it does not return any explicit
   result.  This function assumes that there is enough space allocated
   for the destination so that all of the values from the source list
   can be copied successfully.  This function should cause all of the
   values from the source array of positive integers to be copied to
   the destination array space (including the \verb~-1~ terminating
   sentinel).
\item Implement a \verb~posintfind()~.  This function is similar to the
   \verb~strchr~ function.  This function takes a pointer to an array of
   positive integers as its first parameter, and a single integer as
   its second parameter (the value to be searched for).  This function
   should perform a linear search of the array of positive integers,
   and return the index of the first location where the value being
   searched for is found.  This function will return a \verb~-1~ if the
   value being searched for is not found in the array of positive
   integers.
\item Create a function called \verb~posintcat()~.  This function is similar
   to the \verb~strcat()~ function.  This function takes two pointers to
   arrays of positive integers null terminated with a \verb~-1~ sentinel
   character.  This is a void function.  The first parameters works as
   a source, and the second as the destination.  However, unlike the
   copy function, this function should concatenate the source array
   values on to the end of the destination array.  You can assume that
   the destination array has enough extra allocated space in order to
   hold the values that will be appended to the end.  For example, if
   \verb~int dest[5] = {1, 2, -1}~ and \verb~int src[] = {3, 4, -1}~, then the
   result of calling \verb~posintcat(src, dest)~ will be that \verb~dest~
   now has the values \verb~{1, 2, 3, 4, -1}~.  \textbf{Extra Credit:} much of
   the functionality of this function has already been accomplished
   in other functions.  Reuse functions like \verb~posintlen~ and
   \verb~posintcpy~ to implement this function.
\item Demonstrate all of your functions in your \verb~main()~ function.  Make
   sure you adequately test and demonstrate the correct working of all
   of your functions.
\end{enumerate}


Here is an example output from running a correct implementation of
this assignment:


\begin{verbatim}
List 1: { 4, 2, 7 }
List 2: { 4, 2, 7, 3 }
Result of posintcmp(list1, list2): -1
Result of posintcmp(list2, list1): 1

List 3 (an empty list with enoough room to hold 10 values): {  }
After copy of list2 to list3, list3 = { 4, 2, 7, 3 }

Result of searching for 4 in List 2: 0
Result of searching for 7 in List 2: 2
Result of searching for 8 in List 2: -1

List 4 has enough room for 10 values, but 
   currently contains only 4: { 9, 10, 15, 22 }
Result of concatenating List 2 onto end of 
   List 4: { 9, 10, 15, 22, 4, 2, 7, 3 }
\end{verbatim}

\textbf{NOTE}: Now that our programs have more functions than just the
\verb~main()~ function, the use of the function headers becomes meaningful
and required.  Make sure that all of your functions have function
headers preceding them that document the purpose of the functions, and
the input parameters and return value of the function.
\section*{Assignment Submission}
\label{sec-4}


An eCollege dropbox has been created for this assignment.  You should
upload your version of the assignment to the eCollege dropbox named
\verb~Assg 12 Positive Ints Library~ created for this submission.  Work
submitted by the due date will be considered for evaluation.
\section*{Requirements and Grading Rubrics}
\label{sec-5}
\subsection*{Program Execution, Output and Functional Requirements}
\label{sec-5-1}


\begin{enumerate}
\item Your program must compile, run and produce some sort of output to
   be graded. 0 if not satisfied.
\item 20+ pts. For the correct implementation of new functionality in \verb~postintcmp()~
   function.
\item 20+ pts. For the correct implementation of the \verb~postintcpy()~ function.
\item 25+ pts. For correctly implementing the \verb~postintfind()~ function.
\item 25+ pts. For correctly implementing the \verb~postintcat()~ function.
\item 10+ pts. For demonstrating your functions adequately in your \verb~main()~ function.
\end{enumerate}
\subsection*{Program Style}
\label{sec-5-2}


Your programs must conform to the style and formatting guidelines
given for this course.  The following is a list of the guidelines that
are required for the assignment to be submitted this week.

\begin{enumerate}
\item The file header for the file with your name and program information
  and the function header for your main function must be present, and
  filled out correctly.
\item A function header must be present for all functions you define.
   You must document the purpose, input parameters and return values
   of all functions.  Your function headers must be formatted exactly
   as shown in the style guidelines for the class.
\item You must indent your code correctly and have no embedded tabs in
  your source code. (Don't forget about the Visual Studio Format
  Selection command).
\item You must not have any statements that are hacks in order to keep
   your terminal from closing when your program exits (e.g. no calls
   to system() ).
\item You must have a single space before and after each binary operator.
\item You must have a single blank line after the end of your declaration
  of variables at the top of a function, before the first code
  statement.
\item You must have a single blank space after , and \verb~;~ operators used as a
  separator in lists of variables, parameters or other control
  structures.
\item You must have opening \verb~{~ and closing \verb~}~ for control statement blocks
  on their own line, indented correctly for the level of the control
  statement block.
\item All control statement blocks (if, for, while, etc.) must have \verb~{~
   \verb~}~ enclosing them, even when they are not strictly necessary
   (when there is only 1 statement in the block).
\item You should attempt to use meaningful variable and function names in
   your program, for program clarity.  Of course, when required, you
   must name functions, parameters and variables as specified in the
   assignments.  Variable and function names must conform to correct
   \verb~camelCaseNameingConvention~ .
\item Put the \verb~*~ for pointer variable declarations next to the
   type declaration, with no space between the type and the \verb~*~.
   Also please follow the convention of using \verb~Ptr~ at the end of
   names for pointer variables.
\end{enumerate}

Failure to conform to any of these formatting and programming practice
guidelines for this assignment will result in at least 1/3 of the
points (33) for the assignment being removed for each guideline that
is not followed (up to 3 before getting a 0 for the
assignment). Failure to follow other class/textbook programming
guidelines may result in a loss of points, especially for those
programming practices given in our Deitel textbook that have been in
our required reading so far.

\end{document}
