% Created 2015-01-20 Tue 20:15
\documentclass[11pt]{article}
\usepackage[utf8]{inputenc}
\usepackage[T1]{fontenc}
\usepackage{fixltx2e}
\usepackage{graphicx}
\usepackage{longtable}
\usepackage{float}
\usepackage{wrapfig}
\usepackage{rotating}
\usepackage[normalem]{ulem}
\usepackage{amsmath}
\usepackage{textcomp}
\usepackage{marvosym}
\usepackage{wasysym}
\usepackage{amssymb}
\usepackage{hyperref}
\tolerance=1000
\usepackage{minted}
\usepackage{array}
\usepackage{color}
\author{Course Syllabus}
\date{Spring 2015}
\title{CSci 515 Fundamentals of Programming}
\hypersetup{
  pdfkeywords={},
  pdfsubject={Tentative Course Syllabus},
  pdfcreator={Emacs 24.3.1 (Org mode 8.2.4)}}
\begin{document}

\maketitle

\section*{Instructor}
\label{sec-1}

\begin{center}
\begin{tabular}{l}
Derek Harter, Ph.D., Associate Professor\\
Department of Computer Science\\
Texas A\&M University - Commerce\\
Office: SCI 355\\
E-mail: Derek.Harter@tamuc.edu\\
\end{tabular}
\end{center}
\section*{Class Meetings}
\label{sec-2}

This course will be web enhanced.  Lectures, notes and course
materials will be distributed through our University's eCollege online
course system.

\begin{center}
\begin{tabular}{lll}
Section 03E (Lecture) & W 11:00a - 1:40p & Location: EDS 133\\
Section 03L (Lab) & W  2:00p - 4:00p & Location Jour 102\\
\end{tabular}
\end{center}
\section*{Course Description}
\label{sec-3}
This is an advanced programming course using a high level programming
language, C and C++. Specific objectives are to introduce the
development of algorithms as a disciplined approach to problem
solving; to present programming practices in design, coding,
debugging, testing and documentation of computer programs; to provide
the student with the fundamental knowledge necessary for further study
in the field of computational sciences.
\section*{Student Learning Outcomes:}
\label{sec-4}

\begin{itemize}
\item To understand the internal representation of the various data types.
\item To review the language syntax and learn new syntax you have not previously used in programming applications.
\item To correctly solve programming problems and learn how to develop algorithms.
\item To examine the internal representation of two and three dimension arrays in C/C++.
\item To understand dynamic memory allocation, parameter passing, the use of pointers.
\end{itemize}
\section*{\uline{Mandatory} Textbook}
\label{sec-5}
\emph{C++ How to Program (7th Edition)}, by Paul J. Deitel \& Harvey M. Deitel, ISBN-13: 978 0136117261


\section*{Evaluation (Tentative)}
\label{sec-6}
Your grade for the course will be based on the following (approximate) percentages:

\begin{center}
\begin{tabular}{ll}
First Test & 20\%\\
Second Test & 20\%\\
Labs / Assignments & 30\%\\
Final Test & 30\%\\
\end{tabular}
\end{center}

Final test might be in any form, a program, such as project
presentation, a regular test, or a paper. The instructor will make a
decision after a cooperative discussion with the students.

You should do your own work on exams/projects and for computer
assignments. Copying another student’s work is not acceptable. Any
indication of cheating and/or plagiarism on an exam/assignment/project
will be an automatic 0 (zero) for the exam/assignment/project for all
students involved. Yet, based on cheating and plagiarism activity in
any section of class, instructor holds the right to give F grade to
the identified student(s). Regarding codes in assignments / projects,
you may be required to explain the code you submitted. In case of
discursive explanation, the instructor holds the right to lower your
grade. You are given many programming assignments during semester. The
student submitting other’s source code will take -1 for this specific
assignment. A student having more than three -1 cannot gain any point
for the Lab/Assignments section of class evaluation. It means that you
will automatically lose 30\% of your total grade.

Letter grades will be assigned according to the following scale:

\begin{center}
\begin{tabular}{ll}
Final Average (\%) & Letter Grade\\
\hline
90 - 100 & A\\
80 - 89 & B\\
70 - 79 & C\\
60 - 69 & D\\
Below 60 & F\\
\end{tabular}
\end{center}

\section*{Student's with Disabilities}
\label{sec-7}

The Americans with Disabilities Act (ADA) is a federal
anti-discrimination statute that provides comprehensive civil rights
protection for persons with disabilities.  Among other things, this
legislation requires that all students with disabilities be guaranteed
a learning environment that provides for reasonable accommodation of
their disabilities.  If you have a disability requiring an
accommodation, please contact: 

\begin{center}
\begin{tabular}{l}
Office of Student Disability Resourcesand Services\\
Texas A\&M University-Commerce\\
Gee Library, Room 132\\
Phone (903) 886-5150 or (903) 886-5835\\
Fax (903) 468-8148\\
StudentDisabilityServices@tamuc.edu\\
\end{tabular}
\end{center}

\section*{Academic Ethics}
\label{sec-8}

"All students enrolled at the University shall follow the tenets of
common decency and acceptable behavior conducive to a positive
learning environment." (See Student's Guide Handbook, Policies and
Procedures, Conduct). 

Absolutely no usage of laptops and cellular devices (texting and
talking) in class. Talking and other activities that distract/disturb
others in the class will not be tolerated. Instructor holds the right
to ask you leave the classroom any time based on any of disturbing
attitude. Each student should sign the sign-sheet if asked by
instructor. Late student may not be allowed to participate the
lecture.

Ethics also includes the issue of plagiarism, and copying code for
programming/lab assignments is just as serious as any other type of
plagiarism.  If you are caught sharing or using other people's work in
this class, you will receive a 0 grade and a warning on the first
instance.  A subsequent instance will result in receiving an F grade
for the course, and possible disciplinary proceedings.

\section*{Attendance Policy}
\label{sec-9}

Attendance is mandatory. Students are expected to be present at all
class lectures and are responsible for all material covered in class
and assigned in readings.  Likewise lab sessions are mandatory.  Lab
sessions may not be made up, and absences will affect your final
course grade directly for the missed assignments.

\section*{Course Requirements and Deadlines}
\label{sec-10}
Credit will be given for ONLY those exams, programs, and/or projects
turned in no later than the deadline as announced by the instructor of
this class, unless prior arrangement has been made with the
instructor. Late programs / projects / assignments can or cannot gain
partial credit. Credit for late programs / projects /assignments will be
announced with the description of it.

Assignments and projects will be posted in university’s eCollege
communication system. Detailed information will be provided by the
instructor. Students also should turn in their assignment through
eCollege portal. Each student is responsible for the
content/instructions of email communications.

Study: To plan a minimum of three hours of outside preparation for
each hour of class is a safe time allocation for successfully
completing the course.

Quizzes: Unannounced pop-quizzes may be given in class and/or online
through eCollege to help ensure students stay up with assigned
material.


\section*{Course Schedule (Preliminary)}
\label{sec-11}

\tiny
\begin{center}
\begin{tabular}{|p{0.1in}|p{2.2in}|p{2.2in}|}
\hline
\textbf{W} & \textbf{Content} & \textbf{Assignments}\\
\hline
1 & Introduction, Data types, Cin, Cout & Lab assg. to be completed in 20 minutes. Take-home assg: limits of data types.\\
\hline
2 & Decision making, details of loops, string comparison, nested control structures, logical operators, == and = & Lab assg. to be completed in 20 minutes: Find averages of groups of numbers. Take-home assg.: Library inventory.\\
\hline
3 & File Operations, text files, binary files.  How to read and write, randomness.  Loops to read data from file. & Lab assg. to be completed in two hours. Take-home assg: Library inventory using files\\
\hline
4 & Functions: Modular programming, pass by reference, pass by value, ARGV details for main() function, memory details of functions. & Lab assg. to be completed in two hours. Take-home assg: Complex Math operation with functions\\
\hline
5 & Determining the Size of a Data Type, Scope, static variables, function overloading, predefined (system) functions, default arguments & Lab assg. to be completed in two hours. Take-home assg: Student-course exercises to understand given concepts.\\
\hline
6 & Arrays: Using with different data types, index and access to arrays.  Applications to images. & Lab assg. to be completed in one hours. Take-hoem assg: Calculation of standard derivation in number arrays.  Calculations in 3D arrays (TAMU-C logo)\\
\hline
7 & Array operations, add, shift, replace, deleta an element, parallel arrays with different data types (basics of structs) & Lab assg. to be completed in two hours. Take-home assg: Frequency calculation from a text file\\
\hline
8 & Array applications: Search in sorted and unsorted data.  Sort algorithms, efficient array size calculation for a given problem & Lab assg. to be completed in two hours. Take-home assg: Sort one million number to see difference in sort algorithms.\\
\hline
9 & Structs, memory status of structs, arrays in structs, functions with structs. & Lab assg. to be completed in two hours. Take-home assg: Calculation of relative frequency\\
\hline
10 & Structs, structs in structs, sorting structs, sorting structs in an array & Lab assg. to be completed in two hours. Take-home assg: Vanity numbers, a departmental courseware\\
\hline
11 & Dynamic Programming, Pointers, new and delete operators.  Dynamically created arrays, dynamically created pointers. & Lab assg. to be completed in two hours. Take-home assg: Sorting Structs / Arrays using only pointer\\
\hline
12 & Dynamic Programming with various scenarios.  Friendship network, Facebook problem, Traveling salesman problem & Lab assg. to be completed in two hours. Take-home assg: TSP\\
\hline
13 & Dynamic Programming with structs.  Pointers to various datatypes & Lab assg. to be completed in two hours. Take-home assg: Facebook and Movie database\\
\hline
14 & Templates, pointer based strings, C string and string class & Lab assg. to be completed in one hours. Take-home assg: String manipulations\\
\hline
15 & Family tree problem with dynamic programming, recursive functions & Lab assg. to be completed in one hours. Take-home assg: Fibonacci numbers, Factorials\\
\hline
16 & Review for Final exam, practical examples: constrained sorting, DNA search, bank simulation & Lab assg. to be completed in one hours. Take-home assg: Bonus homework: Parsing a computer program\\
\hline
\end{tabular}
\end{center}
\section*{Reading Schedule}
\label{sec-12}

\small
\begin{center}
\begin{tabular}{rl}
\hline
\textbf{W} & \textbf{Reading}\\
\hline
1 & Ch 1 (Introduction), Ch 2\\
\hline
2 & Ch 4, Ch 5\\
\hline
3 & Ch 15, Ch 17\\
\hline
4 & 6.1 - 6.14\\
\hline
5 & 6.15 - 6.17\\
\hline
6 & 7.1 - 7.4\\
\hline
7 & \\
\hline
8 & 7.7 - 7.8\\
\hline
9 & 21.1 - 21.5\\
\hline
10 & Ch 20\\
\hline
11 & 8.1 - 8.7\\
\hline
12 & \\
\hline
13 & 8.8 - 8.13, Ch 19\\
\hline
14 & Ch 14, Ch 18\\
\hline
15 & \\
\hline
16 & \\
 & \\
\hline
\end{tabular}
\end{center}
% Emacs 24.3.1 (Org mode 8.2.4)
\end{document}
